%format latexg
\documentstyle[german,12pt,din_a4,twoside]{article}
%\documentstyle[german,din_a4]{article}

\sloppy          % Mehr Toleranz beim Zeilenumbruch  %
\hfuzz=1pt       % reg dich nicht "uber alles auf...   %
\vfuzz=1pt       % dito...   %
%\fussy          %"druckreife Ausgabe"%

\pagestyle{headings}
\newcommand\exa{\nopagebreak \begin{flushleft}\smallskip \nopagebreak
\begin{minipage}[t]{6cm}\sloppy}
\newcommand\exb{\end{minipage}\kern 1cm\begin{minipage}[t]{8cm}\sloppy }
\newcommand\exc{\end{minipage}\kern -3cm \smallskip\end{flushleft}}
%\nofiles

\begin{document}
  \date{5. Januar 1996}
  \title{Portugiesische Grammatik}
  \author{Christian Ledermann}
 \maketitle
\typeout{Portugiesische Grammatik}
\begin{abstract}
Portugiesisch wird von 150 Millionen Menschen auf 5
Kontinenten gesprochen, in Portugal, in Brasilien, in den
ehemaligen Kolonien Angola, Cabo Verde, Guinea-Bissau,
Mo�ambique und Sao Tom�, in Macau und Goa.

Die Unterschiede zwischen dem Portugiesischen, das in
Portugal gesprochen wird, und dem Portugiesischen in
Brasilien sind nicht unerheblich, jedoch wird jeder, der
das europ�ische Portugiesisch gelernt hat, in Brasilien
sprachlich gut zurechtkommen.

Dieser Text beinhaltet einen Auszug aus der
Portugiesischen Grammatik um den Lernenden der
Portugiesischen Sprache eine kleine und kompakte
Hilfestellung zu geben. Hier werden nur die Grundz�ge
vermittelt, was f�r ein intensives Studium zu wenig ist,
aber absolut ausreichend f�r denjenigen, der sich im
Urlaub verst�ndlich machen m�chte. In den meisten
Lehrb�chern f�r den Einstieg ins Portugiesische wird die
Grammatik �ber das gesammte Buch verteilt in den Lerninhalt
eingestreut was didaktisch wohl Sinn macht, aber zum
Nachschlagen ungeeignet ist. An diesem Punkt setzt die hier
vorliegende \TeX{}-Datei an.
\end{abstract}

\tableofcontents

\appendix
\input LPLZEITN
\input LPLGRAMM.TEX
\end{document}
%-eof
