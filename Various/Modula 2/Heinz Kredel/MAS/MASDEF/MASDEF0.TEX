% version 0.60, 6.3.91 
% title and preface

\title{MAS \\ Modula--2 Algebra System\\ 
       \mbox{ } \\
       Specifications \\ Definition Modules \\ Indexes}

\author{%Heinz Kredel \\
        Computer Algebra Group \\
        University of Passau
        }

\date{MAS Version 0.6\footnote{Document revision \today}
      }

\maketitle

\mbox{  }

\vfill

\section*{Copyrights}

The MAS system was developed using several 
public domain programs. 
So permission is granted for unrestricted use of MAS 
as long as the copyrights are preserved.
The copyrights are:
\begin{quote}
MAS: \copyright 1989, 1990, 1991, by H. Kredel, University Passau. \\
ALDES / SAC--2: \copyright 1982, by G.E.Collins and R.Loos.
\end{quote}
However remember:
{\em 
We make no warranty and disclaim any usefulness. 
Use this program at your own risk. 
}
If you find a bug in MAS, please let us know. 
Also any suggestions will be welcome. 
Most of the MAS system is available via anonnymous ftp 
from internet address \verb/132.231.10.1/ = 
\verb/alice.fmi.uni-passau.de/.


\chapter*{Preface}

This document contains summarys and indexes 
which are useful for working with the MAS system.
The plan of this document is as follows:

Chapter 1 
gives an short introduction in using the 
different informations containted in this document.

Chapter 2 
contains the specifications of 
several algebraic structures.

Chapter 3
lists all Modula--2 definition modules of 
the ALDES /SAC--2, the DIP and the MAS libraries.

Chapter 4
contains a algorithm comment to procedure name index.

Chapter 5
consists of an procedure header to definition module index.
 
The last part is an procedure name index of this document.  


\section*{Acknowledgements}

Many thanks to all who made contributions or influenced this project:
R. Loos, G. E. Collins and co--workers for the 
ALDES / SAC--2 system;
I. Giese, W. Kynast and H. Czytrek for discussions in the
early stages of the MAS project;
M. Pesch, B. Haible,
V. Weispfenning and T. Becker  
for feedback during 
the later stages of the development of MAS. 
In the current version also contributions from students 
are incorporated:
B. Deyle, H. Friebel, E. Reisinger, 
M. Rothmeier, K. Rieger, T. Wollersberger,
J. M\"uller. 

The MAS system was first developed using the 
following hard-- and software: 
Computer: Atari 1040 ST, (1 MB) 1986.  
Compiler: TDI: Modula--2/ST, Clifton, Bristol, UK, 1986.  
Editor: micro EMACS, by Daniel Lawrence, 1988.  
Shell: Gul\"am by Prabhaker Mateti at Case, 1988.  

\begin{center}
Passau, \today. \hfill H. Kredel\footnote{University of Passau, 
                                 Innstra\ss e 33, D-8390 Passau, FRG,
                                 Tel: (49,0) 851/ 509-315,
                                 E-mail: kredel @ 
                                 unipas.fmi.uni-passau.de}
\end{center}

