\section{ DIP Ideal Decomposition 0 System  } 
\proc{DIGFET} (P,IL,JL: LIST): LIST; \eproc
\bcom DIP G base successful extension test. P is a Groebner base of
an ideal of dimension 0 in inverse lexicographical term ordering. 
i and j are indexes of variables where an field extension is required. 
t=1 if the extension was successful t=0 else.  \ecom 
\proc{DIGISM} (P: LIST): LIST; \eproc
\bcom DIP G base index search for extension multiple univariats.
P is a Groebner base of dimension 0 in inverse lexicographical term 
ordering. I is a list of indexes of variables where an field extension
is required or I=() if no field extension is neccessary.  \ecom 
\proc{DIGISR} (P: LIST): LIST; \eproc
\bcom DIP G base index search for extension reductas. P is a Groebner
base of an ideal of dimension 0 in inverse lexicographical term 
ordering. I is a list of indexes of variables where an field 
extension is required or I=() if no field extension is neccessary.  \ecom 
\proc{DINTFE} (T,IL,JL: LIST): LIST; \eproc
\bcom DIP normalized tupel field extension. T is a normalized tupel
of a zero set with a final Groebner base of dimension 0.
i and j determine the variable indexes for the field extension.
TP is a list of normalized tupels for the field extension for T. 
Trial values are used for the transcendent parameter.  \ecom 
\proc{DINTSR} (T: LIST): LIST; \eproc
\bcom DIP normalized tupel separation refinement. T is a list of
normalized tupels with final Groebner base of dimension 0.
TP is a list of normalized tupels for some field extensions for T.  \ecom 
\proc{DINTSS} (T: LIST): LIST; \eproc
\bcom DIP normalized tupel strong separation. T is a list of normalized
tupels with final Groebner base of dimension 0. TP is a list of 
normalized tupels for some field extensions for T.  \ecom 
\proc{DINTZS} (N: LIST): LIST; \eproc
\bcom DIP nomalized tupels from system zero. N is a zero set.
T is the list of nomalized tupels of N.  \ecom 
\proc{DIRGZS} (VB,PB,W: LIST): LIST; \eproc
\bcom Distributive rational Groebner base zero set. VB is a rest of a
variable list. PB is a Groebner base. W is the total variable 
list. N is the zero set of P.  \ecom 
\proc{DIRLPD} (A,VP: LIST): LIST; \eproc
\bcom DIP rational polynomial ideal primary ideal decomposition.
A is a non empty list of distributive rational polynomials 
representing a Groebner base. The polynomials in A have r variables.
L=(l1, ...,ln) with li=(pi,ei,vpi,qi) i=1, ...,n where 
qi = ideal(pi**e,A) with A contained in qi and e maximal.
Ideal(pi) is a prime ideal in at most r+1 variables. VPI is the 
variable list vor pi.  \ecom 
\proc{DIRLPW} (A,V,L: LIST); \eproc
\bcom DIP rational polynomial ideal primary ideal decomposition write.
A is a non empty list of distributive rational polynomials 
representing a Groebner base. The polynomials in a have r variables.
L=(l1, ...,ln) with li=(pi,eli,vpi,qi) i=1, ...,n where 
qi = ideal(pi)**e with A contained in qi and e maximal.
Ideal(pi) is a prime ideal in at most r+1 variables. VPI is the 
variable list for pi.  \ecom 
\proc{DIRPDA} (A,VP: LIST): LIST; \eproc
\bcom DIP rational polynomial ideal primary ideal decomposition over Q(alpha).
A is a non empty list of distributive rational polynomials 
representing a Groebner base. The polynomials in A have r variables.
L=(l1, ...,ln) with li=(pi,ei,vpi,qi) i=1, ...,n where 
qi = ideal(pi**e,A) with A contained in qi and e maximal.
Ideal(pi) is a prime ideal in at most r+1 variables. VPI is the 
variable list vor pi.  \ecom 
\proc{DITFZS} (N: LIST): LIST; \eproc
\bcom DIP tupel from zero set. N is a zero set. T is a list of
tupels of the zero set.  \ecom 
\proc{DITSPL} (T: LIST; VAR T0,T1: LIST); \eproc
\bcom DIP zero set tupel split. T is a list of normalized tupels
of a zero set. T0 is a list of normalized tupels of a zero set 
with a final Groebner base of a ideal of dimension 0. T1=T-T0.  \ecom 
\section{ DIP Dimension  } 
\proc{DIGBZT} (S: LIST): LIST; \eproc
\bcom Distributive polynomial groebner base common zero test.
S is a groebner basis. t = -1 or 0 if DIMENSION(S) eq -1 or 0, t = 1
if DIMENSION(S) ge 1.  \ecom 
\proc{DILDIM} (G: LIST; VAR DL,S,M: LIST); \eproc
\bcom Distributive polynomial list dimension.
G is a list of distributive polynomials, a groebner base.
d is the dimension of ideal(G). M is a maximal independend
set of variables. S is a set of maximal independent sets of
variables.  \ecom 
\proc{DIDIMS} (G,S,U,M: LIST): LIST; \eproc
\bcom Distributive polynomial dimension maximal independent set.
G is a list of distributive rational polynomials, and a g-base.
S is a maximal independent set of variables.
U is a set of variables of unknown status.
M and MP are lists of maximal independent sets of variables.  \ecom 
\proc{EVGBIT} (S,G: LIST): LIST; \eproc
\bcom Exponent vector groebner base intersection test.
S is a set of variable indices. G is a groebner basis.
t = 0 if intersection = () else t = 1.  \ecom 
\proc{USETCT} (U,V: LIST): LIST; \eproc
\bcom Unordered set containment test. U and V are unordered sets.
t = 1 if U is contained in V else t = 0.  \ecom 
\proc{IXSUBS} (V,I: LIST): LIST; \eproc
\bcom Indexed subset. V is a list.
I is an index list. The elements of V with index from I
are put to VP.  \ecom 
\section{ DIP GCD  } 
\proc{DIRFAC} (P: LIST): LIST; \eproc
\bcom Distributive rational polynomial factorisation.
P is a distributive rational polynomial.
PP=((e1,P1), ...,(en,Pn)), where P=P1**e1+ ... +Pn**en.  \ecom 
\proc{IPLCM} (RL,A,B: LIST): LIST; \eproc
\bcom Integral polynomial least common multiple. A and B are integal
polynomials. C=LCM(A,B), a nonnegative integral polynomial. \ecom 
\section{ DIP Ideal System  } 
\proc{DIPLDV} (A,V: LIST): LIST; \eproc
\bcom Distributive polynomial list dependency on variables.
A is a list of distributive polynomials. V is the variable list.
U is the variable list of variables with positive exponents in A.  \ecom 
\proc{DIRLCT} (A,B: LIST): LIST; \eproc
\bcom Distributive rational polynomial list ideal containement test.
A and B are lists of distributive rational polynomials representing 
groebner bases. t = 1 if ideal(A) is contained in ideal(B),
t = 0 else.  \ecom 
\proc{DIRLIP} (PL,A,B: LIST): LIST; \eproc
\bcom Distributive rational polynomial list ideal product.
A and B are lists of distributive rational polynomials. 
C=GBASIS(p,A*B). \ecom 
\proc{DIRLPI} (A,P,VP: LIST): LIST; \eproc
\bcom Distributive rational polynomial list primary ideal.
A and P are non empty lists of distributive rational polynomials 
representing groebner bases. The polynomials in A have r variables.
ideal(P) is a prime ideal in at most r+1 variables. VP is the 
variable list for P. QP=(P,e,VP,Q) where Q = ideal(P**e,A) 
with A contained in Q and e maximal.  \ecom 
\section{ DIP Rational Function  } 
\proc{IFWRIT} (R,V: LIST); \eproc
\bcom Integral function write. R is an integral function.
R is the variable list. R is written in the output stream.  \ecom 
\proc{RFDEN} (R: LIST): LIST; \eproc
\bcom Rational function denominator. R is a rational function.
BL is the denominator of R, a positive integral polynomial
in RL variables.  \ecom 
\proc{RFDIF} (R,S: LIST): LIST; \eproc
\bcom Rational function difference. R and S are rational functions.
T=R-S.  \ecom 
\proc{RFEXP} (A,NL: LIST): LIST; \eproc
\bcom Rational function exponentiation. A is a rational function,
n is a non-negative beta-integer. B=A**n.  \ecom 
\proc{RFFIP} (RL,A: LIST): LIST; \eproc
\bcom Rational function from integral polynomial. A is an integral
polynomial in RL variables. R is the rational function A/1.  \ecom 
\proc{RFINV} (R: LIST): LIST; \eproc
\bcom Rational function inverse. R is a non-zero rational
function. S=1/R.  \ecom 
\proc{RFNEG} (R: LIST): LIST; \eproc
\bcom Rational function negative. R is a rational function. S=-R.  \ecom 
\proc{RFNOV} (R: LIST): LIST; \eproc
\bcom Rational function number of variables. R is a rational
function. RL is the number of variables of the numerator
and denumerator of R.  \ecom 
\proc{RFNUM} (R: LIST): LIST; \eproc
\bcom Rational function numerator. R is a rational function.
AL is the numerator of R, an integral polynomial.  \ecom 
\proc{RFONE} (R: LIST): LIST; \eproc
\bcom Rational function one. R is a rational function. s=1 if R=1,
s=0 else.  \ecom 
\proc{RFPROD} (R,S: LIST): LIST; \eproc
\bcom Rational function product. R and S are rational functions.
T=R*S.  \ecom 
\proc{RFQ} (R,S: LIST): LIST; \eproc
\bcom Rational function quotient. R and S are rational functions,
S non-zero. T=R/S.  \ecom 
\proc{RFREAD} (V: LIST): LIST; \eproc
\bcom Rational function read. The rational function R is read
from the input stream. V is the variable list. any preceding
blanks are skipped.  \ecom 
\proc{RFRED} (RL,A,B: LIST): LIST; \eproc
\bcom Rational function reduction to lowest terms. A and B are
integral polynomials in RL variables, B non-zero. R is the
rational function A/B in canonical form.  \ecom 
\proc{RFSIGN} (R: LIST): LIST; \eproc
\bcom Rational function sign. R is a rational function. s=sign(R).  \ecom 
\proc{RFSUM} (R,S: LIST): LIST; \eproc
\bcom Rational function sum. R and S are rational functions. T=R+S.  \ecom 
\proc{RFWRIT} (R,V: LIST); \eproc
\bcom Rational function write. R is a rational function. V is the
variable list. R is written in the output stream.  \ecom 
\section{ DIP Rational Groebner Bases  } 
\proc{DIGBC3} (B,PLI,PLJ,EL: LIST): LIST; \eproc
\bcom Distributive polynomial groebner basis criterion 3.
B is a non empty list of reduction sets. pi and pj are
distributive polynomials. e is the least common multiple
of the leading exponent vectors of pi and pj. s=1 if the
reduction of pi and pj is necessary s=0 else.  \ecom 
\proc{DIGBC4} (PLI,PLJ,EL: LIST): LIST; \eproc
\bcom Distributive polynomial groebner basis criterion 4.
pi and pj are polynomials in distributive representation.
e is the least common multiple of the leading exponent vectors
of pi and pj. s=1 if the reduction of pi and pj is necessary
s=0 else.  \ecom 
\proc{DIGBMI} (P: LIST): LIST; \eproc
\bcom Distributive minimal ordered groebner basis. P is a list of
non zero rational polynomials in distributive representation
in r variables. PP is the minimal normed and ordered
groebner basis.  \ecom 
\proc{DILCPL} (P: LIST; VAR D,B: LIST); \eproc
\bcom Distributive polynomial list construct pair list.
P is list of polynomials in distributive representation
in r variables. B is the polynomials pairs list and
D is the pairs list.  \ecom 
\proc{DILUPL} (PL,P,D,B: LIST): LIST; \eproc
\bcom Distributive polynomial list update pair list.
P is list of polynomials in distributive representation
in r variables. B is the polynomials pairs list and
D is the pairs list. p is a non zero polynomial in
distributive representation. D, P and B are modified.
DP is the updated pairs list.  \ecom 
\proc{DIRGBA} (PL,P,TF: LIST): LIST; \eproc
\bcom Distributive rational polynomial groebner basis augmentation.
P is a groebner basis of polynomials in distributive
representation in r variables. p is a polynomial. PP is the
groebner basis of (P,p). t is the trace flag. \ecom 
\proc{DIRGBR} (P,TF: LIST): LIST; \eproc
\bcom Distributive rational polynomial groebner basis recursion.
P is a list of rational polynomials in distributive representation
in r variables. PP is the groebner basis of P. t is the
trace flag. \ecom 
\proc{DIRLIS} (P: LIST): LIST; \eproc
\bcom Distributive rational polynomial list irreducible set.
P is a list of distributive rational polynomials,
PP is the result of reducing each p element of P modulo P-(p)
until no further reductions are possible.  \ecom 
\proc{DIRPGB} (P,TF: LIST): LIST; \eproc
\bcom Distributive rational polynomials groebner basis.
P is a list of rational polynomials in distributive representation
in r variables. PP is the groebner basis of P. t is the
trace flag. \ecom 
\proc{DIRPNF} (P,S: LIST): LIST; \eproc
\bcom Distributive rational polynomial normal form. P is a list
of non zero polynomials in distributive rational
representation in r variables. S is a distributive rational
polynomial. R is a polynomial such that S is reducible to R
modulo P and R is in normalform with respect to P.  \ecom 
\proc{DIRPSP} (A,B: LIST): LIST; \eproc
\bcom Distributive rational polynomial S polynomial. A and B are
rational polynomials in distributive representation. C is
the S polynomial of A and B.  \ecom 
\proc{EVPLM} (L1,L2: LIST): LIST; \eproc
\bcom Exponent vector pair-list merge. L1 and L2 are pair-lists
of exponent vectors in non decreasing order.  L is the merge 
of L1 and L2. L1 and L2 are modified to produce L.  \ecom 
\proc{EVPLSO} (A: LIST): LIST; \eproc
\bcom Exponent vector pair-list sort. A is a list of pair-lists. B is
the result of sorting A into non-decreasing order. Pairs of
elements of A are merged. The list A is modified to produce B.  \ecom 
\section{ DIP Ideal Real Root System  } 
\proc{DIGBSI} (P,T,A: LIST): LIST; \eproc
\bcom Distributive polynomial system algebraic number G basis sign.
P is a goebner basis in inverse lexicographical term order
in r variables (non empty), with all neccessary refinements.
T=(t1,... ,ti) i le r, where tj=(vj,ij,pj) j=1, ...,i and v is 
the character list for the j-th variable, ij is an isolating 
intervall for a real root of the univariate polynomial pjl.
A is a distributive rational polynomial depending maximal on one 
variable. s is the sign of A as element of an algebraic extension
of Q determined by P.  \ecom 
\proc{DIITNT} (T: LIST): LIST; \eproc
\bcom Distributive polynomial system intervall tupel from norm tupel.
T is a refined normalized tupel of a zero set with a final Goebner 
base of dimension 0. TP is a list of intervall tupels for T.  \ecom 
\proc{DIITWR} (TP,EPS: LIST); \eproc
\bcom Distributive polynomial system intervall tupels write. TP is a list
of intervall tupels of a zero set. EPS is LOG10 of the desired 
precision.  \ecom 
\proc{DINTWR} (TP,EPS: LIST); \eproc
\bcom Distributive polynomial system normalized tupels write.
TP is a list of normalized tupels of a zero set. EPS is log10 of 
the desired precision.  \ecom 
\proc{DIROWR} (V,P,EPS: LIST); \eproc
\bcom Distributive polynomial system real root write. V is a variable
list. P is a list (e,p). EPS is the desired precision. e is the 
multiplicity of the root, and p is an irreducible polynomial.  \ecom 
\proc{GBZSET} (V,PP,EPS: LIST); \eproc
\bcom Groebner base real zero set of zero dimensional ideal.
V is a variable list. PP is a list of distributive rational polynomials,
PP is a Groebner base. EPS is is LOG10 of the desired precision.  \ecom 
\proc{RIRWRT} (R,EPS: LIST); \eproc
\bcom Rational intervall refinement write. R=(v,i,p) where v is the
variable character string, i is a rational intervall containing only
one real root of the polynomial p. EPS is the presicion epsilon.  \ecom 
\section{ DIP Zero Dimensional Ideal  } 
\proc{DIRMPG} (IL,F: LIST): LIST; \eproc
\bcom Distributive rational minimal polynomial for a groebner basis.
F is a groebner basis. PP is the minimal polynomial for the
i-th variable for F.  \ecom 
