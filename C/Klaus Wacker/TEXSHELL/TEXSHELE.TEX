% macropackage=lplain
\documentstyle[twoside]{article}    
%\textheight 16.5cm    
%\textwidth 11cm
\textheight 630pt 
\textwidth 15cm 
\oddsidemargin 0.5cm 
\evensidemargin 0.5cm 
% \evensidemargin 0.3cm
\topmargin 0pt
\pagestyle{myheadings}
\begin{document}    
\title{TEXSHELL}
\author{Klaus Wacker}    
\date{6 March 1988}  
\markboth{TEXSHELL}{TEXSHELL}
\maketitle
\section*{Overview}    
{\tt TEX\-SHELL} is
a GEM-shell for TooLs ST-\TeX\ Version 2.1.

{\tt TEX\-SHELL} supports the development cycle of a 
\TeX-document,
i.e., editing --- \TeX\ --- DVI --- editing and so on.
Following are the advantages compared to calling {\tt TEX.TTP} from the 
desktop:
\begin{itemize}

\item 
The name of the \TeX-file needs to be entered only once.

\item
The macropackage used (e.g., plaing, lplain etc.) can
be selected from the menu or can be given in the \TeX-file
(with a line of the form {\tt \%\ macropackage=xxxxx}).
``Plain \TeX '' as well as \LaTeX\ as well as other
formats are supported. 

\item
The \TeX-files can be kept in folders without problem.
{\tt TEX\-SHELL} changes backslashes in the folder path into forward 
slashes before calling {\tt TEX.TTP}.

\item
Handling of the output and auxiliary files generated by \TeX\
is supported.
\end{itemize}

\section*{Description}

\subsection*{The Beginning}

Copy {\tt TEX\-SHELL.\-PRG}, {\tt TEX\-SHELE.\-RSC} and
{\tt TEX\-SHELE.\-TXT} onto your working floppy disk or harddisk.
You may copy them into a folder, but all three into the same.
Then rename
{\tt TEX\-SHELE.\-RSC} and
{\tt TEX\-SHELE.\-TXT} into
{\tt TEX\-SHELL.\-RSC} and
{\tt TEX\-SHELL.\-TXT}.
If you prefer the German version, use
{\tt TEX\-SHELL.\-RSC} and
{\tt TEX\-SHELL.\-TXT} as is\footnote{
If you prefer yet another language, it is sufficient to change
the {\tt .TXT}-
and {\tt .RSC}-files, the latter with a resource-editor.
In the {\tt .TXT}-file the number and order of lines must not be changed
under any circumstances. 
}. 

At startup, {\tt TEX\-SHELL} tries to read the files {\tt TEX\-SHELL.\-INF}
and {\tt ENVI\-RON\-M.\-ENT}.
These files will probably not be found at first use, you should
then select the ``Locate''-menu first.
Locate your favourite editor as well as \TeX\ and DVI and then select
``Save settings''. Thus, this information is saved in the 
new file {\tt TEX\-SHELL.\-INF}. 
The {\tt ENVI\-RON\-M.\-ENT}-file is read from the same path as
{\tt TEX.TTP} as soon as that is located.
If you have not yet created this file, you will find in the following
some hints about what should be in there. 
You should also consult the relevant chapter in the
ST-\TeX-manual.

\subsection*{The File Menu}
\begin{description}

\item[Select TeX-file\ldots]
Here you select with a file selector box the
\TeX-file you intend to work with.
The path of this file does not have to be included in the environment 
variables {\tt TEX\-IN\-PUTS} or {\tt TEX\-READ} as 
the complete filespecification is passed to \TeX.
This is not true if further files are needed from this path
(see under {\bf Edit\ldots}).

When the \TeX-file exists, up to 10 lines will be read from it
in order to find a line of the form
\begin{verbatim} % macropackage=xxxxx \end{verbatim}
{\tt xxxxx} stands for the first name of a file with last name
{\tt .FMT}, which must be in the {\tt TEXFORMATS}-path.
If such a line is found, the appropriate options
in the ``Options''-menu are set. The \%-character must be the first
non-white
character in the line. 
There must be no blanks before and after the equal-sign.
I have adopted these conventions from a mainframe installation of \TeX,
I hope it is some kind of standard. 

The same thing as when you click on ``Select TeX-file\ldots''
happens when no \TeX-file was selected before the first time you click on
``Edit TeX file'' or ``TeX''.

\item[Erase file\ldots]

\item[Rename file\ldots]

\item[Copy file\ldots]
These three menu items have nothing directly to do with
\TeX, they are there just for convenience.

\end{description}

The following two menu items only make sense, if you keep 
your \TeX-file in a different folder than the files generated by
\TeX. The folder paths of the latter are defined by the
environment variables {\tt TEX\-OUT\-PUT},
{\tt TEX\-TRAN\-SCRIPT} and {\tt TEX\-WRITE}. 
These three paths (which can be identical)
are considered here as temporary storage, the folder of
the \TeX-file as permanent.

\begin{description}

\item[Save AUX etc.]
The auxiliary files generated by \TeX\ are copied from the path
{\tt TEX\-WRITE} into the folder of the \TeX-file.
You will be asked for every file found whether you really
want that. 
To be precise, the following happens here: 
All files in the path {\tt TEX\-WRITE}
are considered which were created during the time 
\TeX\ ran and whose last name is {\em not}
{\tt .DVI} or {\tt .LOG}.

Especially \LaTeX\ likes to generate
files like that (e.g., with last name {\tt .AUX}).
It may happen that an {\tt .AUX}-file is generated
which just contains the word ``\verb|\relax|''.
You may well erase something like that.
However, if you use automatic numberings, references,
tables of contents and the like,
it is important to keep them.
The path of these files has to be given in {\tt TEXINPUTS} and {\tt TEXREAD}.
List here first the {\tt TEXWRITE} path,
then the path of the \TeX-file. 
Thus, \TeX\ always finds the latest version of the auxiliary files,
independent of when you last saved them.

\item[Save DVI]
The DVI-file generated by \TeX\ is copied from the {\tt TEX\-OUT\-PUT} path
into the folder of the \TeX-file. 
Here, {\tt TEX\-SHELL} does not check the time since the name is unique.
You will be asked once more whether you really want that.

(There is no menu item {\bf Save LOG}. 
I assume
that after finishing the work nothing interesting remains in the
{\tt .LOG} file.
Until then you may as well keep it in the
{\tt TEXTRANSCRIPT} path.)

\item[Clean disk]
The files generated by \TeX\ are erased.
You will be asked for every file whether you really want to erase it.
Here also {\tt TEX\-SHELL} uses the names of the {\tt .DVI} and {\tt .LOG} 
files. The time is used for the files in the {\tt TEX\-WRITE} path.

\item[Quit]
{\tt TEX\-SHELL} ends here.
\end{description}

\subsection*{The Execute Menu}

\begin{description}

\item[Edit TeX file]
The Editor is called with path and name of the \TeX-file
and the {\tt .LOG} file, if it exists.
After editing, as after selecting, the \TeX-file is searched
for a {\tt macro\-package} line,
in case you have changed or added such a line.

\item[Edit\ldots]
The Editor is called for a file which you select
with a file selector box.
This menu item is meant for the case that
your \TeX\ input consists of several files.
You select the root file with the menu item ``Select TeX file''.
It contains \verb|\input| or \verb|\include| statements which call up subfiles. 
Use the ``Edit\ldots'' menu item to edit these subfiles.

The path of these files has to be listed in the environment variable 
{\tt TEX\-IN\-PUTS}, in case of \verb|\include| also in {\tt TEX\-READ}.

\item[TeX]
Execute \TeX.

\item[DVI]
Execute DVI. You will be warned when \TeX\ has not been executed
at all or 
after the last editing.

\item[Other program\ldots]
Here you can execute any program you like.
For {\tt .TTP}-programs you will be asked for parameters.
\end{description}

Whenever {\tt TEX\-SHELL} calls a program, ``Current Drive''
and ``Current Path'' are set to the drive and path of the program called,
so that it finds its resource files etc.
Thus, you are completely free to arrange your programs, in particular
{\tt TEX\-SHELL}, the editor, \TeX\ and DVI, on drives and folders.
 
\subsection*{The Locate Menu}
\begin{description}

\item[Editor\ldots]
Locate the editor. 
The editor must be able to understand a command line with the full name
(including drive and path) of the file to be worked on.
You will be asked whether it can work on at least two files at the same 
time. If yes, the editor will be given the name of the 
{\tt .LOG} file, if it exists, in addition to that of the \TeX-file.

\item[TeX\ldots]
Locate {\tt TEX.TTP}.

\item[DVI\ldots]
Locate {\tt DVI.PRG}.

\item[TeX-Environment]
Edit the {\tt  ENVI\-RON\-M.\-ENT} file.
Afterwards the parameters are also read by {\tt TEX\-SHELL}.

\item[Save settings]
The settings done in the locate and options menu
are stored in the
file {\tt TEX\-SHELL.\-INF} in the path from which
{\tt TEXSHELL.PRG} was called.
\end{description}

\subsection*{The Options Menu}
\begin{description}

\item[TeX calls DVI]
If set, DVI is called automatically after \TeX.

\item[Wait after TeX]
If set, {\tt TEX\-SHELL} waits after execution
of \TeX\ (and other {\tt .TOS}- and {\tt .TTP}-programs) for
a keystroke.
The ``Return Code'' is displayed.
\end{description}

The following four options are set automatically
if a {\tt macro\-package}-line
is present in the \TeX-file.
You may also set them by hand.

At most one of the following three options can be selected.
If you select one, the other two are automatically
deselected.

\begin{description}

\item[Plain]
Use {\tt PLAIN.FMT} or {\tt PLAING.FMT}.

\item[LaTeX]
Use {\tt LPLAIN.FMT} or {\tt LPLAING.FMT}\footnote{
By editing {\tt TEXSHELL.TXT}, you can use other
format files here.}.

\item[Other FMT\ldots]
You select an {\tt .FMT}-file with a file selector box.
The ``German'' option has no effect in this case
(of course, you may write in German nevertheless).

\item[German]
Use  {\tt PLAING} resp.\ {\tt LPLAING}
instead of {\tt PLAIN} resp.\ {\tt LPLAIN}. 
\end{description}

\section*{Small Print}
{\small
I have nothing to do with
TooLs GmbH except that I bought
ST-\TeX\ from them.

{\tt TEX\-SHELL} is of no use for you unless you have
TooLs ST-\TeX\ Version 2.1 (or higher ?).
You get it from TooLs GmbH, Kessenicher Str.~108, D-5300 Bonn 1.

When developping {\tt TEX\-SHELL}, I have partly used {\tt FSHELL}
by Daniel Roth as a model (PD disk number 100 of the 
magazine ``ST Computer'').

Noncommercial distribution of 
{\tt TEX\-SHELL} in complete and unchanged form
is encouraged.
I however reserve the copyright and all rights.
Use this program at your own risk.
}

If you find a bug or have suggestions, please contact
{\tt WACKER@CERNVM} or
\begin{verse}
Klaus Wacker\\
32, chemin de la Planche Br\^ul\'ee\\
F-01210 Ferney Voltaire.\\
\end{verse}

\end{document}
