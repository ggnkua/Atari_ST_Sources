% ASCREEN-3.2-HANDBUCH --- Anhang E (Fontlist)
% Copyright (C) 1992 by Anselm Lingnau
%
\chapter{\FL}

\section{Was ist \FL?}

Das \ASCREEN-Paket enth�lt auch noch eine weitere Dreingabe:
Das Programm \FL\ gibt eine Liste
der in einem Dokument benutzten Zeichens�tze aus.
Neben dem "`Interesse-Wert"' einer solchen Liste
hat das auch praktischen Nutzen:
wenn Sie ein Dokument weitergeben wollen,
k�nnen Sie mit der Liste sichergehen,
da� keine exotischen Zeichens�tze darin vorkommen,
oder Sie k�nnen anhand der Liste die Zeichens�tze zum Dokument mitliefern.
Wie die Liste genau aussehen soll,
k�nnen Sie selbst festlegen ---
\FL\ kennt von Haus aus den Lindner- und den Strunk-Stil
f�r Zeichensatzlisten, den das respektive \MF\ verstehen kann (naja,
die HKM-\TeX shell bei Stefan Lindner,
und das \MF\ ist ja auch von Lutz Birkhahn, und �berhaupt\dots),
aber Sie k�nnen auch �ber ein Format \`a la {\tt printf()}
eine nahezu beliebige Form der Ausgabe bestimmen.

\section{\FL\ benutzen}

\FL\ ist ein TOS-Programm,
das Sie am besten von einer kommandoorientierten Shell
wie der {\sl Mupfel} aus aufrufen.
Die grundlegende Syntax ist
\begin{verse}
{\tt fontlist} [\<Optionen>] \<DVI-Datei>
\end{verse}
wobei die Liste auf der Standardausgabe landet.
Diese kann wie �blich in eine Datei umgelenkt werden.

Die folgenden Optionen sind definiert:
\begin{description}
\item[-l] Liste im Lindner-Stil ({\tt res101.scr cmr10 1.200})
\item[-s] Liste im Strunk-Stil (\verb*|cmr10 in magnification 1.200|)
\item[-f] \<Format> Formatangabe f�r die Liste (siehe unten)
\item[-e] \<Endung> Ger�teendung f�r den Lindner-Stil (o. �.)
\item[-r] \<Auf"|l�sung> Auf"|l�sung f�r den Lindner-Stil (o. �.)
\item[-m] \<Zahl> Vergr��erung des ganzen Dokuments (in Promille)
\item[-o] \<Datei> Name einer Ausgabedatei (anstatt {\em stdout\/})
\end{description}
Im Format, das mit {\tt -f} angegeben wird,
k�nnen beliebige ASCII-Zeichen vorkommen.
Einige Sequenzen haben jedoch eine besondere Bedeutung:
\begin{center}
\begin{tabular}{cl}
\tt \%F & Zeichensatzname \\
\tt \%R & Auf"|l�sung (siehe {\tt-r}-Option) \\
\tt \%M & Vergr��erung (als Flie�kommazahl) \\
\tt \%m & Vergr��erung (als Ganzzahl) \\
\tt \%E & Ger�teendung (siehe {\tt-e}-Option)
\end{tabular}
\end{center}
Das Format f�r den Lindner-Stil w�re beispielsweise
\begin{verse}
\verb*|res\%R.\%E %M %F|
\end{verse}
das f�r den Strunk-Stil
\begin{verse}
\verb*|%F in magnification %M|
\end{verse}
Geben Sie kein Format an, so ist die Vorgabe ein schlichtes
\begin{verse}
\verb*|%F %M|
\end{verse}

Mehr mu� �ber das Programm \FL\ an sich hier wohl nicht gesagt werden.
Der Quellcode ist mitgeliefert,
und \FL\ sollte auch auf anderen Computern ohne gro�e Probleme laufen,
wenn es dort einen einigerma�en zum ISO-Standard konformen C-Compiler gibt
(ich habe es auf einem IBM RISC System/6000 getestet).
Falls jemand weitere sch�ne Eigenschaften in das Programm einbaut,
w�re ich selbstverst�ndlich sehr an der neuen Version interessiert.

\begin{chapterquote}
It is clear that Lists are very general structures;
indeed, it seems fair to state
that any structure whatsoever can be represented as a List
when appropriate conventions are made.
\author DONALD E. KNUTH, {\sl Fundamental Algorithms\/} (1973)
\end{chapterquote}
