% ASCREEN-3.2-HANDBUCH --- Kapitel 1 (Installation)
% Copyright (C) 1992 by Anselm Lingnau
%
\chapter{Installation}
\pagenumbering{arabic}

\section{Wichtige Vorbemerkung}

Willkommen zu \ASCREEN!
Sie haben sich also \ASCREEN\ und diese Anleitung besorgt
und sind kaum noch zu bremsen,
weil Sie endlich sehen wollen, ob sich die M�he gelohnt hat.
Gleich werden Sie es wissen.
Nehmen Sie sich aber zuerst etwas Zeit
und machen Sie eine Sicherheitskopie der \ASCREEN-""Diskette,
nur f�r den Fall,
da� die Kaffeetasse umkippt, der Hamster ausbricht
oder die Kinder ausprobieren,
was der Men�punkt "`Diskette formatieren\dots"' so alles anstellt.
Wo Sie dabei sind, machen Sie doch gleich ein paar Extrakopien
f�r Ihre Verwandten, Freundinnen und Freunde.
Bis gleich also!

\section{Installation --- Eine halbe Anleitung}

Jetzt sind wir soweit: \ASCREEN\ kann installiert werden.
Die Anleitung zur vorigen Version erkl�rte hier haarklein
den Installationsvorgang f�r die \TeX-""Version von Stefan Lindner.
Heute gibt es aber mindestens vier allgemein verbreitete
Implementierungen von \TeX\ f�r den Atari~ST
(dazu kommen wohl nochmal so viele,
die gewisserma�en "`regionale Bedeutung"' haben),
und \ASCREEN\ vertr�gt sich potentiell mit allen.
Es w�re wohl hochgradig unfair,
die Benutzerinnen und Benutzer einer einzigen Implementierung
so zu bevorzugen,
und darum gebe ich nur einige eher allgemeine Hinweise.

Die Installation besteht aus den folgenden Schritten:
\begin{enumerate}
\item Das \ASCREEN-Programm ({\tt ASCREEN.PRG})
	und seine Resourcedatei ({\tt ASCREEN.RSC})
	m�s"-sen an einen geeigneten Platz auf der Festplatte kopiert werden.
	(Gibt es \TeX-Systeme auf Diskettenbasis? Unwahrscheinlich.)
	Der erste Vorschlag f�r einen solchen Platz w�re "`in die N�he
	des standardm��ig mitgelieferten Previewers"'. Dieser hei�t
	zum Beispiel {\tt SCREEN.TTP} (bei Stefan Lindner) oder
	{\tt DVI\_ST.PRG} (bei Christoph Strunk),
	aber auch etwas wie {\tt PREVIEW.PRG} sieht wahrscheinlich
	aus. Kopieren Sie \ASCREEN\ einfach in dasselbe Verzeichnis.
\item \ASCREEN\ mu� an das verwendete \TeX-System angepa�t werden.
	Sie k�nnen dazu entweder die mitgelieferte Datei {\tt ASCREEN.SET}
	mit einem beliebigen \TeX t-Editor bearbeiten
	(der, mit dem Sie Ihre Dokumente schreiben, ist bestens geeignet)
	oder \ASCREEN\ vom Desktop aus starten (doppelt anklicken)
	und die Einstellungen �ber die Dialoge des Programms vornehmen.
	Insbesondere der Dialog \M{Pfade...} ist hierbei interessant.
	Der einzige etwas trickreiche Punkt betrifft die Einstellung
	der Schablone f�r Zeichensatzdateinamen;
	Kapitel~\ref{k:setup} erkl�rt dies genauer.
\item Inhaber einer \TeX-Shell,
	etwa der {\em \TeX shell\/} von Heidrich, Kie�ling und Maluschka
	(beim Lindner-\TeX dabei) oder Christoph Strunks {\em CTEX\/},
	sollten \ASCREEN\ nun von der Shell aus zug�nglich machen,
	also als Previewer oder Druckertreiber installieren.
	F�r die HKM-\TeX shell ist eine Datei {\tt ASCREEN3.INF}
	beigef�gt, die zur Installation gebraucht wird;
	bei {\em CTEX\/} m�ssen Sie den Previewer
	mit dem passenden Men�eintrag "`finden"'.
	
	Wenn Sie \ASCREEN\ von einer kommandozeilenorientierten
	Shell wie {\em Mupfel\/} oder {\em Gul�m\/} aufrufen m�chten,
	sollten Sie daf�r sorgen,
	da� die Datei {\tt ASCREEN.SET} auch dann gefunden wird,
	wenn das Verzeichnis mit den \TeX-""Programmen
	nicht das aktuelle Verzeichnis ist.
	Am einfachsten setzen Sie dazu mit einem Befehl wie
\begin{typed}
setenv ASCREENSET d:\tex\ascreen.set
\end{typed}
	die Umgebungsvariable {\tt ASCREENSET} auf den vollen Namen
	der Voreinstellungsdatei.
\end{enumerate}

\begin{attention}
Benutzer von Christoph Strunks {\em CTEX\/} sollten eine Kleinigkeit beachten:
man kann Standardargumente f�r Previewer und Druckertreiber angeben,
die dann immer an das betreffende Programm �bergeben werden.
Die normale Voreinstellung f�r den Previewer ist etwas wie {\tt-p=a4}.
\ASCREEN\ versteht diese Option aber recht gr�ndlich mi�
(sie f�hrt dazu, da� der Cachespeicher auf eine Seite begrenzt wird),
so da� Sie sie am besten gleich entfernen.
\end{attention}

\begin{chapterquote}
Und braucht man keine Klempner mehr,
na, da werd' ich halt Installateur!
\author REINHARD MEY, {\sl Ich bin Klempner von Beruf\/} (1974)
\end{chapterquote}
