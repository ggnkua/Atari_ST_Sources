% hyptest.tex --- Beispiel f�r HyperTeX		12. Januar 1992 AL
% Text ist Auszug aus `Praktische Mathematik' von Stummel & Hainer,
%	�bernommen ohne Genehmigung :-(
%
\documentstyle[11pt,hypertex,german]{article}
\def\rg{\mathop{\mathrm{rg}}\nolimits}
\begin{document}
\section{Eliminationsverfahren zur L�sung linearer Gleichungssysteme}

Die L�sung linearer Gleichungssysteme geh�rt zu den Grundaufgaben
der praktischen Mathematik.
Eine der wichtigsten L�sungsmethoden ist das \hyperto{Gau�}{Gauss}sche
Eliminationsverfahren, ein sogenanntes direktes Verfahren,
das nach endlich vielen arithmetischen Operationen die numerische L�sung
der Aufgabe liefert. Die aus der linearen Algebra bekannte
\hyperto{Cramersche Regel}{Cramersche Regel} eignet sich im allgemeinen nicht
zur numerischen L�sung linearer Gleichungssysteme,
da sie f�r die Berechnung der $n$~Unbekannten die Berechnung von
$n+1$ \hyperto{Determinanten}{Determinante} und damit einen zu gro�en
Rechenaufwand erfordert.

Im Abschnitt~$n$ wird das Eliminationsverfahren auf spezielle Klassen
linearer Gleichungssysteme angewandt,
deren Koeffizientenmatrizen \hyperto{positiv definit}{positiv definit},
\hyperto{diagonaldominant}{diagonaldominant}
oder \hyperto{$M$-Matrizen}{M-Matrizen} sind.

\newpage

\leavevmode{\bf\hyperdef{Gau�}{Gauss},}
Carl Friedrich. $^*$ Braunschweig 30.~April 1777,
$\dag$ G�ttingen 23.~Februar 1855, dt.\ Mathematiker, Astronom und Physiker.
--- Der seit 1807 als Prof.\ f�r Astronomie und Direktor der Sternwarte
in G�ttingen wirkende G., bereits zu Lebzeiten als Princeps mathematicorum
bezeichnet, geh�rt zu den bedeutendsten Mathematikern.
1801 ver�ffentlichte er seine {\sl Disquisitiones arithmeticae},
das grundlegende Werk der modernen Zahlentheorie.
Im gleichen Jahr erzielte G. einen besonderen wissenschaftlichen Erfolg,
als W.~Olbers die Wiederauf"|findung des Planetoiden \hyperto{Ceres}{Ceres}
an einer von ihm vorausberechneten Stelle gelang.
G.\ ver�ffentlichte seine hierzu entwickelten Methoden der Bahnbestimmung
1809 in seinem Hauptwerk {\sl Theoria motus corporum coelestium},
in dem er der theoretischen Astronomie eine neue Grundlage gab.
1827 ver�ffentlichte er sein grundlegendes differentialgeometrisches Werk
{\sl Disquisitiones circa superficies curvas}.
Zusammen mit dem Physiker Wilhelm Weber widmete er sich der Erforschung
des Erdmagnetismus, wobei er das nach ihm benannte absolute physikalische
Ma�system aufstellte.
Der von beiden 1833 konstruierte elektromagnetische Telegraph
wurde damals technisch nicht weiterentwickelt.
In diese Zeit fallen auch seine grundlegenden Arbeiten zur Physik,
insbesondere zur Mechanik, zur Potentialtheorie sowie zur
geometrischen Optik.
Auf dem Gebiet der Mathematik sind vor allem noch seine Arbeiten
zur Theorie der unendlichen Reihen,
seine Methoden der numerischen Mathematik
sowie seine Beweise des Fundamentalsatzes der Algebra zu nennen.
({\sl Meyers Taschenlexikon.})

\newpage

\leavevmode{\bf\hyperdef{Determinante}{Determinante}.}
Es gibt genau eine Abbildung $\det: M(n\times n, K) \to K$
mit den folgenden Eigenschaften:
\begin{enumerate}
\item $\det$ ist linear in jeder Zeile
\item Ist $\rg A<n$, so ist $\det A = 0$
\item $\det E=1$.
\end{enumerate}
$\det$ hei�t "`die Determinante"', $\det A$ "`die Determinante von $A$"'.
(J�nich, {\sl Lineare Algebra.})

\leavevmode{\bf\hyperdef{Cramersche Regel}{Cramersche Regel}.}
Ist $\det A\ne 0$ und $Ax=b$, so gilt
\[
x_i = \frac{\det\left(%
	\begin{array}{ccccc}
	a_{11}&\cdots&b_1&\cdots&a_{1n}\\
	\vdots&     &\vdots&    &\vdots\\
	a_{n1}&\cdots&b_n&\cdots&a_{nn}
	\end{array}\right)}{\det\left(%
	\begin{array}{ccccc}
	a_{11}&\cdots&a_{1n}\\
	\vdots&     &\vdots\\
	a_{n1}&\cdots&a_{nn}
	\end{array}\right)},
	i = 1,\ldots, n.
\]
(Die $b$ im Z�hler stehen dabei in der $i$-ten Spalte.)

\leavevmode{\bf\hyperdef{positiv definit}{positiv definit}}
(Diese Definition gebe ich hier nicht, weil allm�hlich genug Material
zum Testen in dieser Datei steht.)

\leavevmode{\bf\hyperdef{diagonaldominant}{diagonaldominant}}
(Diese Definition gebe ich hier nicht, weil allm�hlich genug Material
zum Testen in dieser Datei steht.)

\leavevmode{\bf\hyperdef{$M$-Matrix}{M-Matrix}}
(Diese Definition gebe ich hier nicht, weil allm�hlich genug Material
zum Testen in dieser Datei steht.)

\leavevmode{\bf\hyperdef{Ceres}{Ceres}.}
Ein Kleinplanet, der von Olbers nach Berechnungen von \hyperto{Gau�}{Gauss}
wiederentdeckt wurde. (Diese etwas d�mmliche Erl�uterung dient dazu,
einen neuen Verweis auf \hyperto{Gau�}{Gauss} unterzubringen.)

\end{document}
