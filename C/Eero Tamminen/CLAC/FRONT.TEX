%format latex
%macropackage=lplain

\documentstyle[12pt,refman]{article}

\newcommand{\myop}[1]{{\bf #1}}

\title{Front --- The Manual}
\author{Eero Tamminen \\
	Version~0.90 \\
	24.~May 1994}

\pagestyle{myfootings}
\markboth{Front --- The Manual}{Front (C) 1994 by Eero Tamminen}

\begin{document}

\maketitle \makeauthor
\newpage

\section{Name}

	Front --- a command line interface for Clac().


\section{Synopsis}

	{\sl front\/} [-- {\sl options\/}] [{\sl @filename\/} $|$ {\sl expression\/}]


\section{Description}

	{\sl Clac\/} is a C-function for expression evaluation.
	{\sl Front\/} acts as an user interface for it.

	{\sl Front\/} has three modes of operation:
	\begin{itemize}
	\item	An expression is parsed from arguments.
	\item	Without a filename or expression {\sl Front\/} enters
		interactive mode. Expressions are evaluated as
		they are entered.
	\item	Expressions are read from a file.
	\end{itemize}

	If file named `front.ini' exists in the current directory, it's
	executed before anything else is done.

	You can exit from interactive mode by giving {\sl Front\/} an empty
	line.


\section{Options}

	Command line options recognized by {\sl Front}.
	\begin{description}
	\item[\myop{-d}]	Output result in decimals (default).
	\item[\myop{-b}]	Output result in binary decimals.
	\item[\myop{-o}]	Output result in octal decimals.
	\item[\myop{-h}]	Output result in hex decimals.
	\item[\myop{-r}]	Trigonometric arguments as radians (default).
	\item[\myop{-g}]	Trigonometric arguments as degrees.
	\end{description}


\section{Characters and Operators}

	Possible characters and operators in {\sl Front\/}:
	\begin{itemize}
	\item	All the operators and functions that {\sl Clac\/} supports.
	\item	Assignment operator `$=$'. Eg. `test $= 3.5$'.
	\item	Mode change operator `:'. Look under 'OPTIONS' for
		available modes. Eg. use `:b' to switch on binary output.
	\item	Variable `E', containing the value of Napier's constant.
	\item	Variable `PI', containing value of $ \pi $.
	\item	`?' gives a brief overview of the operators and
		functions that {\sl Front \& Clac\/} currently support.
	\end{itemize}

	You can change the values of all variables including E and Pi
	(if you're one of those people that insist on $ \pi $ being 3).


\section{Limitations}

	255 character limit for expressions (do you need more?).


\section{Bugs}

	Sure. Haven't yet found any, though. Oh, wait a minute, here's
	one ;)...


\section{Examples}

	{\sl From a command line:}
	{\tt \begin{verbatim}
	      clac '$FF.8 - ((%10101.01 * 12) + 5) / 3.7 + (-1)'
	\end{verbatim}}

	{\sl Interactively:}
	{\tt \begin{verbatim}
	      clac -d

	      Expression: foo = sin(90) + 1
	       = 2

	      Expression: foo + 2
	       = 4
	\end{verbatim}}


\section{Thanks to}

	\begin{itemize}
	\item	Jarkko Kniivil\"a, who advised me on C syntax.
	\item	J\"urgen Lock, who patiently answered to my mail.
	\item	Helpful people on \#atari (irc), whom I have
		chatted with: Dirch, Gryf, Infy, Aviva\dots
	\end{itemize}


\section{Copyright}

	Eero Tamminen / puujalka (irc), t150315@cc.tut.fi.

	Freeware.

\end{document}
