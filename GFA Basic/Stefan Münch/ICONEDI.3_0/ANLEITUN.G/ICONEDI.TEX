%format latexg
% IconEdi-manual (c) 1991 by Stefan M"unch
%

%\def\bilder{\\}

\ifx\bilder\undefined
 \documentstyle[twoside,german,din_a4]{report}
 \typeout{Output not illustrated.}
\else
 \documentstyle[twoside,draw,german,din_a4]{report}
 \typeout{Output illustrated.}
\fi

\sloppy
\parskip2ex
\parindent0ex
\setcounter{secnumdepth}{8}
\setcounter{tocdepth}{8}
\pagestyle{headings}

\newdimen\mu \mu=0.0007mm
\newdimen\xu \xu=372\mu
\newdimen\yu \yu=372\mu

% portable definition of the Control-symbol:
\newcommand{\control}{
\setlength{\unitlength}{372\mu}
\begin{picture}(10,11)
\put(4,8){\rule{1\xu}{2\yu}}
\put(3,7){\rule{3\xu}{1\yu}}
\put(3,6){\rule{1\xu}{1\yu}}
\put(5,6){\rule{1\xu}{1\yu}}
\put(2,5){\rule{2\xu}{1\yu}}
\put(5,5){\rule{2\xu}{1\yu}}
\put(2,4){\rule{1\xu}{1\yu}}
\put(6,4){\rule{1\xu}{1\yu}}
\put(1,3){\rule{2\xu}{1\yu}}
\put(6,3){\rule{2\xu}{1\yu}}
\put(1,2){\rule{1\xu}{1\yu}}
\put(7,2){\rule{1\xu}{1\yu}}
\end{picture}}

% portable definition of the Alternate-symbol:
\newcommand{\alternate}{
%\newdimen\mu \mu=0.0007mm
%\newdimen\xu \xu=372\mu
%\newdimen\yu \yu=372\mu
\setlength{\unitlength}{372\mu}
\begin{picture}(10,11)
\put(0,10){\rule{3\xu}{1\yu}}
\put(6,10){\rule{3\xu}{1\yu}}
\put(0,8){\rule{2\xu}{2\yu}}
\put(7,8){\rule{2\xu}{2\yu}}
\put(4,7){\rule{1\xu}{2\yu}}
\put(0,6){\rule{1\xu}{2\yu}}
\put(8,6){\rule{1\xu}{2\yu}}
\put(3,4){\rule{3\xu}{3\yu}}
\put(0,3){\rule{1\xu}{2\yu}}
\put(8,3){\rule{1\xu}{2\yu}}
\put(4,2){\rule{1\xu}{2\yu}}
\put(0,1){\rule{2\xu}{2\yu}}
\put(7,1){\rule{2\xu}{2\yu}}
\put(0,0){\rule{3\xu}{1\yu}}
\put(6,0){\rule{3\xu}{1\yu}}


\end{picture}}

\makeglossary
\makeindex


\begin{document}

% titlepage:
%\input titel.tex
\thispagestyle{empty}
\begin{center}

\vspace*{8cm}

{\Huge IconEdi} 

{\LARGE Der Iconeditor f"ur ST/TT} 

\begin{large}
{\copyright} by \\ 
Stefan M"unch \\
Borbergstr. 38 \\
D-47 Hamm 1 

26. September 1991
\end{large}

\end{center}
\newpage

% abstract
%\input abstract.tex
\setcounter{page}{0}
\thispagestyle{plain}

{\LARGE IconEdi} \\

Ein Programm von: Stefan M"unch, Borbergstr. 38, D-47 Hamm 1

IconEdi wurde mit GFA-Basic 3.5E bzw. aktuelleren Versionen 
erstellt.

Diese Anleitung wurde von mir mit {\LaTeX} in der Implementation
von Christoph Strunk auf einem Atari-ST Rechner erstellt.

Viele der verwendeten Namen von Soft- und Hardware sind 
eingetragene Warenzeichen und als solche zu betrachten.

Diese Anleitung darf, dem Sharewareprinzip folgend, genau wie
das Programm in unver"anderter Form zu nichtkommerziellen 
Zwecken kopiert und weitergegeben werden. Alle Rechte bleiben 
aber bei mir.

\bigskip

{\Large Haftungsausschlu"s} \index{Haftungsausschlu"s}
\index{Garantie}

Wenn Sie sich eine Software-Anleitung anschauen, finden Sie
fast immer Abschnitte wie den folgenden:

F"ur die Funktionsf"ahigkeit des Programms, f"ur Fehlerfreiheit von
Programm und Anleitung wird keine Garantie "ubernommen. F"ur Sch"aden 
jeglicher Art, die durch oder im Zusammenhang mit der Nutzung des
Programms oder der Anleitung entstehen, wird keine Haftung 
"ubernommen.

Ich mu"s dies auch f"ur IconEdi sagen. Es ist bei der Komplexit"at
heutiger Rechnersysteme und Programme nicht m"oglich, alle 
Fehlerm"oglichkeiten zu ber"ucksichtigen und entsprechende 
Vorkehrungen zu treffen, um Sch"aden zu vermeiden. Da"s ich mit
obigem Abschnitt in guter (?) Gesellschaft mit vielen gro"sen 
Softwarefirmen bin, tr"ostet mich wenig und nutzt Ihnen "uberhaupt
nichts. Daher folgendes:

Ich habe bei der Erstellung von Programm und Anleitung alle mir
m"ogliche Sorgfalt walten lassen, um Fehler zu vermeiden. Testen
Sie das Programm auf seine Funktionst"uchtigkeit auf Ihrem 
Rechnersystem. Wenn Fehler auftreten, k"onnen Sie mich davon
in Kenntnis setzen und ich werde mich bem"uhen, die Fehler zu 
beseitigen. Zur Verfahrensweise dabei sehen Sie bitte in Anhang
F dieser Anleitung nach. Hardwaresch"aden k"onnen, soweit ich dies
"uberschauen kann, nicht durch IconEdi verursacht werden. 
An sonstigen Sch"aden kommt vor allem Datenverlust in Frage,
wenn z.B. IconEdi 'abst"urzt' und Sie ein in Arbeit befindliches
Icon nicht mehr speichern k"onnen. Noch gef"ahrlicher wird es, wenn
Sie mehrere Programme gleichzeitig benutzen (wie dies mit 
$\to$MultiGEM z.B. m"oglich ist), IconEdi abst"urzt, und Sie z.B. 
einen Text in einem anderen Programm noch nicht gesichert haben.
Ich hoffe, da"s die geschilderten Situationen fiktiv bleiben ...

Hamm, im September 1991, Stefan M"unch

\newpage

\setcounter{page}{1}
\pagenumbering{Roman}
\tableofcontents
\newpage
\pagenumbering{arabic}
\setcounter{page}{1}

%\input chapter1.tex
\chapter{Einf"uhrung} \index{Einf"uhrung}
\section{Zur Benutzung dieses Handbuchs} 
\index{Zur Benutzung dieses Handbuchs}
Sollten Sie w"ahrend der Lekt"ure dieses Handbuchs auf Ausdr"ucke 
stossen, die Sie nicht verstehen, so schauen sie einmal in 
Anhang B 'Begriffserkl"arungen' dieses Handbuchs oder des 
Atari-Handbuchs nach. Sollten Sie auch hier nicht f"undig werden, 
und auch niemanden 'an der Hand' haben, der Ihnen weiterhelfen 
kann, so seien Sie ermutigt, mich schriftlich um Rat zu fragen. 
Beachten Sie bitte die in Anhang F 'Fehlermeldungen und Fragen 
zu IconEdi' beschriebenen Bedingungen f"ur solche Anfragen.
Vor manchen Begriffen steht ein $\to$-Zeichen. Dieses deutet an,
da"s der dahinterstehende Begriff in Anhang B erkl"art ist.
Viele Verst"andnisschwierigkeiten erledigen sich auch von
selbst, wenn Sie weiterlesen --- es ist leider oft nicht 
vermeidbar, vorzugreifen. Wie so oft, gilt auch bei Software:
man lernt den Umgang am besten beim arbeiten damit. Lesen Sie
sich also die Anleitung einmal durch, machen Sie sich eventuell
Notizen zu Punkten, die Sie nicht verstanden haben, und probieren
Sie es dann einfach aus.

\index{"Anderungen am Handbuch} \index{Erg"anzungen zum Handbuch}
Ein Handbuch hat, einmal gedruckt, den Nachteil, da"s es 
nicht so leicht aktualisiert werden kann wie ein Programm,
von dem man einfach die neue Version auf seine Diskette(n) oder
Festplatte kopiert. 

Damit ein Neuausdruck nicht bei jeder neuen Version des 
Programms notwendig ist, finden Sie jeweils bei einer neuen 
Version im Ordner ANLEITUN.G neben diesem Text (ICONEDI.TEX)
Dateien NEW1.TXT, NEW2.TXT usw. NEW1.TXT enth"alt die 
"Anderungen gegen"uber dem ICONEDI.TEX, das beim ersten Mal
dabei war, NEW2.TXT gegen"uber diesem und NEW1.TXT und so 
weiter. So haben Sie die M"oglichkeit, selbst zu entscheiden,
ob Sie das neue ICONEDI.TEX (diese Datei wird immer auf den 
neuesten Stand gebracht) ausdrucken wollen oder nur die 
Erg"anzungen, die f"ur Sie neu sind. Bei den in den 
$\to$Mailboxen erh"altlichen Versionen werden maximal 10
"Anderungsdateien mitgeliefert: das bedeutet f"ur Sie, da"s 
Sie sich hier mindestens jedes 10. Update besorgen m"ussen,
um alle "Anderungsdateien zu haben. Sollten Ihnen trotzdem
einmal "Anderungsdateien fehlen, k"onnen Sie sie bei mir
nachbestellen. Geben Sie dabei an, welche Sie brauchen,
und beachten Sie die in Anhang L aufgef"uhrten Bedingungen.
Meine Adresse finden Sie auch dort. Sofern Platz auf
der Diskette bleibt, erhalten Sie dabei nat"urlich auch
die neueste Version von IconEdi.

Bei den auf Public-Domain-Disketten erh"altlichen 
Versionen werden keine "Anderungsdateien dabei sein. Dies w"are
angesichts der Zeit, die vergeht, bis eine neue Version
auf diesen erh"altlich ist, unsinnig. Sollten Sie
IconEdi von einer Public-Domain-Diskette haben, wenden
Sie sich bitte an mich, um neue Versionen zu erhalten.
Beachten Sie bitte die in Anhang L genannten Bedingungen 
daf"ur.

\section{Icons} \index{Icons} \index{Was sind Icons ?}

Als BenutzerIn eines Atari-ST/TT-Computers haben Sie 
wahrscheinlich st"andig mit Icons zu tun. Icons sind diese 
kleinen Bildchen, die als Symbole f"ur Laufwerke, Dateien und 
anderes mehr auf dem Bildschirm herumschwirren (naja, schwirren 
tun sie meistens nicht). Icon ist Englisch und bedeutet 'Bild' 
oder 'Ikone'. Letztere Bedeutung vergessen wir ganz schnell 
wieder; mit Sinnbild (oder Symbol) kommen wir der Sache 
wesentlich n"aher. Wenn Sie irgendwo mal in Computerliteratur 
'Piktogramm' lesen, so ist ebenfalls h"ochstwahrscheinlich ein 
Icon gemeint. 

\index{Daten} \index{Maske} \index{Icondaten} \index{Iconmaske}
Beim Betriebssystem des ST/TT bestehen Icons aus Datenteil
und einer Maske (das unterscheidet sie z.B. von $\to$Images,
\glossary{Images} die keine Maske haben). Dies hat einen einfachen 
Grund: ein Icon kann z.B. auf dem $\to$Desktop \glossary{Desktop}
liegen (wie die Laufwerksicons und der Papierkorb), und hier stellt 
die Maske quasi einen eigenen Hintergrund f"ur die Daten des Icons 
dar. Die Farbe der Maske und die Farbe der Daten kann frei bestimmt
werden (Icons k"onnen nicht nur schwarz/wei"s sein). Wenn in einem
Icon an einer Stelle weder der Punkt in der Maske noch der 
in den Daten gesetzt ist, scheint der Hintergrund, auf 
dem das Icon liegt, durch. Bei den Icons von $\to$Gemini 
\glossary{Gemini} hat die Maske "ubrigens eine andere Bedeutung; 
dieses wird in Kapitel 3 'Beispiele' erkl"art. 

Icons haben meist einen Text (der unter dem Icon erscheint 
--- es geht auch ohne, mit einem Trick), und manche Icons 
enthalten auch noch einen Buchstaben, wie z.B. die 
Laufwerksicons des Desktop. Wenn man Icons selektiert (anklickt),
werden sie invertiert. Alle diese Dinge k"onnen Sie mit
IconEdi einstellen.

Seit dem Erscheinen von $\to$alternativen Benutzeroberfl"achen 
\glossary{Alternative Benutzeroberfl"achen} (wie $\to$Gemini,
\glossary{Gemini} Neodesk) und der TOS-Betriebssystemversionen 
ab 2.x erleben Icons auf dem ST/TT eine neue Bl"ute, kann man
doch bei diesen endlich Icons einzelnen Dateien zuweisen und 
diese Icons auf das Desktop ablegen. Klickt man sie dort 
an, wird das zugeh"orige Programm gestartet, und man kann sich 
das Suchen in irgendwelchen Dateiverzeichnissen, das 
Fenster"offnen und -schliessen sparen. Wenn Sie nicht wissen, 
wovon ich spreche, sei Ihnen {\sl w"armstens} empfohlen, sich 
einmal Gemini anzuschauen.

Icons haben aber einen Fehler: sie m"ussen gezeichnet werden. 
Dazu benutzt man zweckm"a"sigerweise ein Programm, da"s einem 
dabei hilft .... einen Iconeditor. 

IconEdi ist nicht der erste Iconeditor f"ur ST/TT, aber ich 
meine, er ist der sch"onste. Er ist nicht perfekt, aber er 
wird (mit Ihrer Hilfe ?) perfekter werden. Und er leistet 
jetzt schon mehr als alle mir bekannten vergleichbaren 
Programme ("ubrigens: wenn ich in diesem Handbuch auch in
der Regel von Icons spreche, IconEdi kann auch ebenso 
$\to$Images \glossary{Images} bearbeiten).

Aber machen Sie sich Ihr eigenes Bild. Denn IconEdi ist nun 
Shareware. Was das bedeutet, sei nun erl"autert:

\section{Shareware} \index{Shareware}

Shareware ist nat"urlich auch Englisch und hat was mit teilen 
zu tun (to share) und irgendwas mit Software. Nun sind Sie 
auch nicht schlauer, gell ?

Shareware ist Software, die kopiert und ausprobiert werden darf, 
f"ur die man aber trotzdem zahlen mu"s, wenn man sich entschlossen 
hat, sie zu benutzen. Sie d"urfen IconEdi weiterkopieren --- aber 
bitte nur komplett, mit allen Dateien, die dabei waren !! Eine 
Liste der Dateien, die dabeigewesen sein sollten, finden Sie in 
Anhang L 'Lieferumfang'. Fehlen Ihnen irgendwelche, dann hauen 
Sie demjenigen, von dem Sie diese Version haben, von mir eins 
auf die Nase und besorgen Sie sich das komplette Paket bei mir. 
Wie das geht und wo Sie es sonst noch bekommen k"onnen, steht 
auch in Anhang L. Wenn Ihnen IconEdi gef"allt und Sie es benutzen 
m"ochten (sich also wirklich Icons erstellen wollen und Sie 
nicht mehr nur mit dem Programm herumspielen), m"ussen Sie daf"ur 
zahlen. Dieses Mu"s ist keines, das ich in jedem Fall 
einklagen k"onnte --- es 
ist ein moralisches Mu"s und ein Appell an Ihre Fairness und 
Ihren gesunden Menschenverstand. Wenn Sie m"ochten, da"s IconEdi 
weiterentwickelt wird, wenn Sie es gut finden, leistungsf"ahige 
Software zu einem geringen Preis zu bekommen, und wenn Sie sich 
diese M"oglichkeit auch in Zukunft erhalten wollen, dann 
unterst"utzen Sie das Sharewareprinzip !! Dies gilt nicht nur
f"ur IconEdi, es gilt genauso f"ur andere Sharewareprogramme. 
Um es auf eine kurze Formel zu bringen: Shareware darf man 
kostenlos {\sl ausprobieren}, wenn man Shareware {\sl benutzt}, 
mu"s man sie {\sl bezahlen}. Tut man dies nicht, arbeitet man
mit einer {\sl Raubkopie} und enth"alt dem Programmierer der 
Software seinen {\sl angemessenen Lohn} vor. 

Viel 'um den Brei'-Gelaber ? Na gut, werden wir konkret.

\index{Was kostet IconEdi ?} \index{IconEdi} \index{Haben wollen}
IconEdi kostet 30,- DM. Sie haben richtig gelesen, drei"sig. Wenn 
Sie der Meinung sind, da"s IconEdi dies nicht wert ist, dann 
besteht auch kein Grund f"ur Sie, es zu {\sl benutzen}, und wir 
sind beide zufrieden. Wenn Ihnen das eine oder andere noch nicht 
gef"allt, k"onnen Sie mir dies schreiben, und ich werde sehen, ob
ich es "andern kann. Beachten Sie bitte hierbei die in Anhang F
genannten Bedingungen f"ur Fragen zu IconEdi. Ob Ihnen alles 
gef"allt oder nicht, ist aber nicht wichtig daf"ur, ob Sie zahlen 
m"ussen: wenn Sie IconEdi benutzen, m"ussen Sie zahlen, auch, wenn 
Ihnen das eine oder andere nicht gef"allt.

\index{Programmautor} \index{"Uberweisung} \index{Sharewaregeb"uhr}
Um zu zahlen schicken Sie entweder 30,- DM an Stefan M"unch, 
Borbergstr. 38, D-4700 Hamm 1, oder "uberweisen Sie 30,- DM 
auf mein Konto bei der Deutschen Bank Hamm, Bankleitzahl
410 700 49, Kontonummer 45 35 63 mit dem Vermerk 'IconEdi' und 
Ihrem Namen und Ihrer Adresse. Nachdem Sie gezahlt haben,
\index{Updates} \index{neue Versionen}
werden Sie automatisch "uber die n"achste neue Version (Update)
von IconEdi mit einer Postkarte informiert. Sie k"onnen dann 
das Update von mir kostenlos gegen Einsendung einer
formatierten, doppelseitigen 3.5-Zoll-Diskette und eines
frankierten R"uckumschlags erhalten.
Legen Sie eine neue Postkarte bei, werden Sie wieder informiert 
und so weiter. 
\index{Hotline} \index{Telefonischer Support}
Wer zahlt, hat ferner das Recht, mich Samstags von
15-17h unter Tel. 049-02381-23324 mit Fragen zu IconEdi zu 
l"ochern, und kann von mir gegen Einsendung oder "Uberweisung von
10,-DM (mit dem Vermerk 'IconEdi-Handbuch' und Namen/Adresse)
eine gedruckte Version des jeweils aktuellen Handbuchs erhalten
(die 10,-DM verstehen sich als Unkostenbeitrag und umfassen alle
Druck-, Kopier- und Versandkosten).

\section{Danksagungen} \index{Danksagungen}
Ich danke: \\
Jochen Linz f"ur Betatesten; 
Gunter Siepmann f"ur Hilfe in h"ochster Not (meine Sourcen !!); 
Dieter Weinert f"ur TT-Betatesten;
Frank Ostrowski f"ur sein GFA-Basic, das besser ist, als sein Ruf; 
Lars van Straelen von GFA f"ur den Tip, wie man GFA-Basic von 
LineA befreit; 
dem Programmierer des GFA-GEM-Utility-Package,
das zwar teilweise grauslich programmiert ist, 
aber trotzdem sehr brauchbar; 
Julian F. Reschke, Dietmar Rabich und Hans-Dieter Jankowski f"ur 
das tolle und viel zu dicke Profibuch;
Stefan H"ohn f"ur seine {\sl hervorragenden} Artikel zu den Themen 
Icons (1+2/90) und Ressourcen (7/90) in der $\to$ST-Computer
\glossary{ST-Computer}; 
Tassilo Nitz und Daniel R"osen, deren FormDo- und 
ObjectDraw-Routinen ich zu dem vermanscht und erweitert habe, 
was die Dialoge in IconEdi nun so sch"on bedienbar macht; 
Karsten Isakovic f"ur sein tolles AutoSwitchOverscan
und den Sysmon;
Stefan Eissing und Gereon Steffens f"ur $\to$Gemini 
\glossary{Gemini}; 
Hans-J"urgen Richstein, dessen Kobold
es zu verdanken ist, da"s ich mich um Backups nicht mehr dr"ucke;  
Olaf Meisiek f"ur Interface
--- b"a"ah, ist das DR-RCS 2.0 schr"ocklich; 
Udo Erdelhoff ('Wann kommt IconEdi-Master ?'); 
den vielen guten Geistern, die mich bei Fragen im Mausnetz oder 
im Fidonetz nicht h"angengelassen haben (und die teilweise auch 
schon genannt sind); 
Atari f"ur den ST und DR f"ur das GEM (a real juwel); 
Christoph Strunk f"ur seine {\sl tolle}  $\to$\TeX-Implementation
\glossary{\TeX} (und nat"urlich Donald E. Knuth und Leslie Lamport
f"ur \TeX und \LaTeX). 

Allen, die ich vergessen habe (oder unterschlagen, um nicht 100
Seiten mit Danksagungen zu f"ullen), ein Blanko-Danke !!


%\input chapter2.tex
\chapter{Die Bedienung} \index{Bedienung}

\section{Installation} \index{Installation}
F"ur den Anfang reicht es, den Ordner ICONEDI dorthin zu kopieren, 
wo Sie ihn haben m"ochten und das Programm ICONEDI.APP (darin) zu 
starten. Wie Sie IconEdi an Ihre pers"onlichen Bed"urfnisse anpassen 
k"onnen, erfahren Sie in Kapitel 4.

Man kann IconEdi einen Dateinamen "ubergeben. IconEdi sieht diese Datei
als Icon an und l"adt sie. So kann man nun auch IconEdi f"ur einen 
bestimmten Dateityp anmelden, und bei Doppelklick auf eine
Datei dieses Typs wird IconEdi gestartet und l"adt die Datei. Unter 
$\to$alternativen Desktops wie $\to$Gemini  
kann man eine Icondatei auf IconEdi schieben, IconEdi startet dann
und l"adt die Datei.

\section{Tastaturfunktionen} \index{Tastaturfunktionen}
Wie jedes bessere Programm kann auch IconEdi fast vollst"andig "uber
die Tastatur bedient werden. Die Tastenzuordnung kann fast 
vollst"andig ver"andert und so an verschiedene Geschm"acker angepa"st 
werden (siehe dazu Kapitel 4). 

Grunds"atzlich gilt: 

\begin{itemize}

 \item Bei den Funktionen in der Men"uleiste steht hinter
 dem Namen der Funktion ein K"urzel f"ur die zugeh"orige Taste. 
 Dabei gilt: 
  \begin{itemize}
   \item wenn ein \alternate -Zeichen vor einem Buchstaben 
   steht, mu"s die Alternate-Taste zusammen mit diesem Buchstaben
   gedr"uckt werden, um die Funktion aufzurufen
   \item wenn ein \control -Zeichen vor einem Buchstaben
   steht, mu"s die Control-Taste zusammen mit diesem Buchstaben
   gedr"uckt werden, um die Funktion aufzurufen
   \item hinter einigen Funktionsnamen steht im Klartext, welche 
   Taste gedr"uckt werden mu"s (Beispiel: 'Space' steht f"ur die 
   Leertaste)
   \item in allen anderen F"allen mu"s die jeweilige Taste alleine 
   gedr"uckt werden
  \end{itemize}

 \item In Dialogen mu"s, um eine Funktion "uber die Tastatur 
 auszuf"uhren, die Alternate-Taste zusammen mit einem Buchstaben 
 (A..Z) gedr"uckt werden. Der jeweilige Buchstabe ist unterstrichen. 
 Eine Funktion ist jeweils zus"atzlich durch Dr"ucken der Return-Taste 
 ausf"uhrbar: dies ist durch einen dicken schwarzen Rahmen um den 
 Funktionsnamen gekennzeichnet.

 Besonderheiten und/oder Abweichungen hiervon werden bei der 
 Erkl"arung des jeweiligen Dialogs beschrieben.

 \item Was viele nicht wissen: das TOS erlaubt es, durch 
 Kombinationen der Alternate-, Shift-, Insert-, Clr/Home- und 
 Cursortasten den Mauszeiger zu bewegen und Mausklicks 
 auszuf"uhren. Dies ist im Handbuch zum ST/TT beschrieben (bei 
 mir Kapitel 2.3.3. 'Steuerung der Maus "uber die Cursortasten')
 und funktioniert selbstverst"andlich auch in IconEdi.

\end{itemize}

\section{Der Arbeitsbildschirm von IconEdi}
\index{Arbeitsbildschirm von IconEdi}

\ifx\bilder\undefined
 Ein Beispiel f"ur den Arbeitsbildschirm von IconEdi finden 
 Sie als 'START.IMG' im Ordner 'BILDER' im Anleitungs-Ordner
 des IconEdi-Pakets.
 Im folgenden wird des "ofteren auf dieses Bild verwiesen.
\else
 \begin{draw}{385.00}{250.00}{Beispiel f"ur den Arbeitsbildschirm von IconEdi}
  \put(0.00,0.00){\special{CS!g 0.62 bilder/start.img}}
 \end{draw}
\fi

\subsection{Das Fenster} \index{Fenster von IconEdi} 

Der Bildschirm befindet sich in einem vollst"andigen GEM-Fenster, 
wie Sie es auch von anderen GEM-Programmen kennen. Das 
GEM-Fenster meine ich. Da"s der Arbeitsbildschirm in einem 
GEM-Fenster liegt ist so verbreitet noch nicht. Viele Programme 
benutzen daf"ur ein eigenes $\to$Desktop. \glossary{Desktop}
Dies ist jedoch bei der Verwendung von $\to$MultiGEM
\glossary{MultiGEM} oder einem zuk"unftigen Multitasking-TOS 
h"ochst l"astig, da man dann st"andig zwischen den Desktops hin- 
und herschalten mu"s. IconEdi zeigt, da"s es auch anders geht.

\index{GEM-Fenster}
Die Bedienung von GEM-Fenstern ist "ubrigens im Handbuch zum ST/TT 
beschrieben. Bei der Bet"atigung des Schlie"sfelds wird IconEdi
beendet.

\subsubsection{Die Titelleiste} \index{Titelleiste}
In der Titelleiste des Fensters wird jeweils der aktuelle 
Pfad dargestellt. Im obigen Bild steht dort 
'E:$\backslash$ICONEDI$\backslash$TEST$\backslash$'.
% E:\ICONEDI\TEST\
Wenn Sie ein Icon geladen haben, stehen dort auch noch dessen 
Dateiname und Dateiextension. Beispiel: Sie haben das Icon mit 
dem Dateinamen 'OSKAR' und der Dateiextension 'ICO' aus dem 
Ordner 'ICONS' (im Ordner 'ICONEDI') von Laufwerk A: geladen. 
Dann steht dort 
'A:$\backslash$ICONEDI$\backslash$ICONS$\backslash$OSKAR.ICO'. 
% A:\ICONEDI\ICONS\OSKAR.ICO
Wenn die Datei des Icons keine $\to$Extension 
\glossary{Extension} hatte, wird auch keine dargestellt. 
Auch der Punkt f"allt dann weg.

Beim Laden, Speichern, vom Klemmbrett ziehen und beim Umschalten 
zwischen Icon- und Imagemodus kann sich der aktuelle Pfad "andern. 
Die Darstellung in der Titelzeile wird dann entsprechend angepa"st. 

\subsubsection{Die Informationsleiste} \index{Informationsleiste}
In der Informationsleiste wird , wenn sich der Mauszeiger im 
Arbeitsraster befindet, die X- und Y-Position des Zeigers 
angegeben. Im obigen Bild steht dort 
'X~:~~18~/~~26~~~~Y~:~~3~/~201'. Das bedeutet, da"s der Mauszeiger 
zuletzt an der relativen Position (18,3) und der absoluten 
Position (26,201) im Arbeitsraster stand. 
Die relative Position ist die von der oberen Ecke des 
Arbeitsrasters gerechnete, die absolute ist die zur oberen Ecke 
des Icons gerechnete (es wird ja eventuell nur ein Ausschnitt 
des Icons dargestellt). Diese Werte werden, wenn sich der 
Mauszeiger im Arbeitsraster befindet, laufend aktualisiert.

\subsection{Das Arbeitsraster} \index{Arbeitsraster}
Das Arbeitsraster ist jener Teil, in dem sich im obigen Bild die
gro"se Rose befindet. In diesem Raster k"onnen Sie an Ihrem Werk
herummalen, und zwar entweder an den Daten oder der Maske. Manche
Iconeditoren stellen Daten und Maske im gleichen Raster dar; ich
finde das ausgesprochen unpraktisch. Wechseln zwischen Daten-
und Maskenbearbeitung k"onnen Sie mit dem Men"upunkt 'Wechsel' oder
der zugeh"origen Taste oder durch Anklicken der Ergebnisk"asten.

\subsubsection{Die Spezialelemente} \index{Spezialelemente}
Die Spezialelemente  haben nur eine Funktion, wenn in IconEdi ein 
Icon bearbeitet wird, das gr"o"ser ist, als im Raster darstellbar.

IconEdi kann (mit Einschr"ankungen) Icons bis zu einer Gr"o"se von 
640 * 400 Punkten bearbeiten (je nach Bildschirmaufl"osung eventuell 
weniger, im obigen Beispiel --- die Aufl"osung dort war die hohe 
ST-Aufl"osung, 640 mal 400 Punkte --- sind es maximal 624 mal 368 
Punkte), in dem Raster ist aber nur ein Teil solcher 'Monster' 
darstellbar, da die dort dargestellten 'Punkte' nicht beliebig klein 
sein k"onnen, um noch gescheit positionieren zu k"onnen. Daher 
befinden sich unter dem und rechts des Rasters Rollboxen und 
Rollpfeile, mit denen der Ausschnitt, der bearbeitet werden soll, 
gew"ahlt werden kann. Die Rollboxen verhalten sich (noch) nicht 
GEM-konform; dies wird in n"aherer Zukunft ge"andert. 

Eine Besonderheit stellt das Feld rechts unten neben dem 
Rollpfeil-Rechts und unter dem Rollpfeil-Unten dar (das 
Feld mit den 'konzentrischen' Quadraten). Klickt man dieses an, 
so erscheint eine Box, in der man den zu bearbeitenden Ausschnitt 
durch w"ahlen durch Anklicken mit der linken Maustaste w"ahlen 
kann. Abbrechen kann man die Auswahl durch Dr"ucken der rechten 
Maustaste oder der Esc-Taste.

Beachten Sie bitte auch, da"s ein Icon der Gr"o"se 640 mal 400 
"uber 64000 Bytes lang ist und eine RSC-Datei maximal 65536 Bytes
lang sein darf. Ein Image dieser Gr"o"se ist, da es keine Maske
hat, etwa halb so lang. 

\subsubsection{Freihandzeichnen} \index{Freihandzeichnen}
Klickt man in dem Raster mit der linken Maustaste einen Punkt an, 
wird dieser invertiert. H"alt man die Maustaste gedr"uckt und 
bewegt die Maus hin und her, malt man 'Freihand'. Und zwar gilt 
hier: war der erste angeklickte Punkt schwarz, wird wei"s 
gezeichnet, war er wei"s, wird schwarz gezeichnet. 

Wenn man die Maus dabei schnell bewegt, stellt man fest, da"s der 
Mauszeiger zur"uckspringt. Dies hat folgenden Zweck: um 
zusammenh"angende Kurven (eine Linie ist eine spezielle Kurve) 
zeichnen zu k"onnen, ist es ausgesprochen unpraktisch, wenn bei 
etwas schnellerer Mausbewegung L"ocher bleiben. Um solche L"ocher 
zu vermeiden, setzt IconEdi den Mauszeiger zur"uck. Ich denke, da"s 
man sich nach kurzer Zeit daran gew"ohnt hat und es nicht mehr 
missen m"ochte.

Der eine oder andere wird jetzt einwenden, ich solle doch dann 
bitte die Routine entsprechend beschleunigen, damit dieses 
Zur"uckspringen nicht n"otig ist. Das verschiebt aber das Problem 
nur (und die Routine ist schon optimiert). Das Problem ist 
prinzipiell nicht l"osbar, da von der Maus nicht kontinuierliche 
Werte geliefert werden. Wenn Sie den Mauszeiger mal schnell "uber 
den Bildschirm bewegen und genau hinsehen, werden Sie feststellen, 
da"s auch hier nicht alle Zwischenstufen des Mauszeigers gezeichnet 
werden: der Zeiger 'springt'. Besonders deutlich wird dies, wenn 
man z.B. einen Farbemulator laufen hat (z.B. Panda von 
$\to$Maxon-Sonderdisk \glossary{Maxon-Sonderdisketten} 18), der 
ordentlich Rechenzeit kostet.

Zeichenprogramme wie z.B. Special Paint (Maxon-Sonderdisk 21) 
oder STAD (von Application Systems Heidelberg) benutzen folgenden 
Trick, um L"ocher beim Freihandzeichnen zu vermeiden: sie verbinden 
die L"ocher mit einer Linie. Das erkennt man sehr gut, wenn man 
einen Trackball hat und dessen Kugel 
beim Zeichnen einen ordentlichen Schubs gibt...

F"ur IconEdi, das eine Art 'Dauer-Lupe' zum Zeichnen benutzt, ist 
diese L"osung nicht geeignet, da der Rechner dann mit dem Verbinden 
der L"ocher sehr schnell nicht mehr nachk"ame. "Ubrigens: ich kenne 
kein Malprogramm, bei dem in der {\sl Lupe} die L"ocher {\sl nicht} 
auftreten.

\subsubsection{Der F"ullmodus} \index{F"ullmodus}
Klickt man in dem Raster die {\sl rechte} Maustaste, gelangt man 
in den F"ullmodus ($\to$Modus) \glossary{Modus}. Der Mauszeiger 
"andert sich in ein (Ziel-)Kreuz und das Wort 'F"ullmodus' 
erscheint. Dr"uckt man nun die linke Maustaste, so wird ab dem 
Punkt, "uber dem sich der Zeiger befindet, mit einem einstellbaren 
Muster (Voreinstellung: Atari-Muster) gef"ullt. Klickt man hingegen 
die rechte Maustaste, wird ab dem Punkt entweder mit schwarz oder 
wei"s gef"ullt, und zwar abh"angig davon, welche Farbe der Punkt hat: 
ist er schwarz, wird wei"s gef"ullt, ist er wei"s, wird schwarz 
gef"ullt. Nach dem F"ullen ist der F"ullmodus beendet.

Den F"ullmodus kann man ohne zu F"ullen mit der Esc-Taste abbrechen.

\subsection{Die 'Ergebnis'-K"asten} \index{Ergebnis-K"asten}
Rechts an das Arbeitsraster schlie"sen vier K"asten an. In dem 
untersten dieser wird der Datenteil des Icons schwarz auf wei"s 
dargestellt, in dem dar"uber die Maske, in dem zweitobersten das 
komplette Icon auf einem w"ahlbaren Hintergrund (Voreinstellung: 
das vom Desktop bekannte 'Grau'), im obersten das komplette Icon 
auf dem gleichen Hintergrund so, wie es angew"ahlt aussieht.

Die K"asten haben einen Rand von einem Pixel. Somit ist immer noch 
der Hintergrund in den beiden oberen sichtbar.

Klicken Sie mit der linken Maustaste einen der K"asten an, so 
geschieht folgendes: beim untersten Kasten wird in den 
Datenbearbeitungsmodus geschaltet und die Icondaten werden im 
Arbeitsraster dargestellt, beim zweituntersten wird in den 
Maskenbearbeitungsmodus geschaltet und die Maske wird im 
Arbeitsraster dargestellt. Bei den beiden oberen K"asten erscheint 
eine gestrichelte Box. Schieben Sie diese auf das Klemmbrett, 
so wird das aktuelle Icon auf das Klemmbrett gespeichert und kann 
so einfach in anderen Programmen, die ebenfalls das Klemmbrett 
benutzen, von diesem heruntergezogen werden, ohne da"s Sie sich 
durch Ordner oder Laufwerke w"uhlen m"ussen. N"aheres dazu jedoch 
gleich.

\subsection{Die Verschiebeleiste} \index{Verschiebeleiste}
Rechts von den Ergebnisk"asten ist die Verschiebeleiste sichtbar.
Diese enth"alt acht Felder mit Pfeilen, und zwar jeweils eins mit 
einem Pfeil (nach oben, unten, rechts, links) und mit zwei Pfeilen.
Klicken Sie eins der Felder mit einem Pfeil mit der linken 
Maustaste an, wird der sichtbare Teil des Icons um genau einen 
Punkt in die entsprechende Richtung verschoben. Klicken Sie ein 
Feld mit zwei Pfeilen mit der linken Maustaste an, so wird solange 
in die entsprechende Richtung verschoben, bis Sie die Maustaste 
loslassen. Klicken Sie jeweils die rechte Maustaste an, so werden
Daten {\sl und} Maske des Icons verschoben.

Sie k"onnen auch "uber die Cursortasten auf der Tastatur verschieben.
Wenn Sie zus"atzlich zu einer Cursortaste die Shifttaste dr"ucken,
so werden Daten und Maske verschoben (dies entspricht also dem 
Verschieben mit rechter Maustaste).

In der Regel wird beim Verschieben auch der Inhalt des 
Arbeitsrasters verschoben, so da"s das dort sichtbare immer eine
exakte (vergr"o"serte) Kopie des im zugeh"origen Ergebniskasten 
sichtbaren ist. Wenn Sie allerdings das Fenster oder den
Fensterinhalt so verschoben haben, da"s ein Teil des 
Arbeitsrasters au"serhalb des Bildschirms liegt, so ist das
Verschieben des Arbeitsrasterinhalts nicht m"oglich. Dann mu"s 
zum Verschieben in einen $\to$Modus geschaltet werden. Dies 
geschieht automatisch, wenn Sie ein Feld der Verschiebeleiste 
anklicken oder die Cursortasten dr"ucken und wird dadurch angezeigt, 
da"s das Wort 'Verschiebemodus' erscheint und der Mauszeiger sich 
in eine offene Hand "andert. Der Modus wird verlassen, wenn Sie
au"serhalb der Verschiebeleiste klicken oder eine der folgenden
Tasten dr"ucken: Esc, Leertaste, Return oder Undo. Dann wird auch
das Arbeitsraster aktualisiert.

\subsection{Das Klemmbrett (engl.: Clipboard)}
\index{Klemmbrett} \index{Clipboard} \index{Scrapdirectory}
\index{Zwischenablage}
Das Klemmbrett ist eine 'Zwischenablage', die dem unkomplizierten
Datenaustausch zwischen Programmen dient. Bieten zwei Programme 
beide die Klemmbrettbenutzung an und wissen sie mit den Daten des 
jeweils anderen etwas anzufangen (IconEdi kann z.B. mit Texten,
wie sie z.B. Wordplus auf das Klemmbrett schreiben kann, gar 
nichts anfangen), so kann man die Daten in dem einen Programm 
auf das Klemmbrett schieben, in das andere Programm wechseln und 
die Daten vom Klemmbrett ziehen, ohne sich erst durch die 
Dateiauswahlbox w"uhlen zu m"ussen.

Da f"ur die volle Ausnutzung der M"oglichkeiten des Klemmbretts 
mindestens zwei Programme 'sich verstehen' m"ussen, gibt es in 
IconEdi ein Problem: es gibt noch kein einziges anderes Programm, 
das mit IconEdi Daten "uber das Klemmbrett austauschen kann. 
Potentielle Kandidaten w"aren an erster Stelle alle Resource-
Editoren... nunja, mal sehen, was sich da tut. IconEdi bietet 
diese Schnittstelle jedenfalls an, und es definiert ein 
durchdachtes, erweiterbares und trotzdem einfaches Format zum 
Datenaustausch. Dieses ist das neue Standardformat von IconEdi, 
das das bisher verwendete Format abl"ost (es stammte von 
'Icondesign', einem Iconeditor von Stefan H"ohn, 
$\to$Maxon-Sonderdisk 17; dieses Format kann auch weiterhin 
gelesen werden). Die Beschreibung des Formats findet sich in 
Anhang P 'F"ur Programmierer'.

Au"serdem kann IconEdi Resource-Dateien lesen, aber nur solche, 
die genau ein Icon (oder ein Image) enthalten, und zwar als 
erstes Objekt im ersten (Dialog-)Baum. 

Es gibt noch ein Problem mit dem Klemmbrett: das Betriebssystem 
des ST/TT initialisiert dies nicht standardm"a"sig ! Ich betrachte 
dies als einen groben Fehler. Aufgrund dieses Fehlers gibt es die 
Empfehlung, da"s Programme, die das Klemmbrett benutzen wollen, 
dieses initialisieren sollen, wenn es nicht existiert.

Diese Initialisierung sieht folgenderma"sen aus: man legt einen 
Ordner an, der das Klemmbrett symbolisiert (in diesem werden dann 
auf das Klemmbrett geschobene Daten gespeichert), und 'sagt' dem 
Betriebssystem Bescheid, da"s es doch bitte anderen Programmen auf
Nachfrage sagen soll, da"s jetzt ein Klemmbrett in Form dieses 
Ordners existiert.

Ich will das aber nicht einfach irgendwo einen Ordner anlegen,
da, wo Sie ihn gar nicht haben wollen, unter einem Namen, der
Ihnen "uberhaupt nicht pa"st.

Daher verh"alt sich IconEdi entgegen der Empfehlung. Es 
initialisiert das Klemmbrett nicht automatisch. Trotzdem kann man 
nat"urlich IconEdi dazu bewegen, das Klemmbrett zu initialisieren. 
Wie das geht, ist in Kapitel 4 beschrieben. Denn das Klemmbrett ist 
auch dann ganz praktisch, wenn kein anderes Programm es nutzt: als 
Zwischenablage f"ur IconEdi selber.

Gehen wir davon aus, da"s das Klemmbrett initialisiert ist, wenn Sie
IconEdi starten (es ist z.B. initialisiert, wenn Sie die 
Gemini-Shell benutzen).

Dann befindet sich rechts oben neben der Verschiebeleiste ein Icon, 
das das Klemmbrett symbolisiert. Befindet sich etwas auf dem 
Klemmbrett, mit dem IconEdi eventuell etwas anfangen kann, so ist 
das Klemmbrett gef"ullt --- im Beispielbildschirm sehen Sie dort ein 
Bildchen mit dem Schriftzug 'ICO' (Standardextension des neuen 
Formats). Wenn Sie nun das Klemmbrett mit der linken Maustaste 
anklicken, erscheint eine gestrichelte Box. Ziehen Sie diese in das 
Arbeitsraster und lassen die Maustaste los, wird der Inhalt des 
Klemmbretts geladen.

\subsection{Sonstige Informationen im Arbeitsfenster} 
\index{Sonstige Informationen im Arbeitsfenster}
\index{Weitere Informationen im Arbeitsfenster}
\index{Zus"atzliche Informationen im Arbeitsfenster}
\index{Andere Informationen im Arbeitsfenster}
\index{Informationen im Arbeitsfenster}
Rechts neben der Verschiebeleiste werden einige Informationen 
ausgegeben. Dies sind von oben nach unten:
\begin{itemize}
\item entweder 'Icon' oder 'Image': dies gibt an, ob IconEdi
Ihr Kunstwerk als Icon oder als Image speichert. Was geeigneter 
ist, h"angt davon ab, was Sie damit tun wollen. Dazu sp"ater mehr.
\item entweder 'Daten' oder 'Maske': dies gibt an, ob sich gerade 
die Icondaten oder die Iconmaske im Arbeitsraster befinden. Da 
diese ja auch gleich sein k"onnen: hieran k"onnen Sie es erkennen.
\item entweder 'Schwarz' oder 'Wei"s': gibt die Malfarbe an. 
Bezieht sich nur auf die Funktionen, die im Men"u 'Malkasten' 
verf"ugbar sind, n"aheres dazu bei der Beschreibung dieser.
\item Gr"o"se des Icons oder Images, Breite mal H"ohe. 
\item Gr"o"se des sichtbaren Ausschnitts, Breite mal H"ohe; wird nur 
ausgegeben, wenn der sichtbare Ausschnitt kleiner ist als das 
gesamte Icon/Image.
\item der gerade aktuelle Modus. Normalerweise steht dort nichts,
befindet man sich z.B. im F"ullmodus, steht dort 'F"ullmodus'.
\end{itemize}

\newpage

\section{Die Funktionen von IconEdi}
\index{Funktionen von IconEdi}
Im folgenden werden der Reihe nach die Funktionen der Men"us und 
Dialoge beschrieben.

\subsection{Men"u 'IconEdi'} \index{Men"u IconEdi} 
\index{IconEdi-Men"u}

\ifx\bilder\undefined
 Ein Bild dieses Men"us finden Sie als 'MENU1.IMG' im Ordner 
 'BILDER' im Anleitung-Ordner des IconEdi-Pakets.
\else
 \begin{draw}{385.00}{110.00}{Men"u IconEdi}
  \put(0.00,0.00){\special{CS!g 0.7 bilder/menu1.img}}
 \end{draw}
\fi

\subsubsection{Version} \index{Version}
Standard-Tastaturk"urzel: V \\
Gibt eine Dialogbox aus, in der die Versionsnummer von IconEdi 
dargestellt wird. Die Box kann durch Dr"ucken der Return-Taste oder 
durch Anklicken mit der linken Maustaste geschlossen werden.

\subsubsection{Accessories} \index{Accessories}
Die geladenen Accessories sind unter IconEdi voll
verf"ugbar.

\subsection{Men"u 'Disk'} \index{Men"u Disk} \index{Disk-Men"u}

\ifx\bilder\undefined
 Ein Bild dieses Men"us finden Sie als 'MENU2.IMG' im Ordner 
 'BILDER' im Anleitungs-Ordner des IconEdi-Pakets.
\else
 \begin{draw}{180.00}{140.00}{Men"u Disk}
  \put(0.00,0.00){\special{CS!g 0.7 bilder/menu2.img}}
 \end{draw}
\fi


\subsubsection{Neuladen}
Standard-Tastaturk"urzel: {\control}N \\ \index{Neuladen}
L"adt das zuletzt geladene oder gespeicherte Icon oder Image
(dessen Name in der Titelzeile des Fensters steht) neu ein, ohne 
da"s eine Dateiauswahlbox erscheint. Eventuelle "Anderungen seit 
dem letzten Laden gehen dabei verloren (nicht ganz, wegen Undo !).

Steht in der Titelzeile des Fensters kein Name, sondern nur ein 
Pfad (das letzte Zeichen ist dann ein $\backslash$), wird die 
Dateiauswahlbox aufgerufen.

\subsubsection{Sichern}
Standard-Tastaturk"urzel: {\control}S \\ \index{Sichern}
Speichert das aktuelle Werk entweder als Icon oder Image (je 
nachdem, ob rechts neben der Verschiebeleiste 'Icon' oder 'Image' 
steht) unter dem in der Titelzeile stehenden Namen, ohne da"s eine 
Dateiauswahlbox erscheint. Dabei wird eine unter diesem Namen 
schon existierende Datei "uberschrieben.

Steht in der Titelzeile des Fensters kein Name, sondern nur ein 
Pfad (das letzte Zeichen ist dann ein $\backslash$), wird die 
Dateiauswahlbox aufgerufen.

\subsubsection{Laden}
Standard-Tastaturk"urzel: {\control}L \\ \index{Laden}
Eine Dateiauswahlbox erscheint, um eine Datei auszuw"ahlen, die 
geladen werden soll. Dabei wird der aktuelle Pfad angew"ahlt (der 
in der Titelzeile steht), und die Auswahl wird auf Dateien 
beschr"ankt, die die Extension haben, die Sie unter 'Extension 
"andern' f"ur das Laden einstellen k"onnen. Standardm"a"sig ist dies '*', 
so da"s alle Dateien angezeigt werden. Sie k"onnen dann eine Datei 
ausw"ahlen. Existiert die gew"ahlte nicht (Sie k"onnen ja auch "uber 
die Tastatur den Namen einer nichtexistierenden Datei eingeben) 
oder brechen Sie die Auswahl ab, passiert gar nichts. Andernfalls 
wird die Datei geladen. 

Tritt dabei kein Fehler auf (m"oglich w"aren z.B.: Ladefehler, 
IconEdi kann mit der Datei nichts anfangen usw.), wird das neue 
Icon (oder Image) dargestellt (und nat"urlich entsprechend die 
Gr"o"se des Rasters usw. angepa"st). Auch wird dann der Pfad und 
Name des Geladenen in die Fenstertitelzeile "ubernommen und der 
Pfad wird beim n"achsten Laden oder Speichern als der aktuelle 
angenommen.

F"ur das Laden spielt es keine Rolle, ob man sich im Icon- oder 
Imagebearbeitungsmodus befindet. IconEdi behandelt sowieso alles 
intern als Icon, und beim Laden kann au"serdem eindeutig 
unterschieden werden, um was es sich handelt. Dies ist nur 
f"ur das Speichern entscheidend.

\index{Laden anderer Formate}
IconEdi pr"uft, ob es mit der angegebenen Datei etwas anfangen kann. 
Dies kann beim neuen IconEdi-Format leicht und eindeutig erkannt 
werden. Damit IconEdi aber auch andere Formate laden kann, kann ein 
zus"atzliches Format unter 'Ausgabe' angegeben werden. Standardm"a"sig 
wird davon ausgegangen, da"s es sich um das Icondesign-Format 
handelt, wenn die Datei nicht im IconEdi-Format ist. IconEdi kann 
aber auch Resourcedateien (und zwar solche mit genau einem 
(Dialog-)Baum, in dem sich genau ein Icon oder Image befindet) und 
unformatierte Dateien einlesen (und mit letzterem auch Icons 
einlesen, die z.B. von dem Iconeditor von 'Opaque' 
(Maxon-Sonderdisk 22) erstellt wurden). Als was IconEdi das 
Geladene betrachten soll, k"onnen Sie unter 'Ausgabe' einstellen
(auch, wenn das mit Ausgabe wenig und mit Eingabe viel zu tun hat).

\subsubsection{Sichern unter}
Standard-Tastaturk"urzel: {\control}U \\ \index{Sichern unter}
Eine Dateiauswahlbox erscheint, um den Dateinamen auszuw"ahlen, 
unter dem gesichert werden soll. Dabei wird der aktuelle Pfad 
angew"ahlt (der in der Titelzeile steht), und die Auswahl wird auf 
Dateien beschr"ankt, die die Extension haben, die Sie unter 
'Extension "andern' f"ur das Sichern einstellen k"onnen.
Standardm"a"sig ist dies 'ICO'. Sie k"onnen dann einen Dateinamen 
ausw"ahlen. Brechen Sie die Auswahl ab oder ist der eingegebene 
\index{Umlaute in Dateinamen}
Name nicht zul"assig (IconEdi erlaubt keine Umlaute beim Sichern 
von Dateien !), passiert gar nichts. Ansonsten wird gesichert. 
Dabei wird in jedem Fall die Extension benutzt, die unter 
'Extension "andern' daf"ur eingegeben wurde, auch, wenn Sie eine 
andere eingegeben haben.

Tritt beim Sichern kein Fehler auf (m"oglich w"aren z.B.: 
Schreibfehler, nicht genug RAM-Speicher zum Aufbereiten der Daten 
vor dem Sichern usw.), wird der Pfad und Name des Gespeicherten 
in die Fenstertitelzeile "ubernommen und der Pfad wird beim 
n"achsten Laden oder Sichern als der aktuelle angenommen.

F"ur das Sichern ist es wichtig, ob man sich im Icon- oder 
Imagebearbeitungsmodus befindet. Danach entscheidet IconEdi, ob es
das Werk als Icon oder als Image sichern soll. Was sinnvoll ist, 
h"angt vom Verwendungszweck ab.

\index{Sichern anderer Formate}
Beim Sichern kann IconEdi gleichzeitig mehrere Dateien erstellen 
und damit das Werk sowohl im IconEdi-Format sichern wie als 
Resourcedatei, Sprite, Maus oder F"ullmuster. Dies wird unter 
'Ausgabe' eingestellt.

\subsubsection{Vom Klemmbrett}
Standard-Tastaturk"urzel: {\alternate}V \\ \index{Vom Klemmbrett}
Sofern sich auf dem Klemmbrett etwas befindet, mit dem IconEdi
etwas anfangen kann, wird es geladen. Der Name in der Titelzeile
wird gel"oscht. Diese Funktion ist nur zug"anglich, wenn das 
Klemmbrett vorhanden ist.

\subsubsection{Zum Klemmbrett}
Standard-Tastaturk"urzel: {\alternate}V \\ \index{Zum Klemmbrett}
Das in Arbeit befindliche Werk wird auf das Klemmbrett gespeichert,
und zwar, je nachdem, ob man sich im Icon- oder im 
Imagebearbeitungsmodus befindet, als Icon oder als Image. Es wird
im IconEdi-Format und als RSC-Datei gespeichert. Diese Funktion 
ist nur zug"anglich, wenn das Klemmbrett vorhanden ist.

\subsubsection{Arbeit sichern}
Standard-Tastaturk"urzel: {\control}A \\ \index{Arbeit sichern} 
\index{Einstellungen sichern}
Hiermit werden alle Einstellungen, die Sie in IconEdi vornehmen
k"onnen, gespeichert. Es sind dies:
\begin{itemize}
 \item alle Pfade
 \item alle aktuellen Modi
 \item Gr"o"se des Arbeitsrasters
 \item Gr"o"se des in Arbeit befindlichen Icons
 \item die von Ihnen einstellbaren F"ullmuster
 \item Farben und Flags/Status des Icons (unter 'Icontext' 
  einstellbar)
 \item die Extensions, die IconEdi benutzen soll
 \item die Einstellungen unter 'Diverses'
\end{itemize}

\subsubsection{Quit}
Standard-Tastaturk"urzel: {\control}Q \\
\index{Quit} \index{Verlassen des Programms} \index{Programmende}
\index{Ende} \index{Schluss}
Mit dieser Funktion beenden Sie IconEdi (to quit, engl.:
beenden). Standardm"a"sig wird nachgefragt, ob Sie dies auch 
wirklich wollen (mir fehlt ja daf"ur jegliches Verst"andnis).
Diese Nachfrage kann man abschalten (unter 'Diverses').
IconEdi wird auch durch Anklicken des Schlie"sfelds des Fensters
beendet.

\newpage

\subsection{Men"u 'Malkasten'} \index{Men"u Malkasten}
\index{Malkasten-Men"u}

\ifx\bilder\undefined
 Ein Bild dieses Men"us finden Sie als 'MENU3.IMG' im Ordner 
 'BILDER' im Anleitungs-Ordner des IconEdi-Pakets.
\else
 \begin{draw}{130.00}{180.00}{Men"u Malkasten}
  \put(0.00,0.00){\special{CS!g 0.7 bilder/menu3.img}}
 \end{draw}
\fi

F"ur alle der Funktionen in diesem Men"u --- mit Ausnahme der
Funktion 'Farbwechsel' --- gilt: nach dem Aufruf befindet
sich IconEdi im Malmodus, Sie m"ussen also den Modus
irgendwann beenden, bevor Sie andere Funktionen ausf"uhren k"onnen.
Da"s ein neuer Modus vorliegt, wird dadurch kenntlich gemacht,
da"s der Mauszeiger sich in ein (d"unnes) Kreuz "andert. Nun m"ussen
Sie, um z.B. eine Linie zu ziehen, den Startpunkt 
der Linie durch Anklicken eines Punktes im Arbeitsraster mit der
linken Maustaste festlegen. Wenn Sie den Startpunkt festgelegt 
haben, wird eine Hilfslinie dargestellt, damit Sie wissen, wie
die fertige Linie (in etwa) aussieht. Diese bewegt sich mit,
wenn Sie den Mauszeiger bewegen. Klicken Sie jetzt mit der
rechten Maustaste, oder mit der linken au"serhalb des 
Arbeitsrasters, oder dr"ucken Sie die Esc-Taste, verschwindet die
Linie, und Sie k"onnen einen neuen Startpunkt festlegen. Dr"ucken
Sie erneut die rechte Maustaste (oder die linke au"serhalb des
Arbeitsrasters oder die Esc-Taste), so ist der Malmodus 
verlassen und der Mauzeiger hat wieder Pfeilform. Aber wir wollten
ja eine Linie zeichnen. Also, Startpunkt festlegen, genauso 
Endpunkt festlegen, und die Linie wird gezeichnet. Sie befinden
sich dann immer noch im Malmodus und k"onnen eine neue Linie
zeichnen oder den Modus wie beschrieben abbrechen.

F"ur einige der Funktionen mu"s mehr als Start- und Endpunkt
festgelegt werden, z.B. Kreisb"ogen. Beschreibung dort.

Im Malmodus k"onnen Sie (als einzige andere Funktion) jederzeit die 
Farbe wechseln. Sie k"onnen also z.B. erst zwei wei"se Linien 
zeichnen, dann umschalten und eine schwarze zeichnen.

\subsubsection{Farbwechsel}
Standard-Tastaturk"urzel: W \\
\index{Farbwechsel} \index{Schwarz} \index{Wei"s}
Die Malfarbe wechselt (zwischen schwarz und wei"s). Die gerade 
aktuelle Farbe wird neben der Verschiebeleiste angezeigt. Die 
Malfarbe bezieht sich nur auf die Umrandung von Objekten, 
gef"ullte Objekte (z.B. Scheiben, Fl"achen) werden also mit wei"sem 
Rand gezeichnet, wenn die Malfarbe wei"s ist, sonst mit schwarzem.
Bei nicht ausgef"ullten Objekten (Linie, Rechteck etc.) ist der 
Rand das Objekt, und die Malfarbe bestimmt die Objektfarbe. 
Hilfslinien, -kreise etc. werden immer in der entgegengesetzten 
Farbe des Hintergrunds dargestellt; so sind sie am besten 
sichtbar.

\subsubsection{Radiergummi}
Standard-Tastaturk"urzel: G \\
\index{Radiergummi} \index{L"oschen im Arbeitsraster}
Haben Sie diese Funktion aufgerufen, k"onnen Sie ein 
Rechteck-Lasso um Teile des Arbeitsrasters setzen (Startpunkt 
setzen, Lasso bewegen) und durch Festlegen des Endpunkts den 
eingefangenen Teil l"oschen.

\subsubsection{Linien}
Standard-Tastaturk"urzel: L \\ \index{Linien}
Linien werden durch Festsetzen von Startpunkt und Endpunkt 
gezeichnet.

\subsubsection{Rechtecke}
Standard-Tastaturk"urzel: R \\ \index{Rechtecke}
Rechtecke werden durch Festsetzen der Startecke und der
Endecke gezeichnet.

\subsubsection{Fl"achen}
Standard-Tastaturk"urzel: F \\ \index{Fl"achen}
Fl"achen werden durch Festsetzen der Startecke und der
Endecke gezeichnet. Es wird das unter 'F"ullmuster' 
eingestellte Muster benutzt.

\subsubsection{Kreise}
Standard-Tastaturk"urzel: K \\ \index{Kreise}
Kreise werden durch Festsetzen des Mittelpunkts und eines
Kreisbogenpunkts gezeichnet.

\subsubsection{Kreisb"ogen}
Standard-Tastaturk"urzel: {\alternate}K \\ \index{Kreisb"ogen}
Kreisb"ogen werden durch Festsetzen des Mittelpunkts, eines
Kreisbogenpunkts, des Startpunkts des Bogens und des Endpunkts
des Bogens gezeichnet. Es sind also 4 Klicks n"otig. Bricht
man ab (rechte Maustaste, Esc, linke Maustaste au"serhalb des
Rasters), wird nur der zuletzt gezeichnete Punkt zur"uckgenommen.

\subsubsection{Scheiben}
Standard-Tastaturk"urzel: S \\ \index{Scheiben}
Scheiben werden durch Festsetzen des Mittelpunkts und eines
Kreisbogenpunkts gezeichnet. Es wird das unter 'F"ullmuster' 
eingestellte Muster benutzt.

\subsubsection{Sch.segmente}
Standard-Tastaturk"urzel: {\alternate}S \\ \index{Scheibensegmente}
Scheibensegmente werden durch Festsetzen des Mittelpunkts, eines
Kreisbogenpunkts, des Startpunkts des Segments und des Endpunkts
des Segments gezeichnet. Es sind also 4 Klicks n"otig. Bricht
man ab (rechte Maustaste, Esc, linke Maustaste au"serhalb des
Rasters), wird nur der zuletzt gezeichnete Punkt zur"uckgenommen.
Es wird das unter 'F"ullmuster' eingestellte Muster benutzt.

\subsubsection{Ellipsen}
Standard-Tastaturk"urzel: E \\ \index{Ellipsen}
Ellipsen werden durch Festsetzen des Mittelpunkts und eines
Ellipsenbogenpunkts gezeichnet.

\subsubsection{Ellipsenb"ogen}
Standard-Tastaturk"urzel: {\alternate}E \\ \index{Ellipsenb"ogen}
Ellipsenb"ogen werden durch Festsetzen des Mittelpunkts, eines
Ellipsenbogenpunkts, des Startpunkts des Bogens und des Endpunkts
des Bogens gezeichnet. Es sind also 4 Klicks n"otig. Bricht
man ab (rechte Maustaste, Esc, linke Maustaste au"serhalb des
Rasters), wird nur der zuletzt gezeichnete Punkt zur"uckgenommen.

\subsubsection{PEllipsen}
Standard-Tastaturk"urzel: P \\ 
\index{Gef"ullte Ellipsen} \index{PEllipsen}
Gef"ullte Ellipsen werden durch Festsetzen des Mittelpunkts und 
eines Ellipsenbogenpunkts gezeichnet. Es wird das unter 
'F"ullmuster' eingestellte Muster benutzt.

\subsubsection{PEll.segmente}
Standard-Tastaturk"urzel: {\alternate}P \\
\index{Segmente gef"ullter Ellipsen} \index{PEllipsensegmente}
\index{Gef"ullte-Ellipsen-Segmente}
Segmente gef"ullter Ellipsen werden durch Festsetzen des 
Mittelpunkts, eines Ellipsenbogenpunkts, des Startpunkts des 
Segments und des Endpunkts des Segments gezeichnet. Es sind also 
4 Klicks n"otig. Bricht man ab (rechte Maustaste, Esc, linke 
Maustaste au"serhalb des Rasters), wird nur der zuletzt gezeichnete 
Punkt zur"uckgenommen. Es wird das unter 'F"ullmuster' eingestellte 
Muster benutzt.

\newpage


\subsection{Men"u 'Standard'} \index{Standard-Men"u}
\index{Men"u Standard}

\ifx\bilder\undefined
 Ein Bild dieses Men"us finden Sie als 'MENU4.IMG' im Ordner 
 'BILDER' im Anleitungs-Ordner des IconEdi-Pakets.
\else
 \begin{draw}{110.00}{190.00}{Men"u Standard}
  \put(0.00,0.00){\special{CS!g 0.7 bilder/menu4.img}}
 \end{draw}
\fi

\subsubsection{Image/Icon}
Standard-Tastaturk"urzel: Tab \\
\index{Image/Icon} \index{Icon/Image}
Wechselt zwischen Icon- und Imagebarbeitungsmodus.
Dieser bestimmt, ob beim Speichern als Icon oder als
Image gespeichert wird.

\subsubsection{Zur"uck}
Standard-Tastaturk"urzel: Undo \\
\index{Zur"uck} \index{Undo}
Macht eine Aktion r"uckg"angig. Grunds"atzlich sind so
alle Aktionen revidierbar, die Daten und/oder Maske
vera"ndern und die man nicht durch Wiederholung der 
Funktion r"uckg"angig machen kann (wie z.B. Tauschen, 
Verschieben und Spiegeln). Ausgenommen ist das Freihandzeichnen. 
Ausgenommen ist auch teilweise das Laden und die "Anderung 
der Gr"o"se: L"adt man ein mindestens genau so grosses Icon
oder Image oder vergr"o"sert, ist ein zur"uck ohne 
Datenverlust m"oeglich, sonst nicht.

\subsubsection{Wechsel}
Standard-Tastaturk"urzel: Space (Leertaste) \\ \index{Wechsel}
Wechselt zwischen Daten- und Maskenbearbeitung. Wechseln 
kann man auch durch Anklicken der beiden unteren 
Ergebnisk"asten.

\subsubsection{Tauschen}
Standard-Tastaturk"urzel: T \\ \index{Tauschen}
Vertauscht Daten und Maske. 
Bezieht sich auf die kompletten Daten/Maske (bei sehr grossen
Icons wird ja nur ein Teil im Raster dargestellt).
Wird durch Wiederholung r"uckg"angig gemacht.

\subsubsection{Invert}
Standard-Tastaturk"urzel: I \\ \index{Invertieren}
Invertiert den im Raster sichtbaren Teil. 
Wird durch Wiederholung r"uckg"angig gemacht.

\subsubsection{Copy}
Standard-Tastaturk"urzel: C \\ \index{Copy} \index{Kopieren}
Kopiert aus dem in Arbeit befindlichen Teil 
(z.B. Daten) in den anderen (z.B. Maske).
Bezieht sich auf die kompletten Daten/Maske (bei sehr grossen
Icons wird ja nur ein Teil im Raster dargestellt).

\subsubsection{XSpiegel}
Standard-Tastaturk"urzel: X \\
\index{XSpiegel} \index{Spiegeln an der X-Achse}
Spiegelt den im Raster sichtbaren Teil an der (mittleren) X-Achse
(also der Horizontalachse).
Wird durch Wiederholung r"uckg"angig gemacht.

\subsubsection{YSpiegel}
Standard-Tastaturk"urzel: Y \\
\index{YSpiegel} \index{Spiegeln an der Y-Achse}
Spiegelt den im Raster sichtbaren Teil an der (mittleren) Y-Achse
(also der Vertikalachse).
Wird durch Wiederholung r"uckg"angig gemacht.

\subsubsection{XCopy}
Standard-Tastaturk"urzel: {\control}X \\
\index{XCopy} \index{Kopieren "uber die X-Achse}
Kopiert von der oberen H"alfte des im Raster sichtbaren Teils
auf die untere an der (mittleren) X-Achse entlang.

\subsubsection{YCopy}
Standard-Tastaturk"urzel: {\control}Y \\
\index{YCopy} \index{Kopieren "uber die Y-Achse}
Kopiert von der linken H"alfte des im Raster sichtbaren Teils
auf die rechte an der (mittleren) Y-Achse entlang.

\subsubsection{Dreh 90G}
Standard-Tastaturk"urzel: * \\
\index{Drehen um 90 Grad} \index{Um 90 Grad drehen}
Dreht den im Raster sichtbaren Teil um 90 Grad.

\subsubsection{Dreh 180G}
Standard-Tastaturk"urzel: - \\
\index{Drehen um 180 Grad} \index{Um 180 Grad drehen}
Dreht den im Raster sichtbaren Teil um 180 Grad.

\subsubsection{Dreh 270G}
Standard-Tastaturk"urzel: + \\
\index{Drehen um 270 Grad} \index{Um 270 Grad drehen}
Dreht den im Raster sichtbaren Teil um 270 Grad.

\newpage


\subsection{Men"u 'Spezial'} \index{Spezial-Men"u}
\index{Men"u Spezial}

\ifx\bilder\undefined
 Ein Bild dieses Men"us finden Sie als 'MENU5.IMG' im Ordner 
 'BILDER' im Anleitungs-Ordner des IconEdi-Pakets.
\else
 \begin{draw}{130.00}{210.00}{Men"u Spezial}
  \put(0.00,0.00){\special{CS!g 0.7 bilder/menu5.img}}
 \end{draw}
\fi

\subsubsection{Outline-Maske}
Standard-Tastaturk"urzel: {\control}O \\
\index{Automatische Maskenerstellung} \index{Outline-Maske}
Erstellt automatisch eine Maske f"ur die kompletten Daten, 
indem die Daten in die Maske kopiert und um jeden Punkt 
der Kopie herum Punkte gesetzt werden.

\subsubsection{Fill-Maske}
Standard-Tastaturk"urzel: {\control}F \\ \index{Fill-Maske}
Erstellt automatisch eine Maske f"ur die kompletten Daten, 
indem die Daten in die Maske kopiert und geschlossene
Bereiche in der Kopie gef"ullt werden.

\subsubsection{Fill\&Out-Maske}
Standard-Tastaturk"urzel: {\control}M \\ \index{Fill\&Out-Maske}
Erstellt automatisch eine Maske f"ur die kompletten Daten, 
indem die Daten in die Maske kopiert, geschlossene Bereiche
in der Kopie gef"ullt und dann um jeden Punkt herum Punkte 
gesetzt werden.

\subsubsection{Gemini-Maske}
Standard-Tastaturk"urzel: {\control}G \\ \index{Gemini-Maske}
Erstellt automatisch eine Maske f"ur die kompletten Daten, 
indem die Daten in die Maske kopiert, geschlossene
Bereiche in der Kopie gef"ullt werden und dann noch die Punkte
in der Maske gesetzt werden, die in den Daten gesetzt sind.

\subsubsection{Import}
Standard-Tastaturk"urzel: {\control}I \\
\index{Import} \index{Aus Bildern importieren}

\ifx\bilder\undefined
 Ein Bild der nach Funktionsaufruf erscheinenden Dialogbox
 finden Sie als 'IMPORT.IMG' im Ordner 
 'BILDER' im Anleitungs-Ordner des IconEdi-Pakets.
\else
 \begin{draw}{360.00}{240.00}{Die Import-Dialogbox}
  \put(0.00,0.00){\special{CS!g 0.6 bilder/import.img}}
 \end{draw}
\fi

Diese Funktion erlaubt es, aus Bildern Daten oder Maske
zu 'importieren'. 
Sie k"onnen hier Bilder laden, speichern, ein leeres Bild 
erzeugen (Ausma"se: 640 * 400; wichtig f"ur den Export, wenn
man kein Bild geladen hat), ein geladenes verwerfen (l"oschen),
und zum n"achsten geladenen Bild bl"attern. Das jeweils aktuelle
Bild wird in einer Box dargestellt (siehe oben). Ist das Bild 
gr"o"ser als 'am St"uck' darstellbar, k"onnen Sie mit den
Pfeilbuttons oder den Cursortasten das Bild verschieben: mit
den einzelnen Pfeilen oder Cursortasten zeilenweise (punktweise),
mit den doppelten Pfeilen oder den Cursortasten zusammen mit
der Shift-Taste seitenweise (um 100 Punkte). 

Geladen werden k"onnen folgende Bildformate:
\begin{itemize}
 \item das GEM-Image-Format (IMG)
 \item das 'STAD'-Format (h"aufig mit der Extension 'PAC')
 \item das 'Degas'-monochrom-Format (PI3)
 \item das Screen-/'Doodle'-Format (PIC/DOO)
\end{itemize}

Gespeichert werden kann im GEM-Image-Format und im 
'Doodle'-Format. Dies k"onnen Sie unter 'Diverses' einstellen;
hat ein Bild aber andere Ausma"se als 640 * 400 Punkte, wird
es immer im GEM-Image-Format abgespeichert, egal, was Sie
eingestellt haben.

Wenn Sie aus einem Bild in einem anderen als den
aufgef"uhrten Formaten importieren wollen, m"ussen Sie
es erst in eines dieser Formate konvertieren. Es gibt
inzwischen eine ganze Reihe von Bildkonvertern f"ur ST/TT,
auch unter den als Public-Domain erh"altlichen finden sich
sehr brauchbare. Die meisten Malprogramme sind auch in der 
Lage, direkt in einem der aufgef"uhrten Formate zu speichern.

Wie wird nun importiert ? Sie haben ein Bild geladen. Bewegen
Sie dann den Mauszeiger in das Bild und dr"ucken die
linke Maustaste. Ein Rahmen in der (gew"ahlten) Icongr"o"se 
erscheint. Halten Sie die Maustaste gedr"uckt und bewegen den
Mauszeiger, wandert der Rahmen mit. So k"onnen Sie einen 
Ausschnitt des Bilds w"ahlen. Klicken Sie dann auf 'OK', wird
der Ausschnitt "ubernommen.

Haben Sie keine Auswahl getroffen und klicken 'OK', wird nichts
"ubernommen.  

\subsubsection{Export}
Standard-Tastaturk"urzel: {\control}E \\
\index{Export} \index{In Bilder exportieren}
Diese Funktion erlaubt es, in Bilder Daten oder Maske
zu 'exportieren'. Nach dem Aufruf dieser Funktion erscheint 
die unter 'Import' dargestellte und beschriebene Dialogbox.

Wie wird nun exportiert ? Sie haben ein Bild geladen. Bewegen
Sie dann den Mauszeiger in das Bild und dr"ucken die
linke Maustaste. Daten oder Maske des in Arbeit befindlichen
Icons erscheinen und k"oennen bei gedr"ckter linker Maustaste
durch Verschieben des Mauszeigers im Bild plaziert werden.
Klicken Sie dann auf 'OK', wird an die gew"ahlte Position
im Bild kopiert.
 
\subsubsection{Daten l"oschen}
Standard-Tastaturk"urzel: Home \\
\index{Daten l"oschen} \index{L"oschen der Daten}
L"oscht die Daten.

\subsubsection{Maske l"oschen}
Standard-Tastaturk"urzel: Clr \\
\index{Maske l"oschen} \index{L"oschen der Maske}
L"oscht die Maske.

\subsubsection{Beide l"oschen} 
Standard-Tastaturk"urzel: Del \\
\index{Beide l"oschen} \index{L"oschen von Daten und Maske}
L"oescht Daten und Maske.

\subsubsection{RSC${\Rightarrow}$Icon}
Standard-Tastaturk"urzel: {\alternate}C \\
\index{RSC${\Rightarrow}$Icon} \index{Aus RSC-Dateien 'klauen'}
Diese Funktion sucht Icons (und Images) aus RSC-Dateien, wandelt
sie in das IconEdi-Format und speichert sie ab. Dazu erscheint
die Dateiauswahlbox. Hier w"ahlt man eine RSC-Datei aus. Unter
dem Namen dieser RSC-Datei wird dann in dem Pfad, der in der
Fenstertitelzeile sichtbar ist, ein Ordner angelegt; in diesen 
werden die gefundenen Icons gespeichert. Die Namen der Icons werden
automatisch vergeben: und zwar wird, falls das Icon einen Text hat,
dieser als Name genommen --- eventuell enthaltene Sonderzeichen,
die nicht in Dateinamen benutzt werden d"urfen, werden dabei
ausgefiltert. Existiert eine Datei unter diesem Namen schon, 
werden an den Namen solange andere Zahlen angehangen, bis ein Name
entstanden ist, unter dem noch keine Datei existiert. 

W"ahrend der Suche wird in einer Infobox ausgegeben, ob das
gerade untersuchte Objekt ein Icon oder Image ist oder nicht;
handelt es sich um ein solches, wird es dargestellt. Die
Suche kann durch Dr"ucken der Esc-Taste abgebrochen werden  
(eventuell mu"s diese mehrmals gedr"uckt oder gedr"uckt gehalten 
werden).

\subsubsection{F"ullmustertest}
Standard-Tastaturk"urzel: {\alternate}F \\ \index{F"ullmustertest}

\ifx\bilder\undefined
 Ein Bild der nach Funktionsaufruf erscheinenden Dialogbox
 finden Sie als 'FUELMTST.IMG' im Ordner 
 'BILDER' im Anleitungs-Ordner des IconEdi-Pakets.
\else
 \begin{draw}{370.00}{180.00}{Die F"ullmustertest-Dialogbox}
  \put(0.00,0.00){\special{CS!g 0.7 bilder/fuelmtst.img}}
 \end{draw}
\fi

Sie k"oennen mit IconEdi auch F"ullmuster erstellen. Mit dieser
Funktion kann ein erstelltes Muster getestet werden. 
Als F"ullmuster wird dazu ein Ausschnitt von 16 * 16 Punkten aus
der linken oberen Ecke der Icondaten interpretiert. Der Test wird
durch Anklicken von 'OK' beendet.


\subsubsection{Sprite/Maustest}
Standard-Tastaturk"urzel: {\alternate}T \\
\index{Spritetest} \index{Maustest} \index{Sprite/Maustest}

\ifx\bilder\undefined
 Ein Bild der nach Funktionsaufruf erscheinenden Dialogbox
 finden Sie als 'MAUSTEST.IMG' im Ordner 
 'BILDER' im Anleitungs-Ordner des IconEdi-Pakets.
\else
 \newpage
 \begin{draw}{180.00}{170.00}{Die Sprite/Maustest-Dialogbox}
  \put(0.00,0.00){\special{CS!g 0.7 bilder/maustest.img}}
 \end{draw}
\fi

Sie k"oennen mit IconEdi auch Mauszeiger (und Sprites) erstellen.
Mit dieser Funktion kann ein erstellter Mauszeiger getestet werden. 
Das Testergebnis ist auf Sprites "ubertragbar.
Als Mauszeiger wird ein Ausschnitt von 16 * 16 Punkten aus
der linken oberen Ecke von Daten und Maske interpretiert. 
Der Aktionspunkt des Mauszeigers kann mit 'S/M-Akt.punkt'
festgelegt werden. Wenn der Dialog erscheint, wird der Mauszeiger
f"ur den Test ge"andert, und man kann Aussehen und Klickverhalten
anhand der Buttons (1-3) und der Hintergrundmuster in der Dialogbox
beurteilen. Der Test wird durch Anklicken von 'OK' beendet (oder
durch Dr"ucken der Return-Taste).

\subsubsection{S/M-Akt.punkt}
Standard-Tastaturk"urzel: {\alternate}N \\
\index{Aktionspunkt} \index{Mausaktionspunkt} 
\index{Spriteaktionspunkt}
Nach Aufruf dieser Funktion kann man X- und Y-Position f"ur
den Maus- bzw. Sprite-Aktionspunkt festlegen. Dazu erscheint
zweimal eine Dialogbox zur Zahleingabe. 

\subsubsection{S/M-Farbe}
Standard-Tastaturk"urzel: {\alternate}Y \\
\index{S/M-Farbe} \index{Spritefarbe} \index{Mausfarbe}

\ifx\bilder\undefined
 Ein Bild der nach Funktionsaufruf erscheinenden Dialogbox
 finden Sie als 'SMFARB.IMG' im Ordner 
 'BILDER' im Anleitungs-Ordner des IconEdi-Pakets.
\else
 \begin{draw}{280.00}{100.00}{Die Sprite/Mausfarbe-Dialogbox}
  \put(0.00,0.00){\special{CS!g 0.7 bilder/smfarb.img}}
 \end{draw}
\fi

Hier k"oennen Spritehintergrund- bzw. Mausmaskenfarbe,
Spritevordergrund- bzw. Mauscursorfarbe und das Format (nur
f"ur Sprites) festgelegt werden.

\newpage


\subsection{Men"u 'Optionen'} \index{Optionen-Men"u}
\index{Men"u Optionen}

\ifx\bilder\undefined
 Ein Bild dieses Men"us finden Sie als 'MENU6.IMG' im Ordner 
 'BILDER' im Anleitungs-Ordner des IconEdi-Pakets.
\else
 \begin{draw}{120.00}{210.00}{Men"u Optionen}
  \put(0.00,0.00){\special{CS!g 0.7 bilder/menu6.img}}
 \end{draw}
\fi

\subsubsection{Icontext}
Standard-Tastaturk"urzel: {\alternate}I \\ \index{Icontext} 
\ifx\bilder\undefined
 Ein Bild der nach Funktionsaufruf erscheinenden Dialogbox
 finden Sie als 'ICONTEXT.IMG' im Ordner 
 'BILDER' im Anleitungs-Ordner des IconEdi-Pakets.
\else
 \begin{draw}{360.00}{220.00}{Die Icontext-Dialogbox}
  \put(0.00,0.00){\special{CS!g 0.6 bilder/icontext.img}}
 \end{draw}
\fi

Hier k"onnen Sie alle Einstellungen vornehmen, die ein Icon
neben Daten und Maske 'ausmachen'. Dazu geh"oren:
\begin{itemize}
 \item alle Objekt-Flags und -Stati, die f"ur ein Icon relevant 
       sind: diese k"onnen mit den entsprechenden Buttons 
       eingestellt werden
 \item die Farbe der gesetzten Punkte (GP-Farbe, Daten) und der 
       nicht gesetzten Punkte (NGP-Farbe, Maske)
 \item der Icon-Buchstabe
 \item der Icontext
 \item die Positionen von Icontext, Buchstabe und der Wirkbereich:
       diese k"onnen durch Anklicken und gedr"uckt Halten der
       linken Maustaste verschoben werden
\end{itemize}

Wenn man Icontext, -farben oder -buchstaben ver"andert, mu"s man
die Ver"anderungen 'setzen', damit sie angezeigt werden. Bei 
(sehr) gro"sen Icons ist die Anzeigebox nicht aktiv. 

\subsubsection{F"ullmuster}
Standard-Tastaturk"urzel: {\alternate}M \\ \index{F"ullmuster} 

\ifx\bilder\undefined
 Ein Bild der nach Funktionsaufruf erscheinenden Dialogbox
 finden Sie als 'FUELLMUS.IMG' im Ordner 
 'BILDER' im Anleitungs-Ordner des IconEdi-Pakets.
\else
 \begin{draw}{380.00}{180.00}{Die F"ullmuster-Dialogbox}
  \put(0.00,0.00){\special{CS!g 0.6 bilder/fuellmus.img}}
 \end{draw}
\fi

Hier kann das F"ullmuster gew"ahlt werden, das beim n"achsten F"ullen
oder beim Zeichnen gef"ullter Objekte (Scheiben, Fl"achen) benutzt
werden soll. Dazu klickt man einfach eines der dargestellten Muster 
an. Eigene Muster k"onnen geladen, gespeichert und aus dem
Editor "ubernommen werden. 

\subsubsection{Hintergrund}
Standard-Tastaturk"urzel: {\alternate}H \\ \index{Hintergrund}
Es erscheint die gleiche Dialogbox wie unter 'F"ullmuster'.
Das hier gew"ahlte Muster wird in den beiden oberen Ergebnisk"asten
und im Icontext-Dialog als Hintergrund verwendet.

\subsubsection{Andere Gr"o"se}
Standard-Tastaturk"urzel: 0 \\
\index{Andere Gr"o"se} \index{Gr"o"se}

\ifx\bilder\undefined
 Ein Bild der nach Funktionsaufruf erscheinenden Dialogbox
 finden Sie als 'GROESSE.IMG' im Ordner 
 'BILDER' im Anleitungs-Ordner des IconEdi-Pakets.
\else
 \begin{draw}{190.00}{210.00}{Die Gr"o"seneinstellungs-Dialogbox}
  \put(0.00,0.00){\special{CS!g 0.7 bilder/groesse.img}}
 \end{draw}
\fi

Zur Wahl einer neuen Icongr"o"se. Sie m"ussen erst Breite, dann H"ohe
eingeben. Geben Sie f"ur eines der beiden Null ein (zum Beispiel,
indem Sie ohne Eingabe 'OK' bet"atigen), bleibt der alte Wert 
erhalten.

Als Breite wird nur ein Vielfaches von 16 akzeptiert, der 
eingegebene Wert wird gegebenenfalls aufgerundet. Aus z.B. 40 wird 
also 48. Mindestwert f"ur die Breite ist 16, f"ur die H"ohe 8.
Auch hier wird gegebenenfalls aufgerundet. Bei allen 
Ver"anderungen der Icongr"o"se wird (falls m"oglich) die Rastergr"o"se 
angepa"st.

\subsubsection{16 * 16}
Standard-Tastaturk"urzel: 1 \\ \index{16 * 16}
W"ahlt als Icongr"o"se 16 * 16.

\subsubsection{32 * 32}
Standard-Tastaturk"urzel: 3 \\ \index{32 * 32}
W"ahlt als Icongr"o"se 32 * 32.

\subsubsection{48 * 48}
Standard-Tastaturk"urzel: 4 \\ \index{48 * 48}
W"ahlt als Icongr"o"se 48 * 48.

\subsubsection{64 * 64}
Standard-Tastaturk"urzel: 6 \\ \index{64 * 64}
W"ahlt als Icongr"o"se 64 * 64.

\subsubsection{Maximal}
Standard-Tastaturk"urzel: 9 \\ \index{Maximale Gr"o"se}
W"ahlt als Icongr"o"se die maximal m"ogliche. Diese h"angt von 
der Bildschirmgr"o"se ab und betr"agt maximal 640 * 400 Punkte.

\subsubsection{Ausgabe}
Standard-Tastaturk"urzel: {\alternate}A \\
 \index{Ausgabe} \index{Eingabe}

\ifx\bilder\undefined
 Ein Bild der nach Funktionsaufruf erscheinenden Dialogbox
 finden Sie als 'AUSGABE.IMG' im Ordner 
 'BILDER' im Anleitungs-Ordner des IconEdi-Pakets.
\else
 \begin{draw}{230.00}{170.00}{Die Ausgabe-Dialogbox}
  \put(0.00,0.00){\special{CS!g 0.6 bilder/ausgabe.img}}
 \end{draw}
\fi

Hier kann eingestellt werden, in welche Formate zus"atzlich
gespeichert werden sollen und welches Format zum Laden benutzt
werden soll, wenn eine Datei nicht im IconEdi-Format vorliegt.

Au"ser im IconEdi-Format k"onnen gespeichert werden:
\begin{itemize}
 \item F"ullmusterdaten 
 \item Mausdaten
 \item Spritedaten
 \item RSC-Datei (mit dem Icon oder Image)
\end{itemize}

Au"ser im IconEdi-Format k"onnen geladen werden:
\begin{itemize}
 \item 'Icondesign'-Format
 \item Unformatierte Daten (hier m"ussen Breite und H"ohe angegeben 
       werden; so kann man auch Daten des Iconeditors von 'Opaque'
       laden)
 \item RSC-Dateien (mit genau einem Baum und einem Icon oder Image
       in diesem)
\end{itemize}

Es kann f"ur die Eingabe nur ein zus"atzliches Format angegeben 
werden, da es keine M"oglichkeit gibt, die Formate eindeutig zu 
unterscheiden. Das IconEdi-Format wird immer erkannt, egal, was
Sie hier eingestellt haben. 

\subsubsection{Raster}
Standard-Tastaturk"urzel: {\alternate}R \\
\index{Raster an/ausschalten}
Schaltet das Raster an bzw. aus.

\subsubsection{Rastergr"o"se}
Standard-Tastaturk"urzel: {\alternate}O \\
 \index{Gr"o"se des Rasters} \index{Rastergr"o"se}

Hier legt man die maximale Gr"o"se des Rasters fest. Dadurch
kann man bestimmen, wie klein die Punkte im Raster minimal werden,
auch wenn man sehr gro"se Icons bearbeitet.
Es erscheint zweimal die Dialogbox zur Zahleingabe (siehe
unter 'Andere Gr"o"se') zur Eingabe von Breite und H"ohe.

\subsubsection{Ext. "andern}
Standard-Tastaturk"urzel: {\alternate}X \\
 \index{"Andern der Extensions} \index{Extensions "andern}
 \index{Dateierweiterungen "andern}

\ifx\bilder\undefined
 Ein Bild der nach Funktionsaufruf erscheinenden Dialogbox
 finden Sie als 'EXTAENDE.IMG' im Ordner 
 'BILDER' im Anleitungs-Ordner des IconEdi-Pakets.
\else
 \begin{draw}{200.00}{210.00}{Die Extension-"andern-Dialogbox}
  \put(0.00,0.00){\special{CS!g 0.6 bilder/extaende.img}}
 \end{draw}
\fi

Hier werden die $\to$Extensions eingestellt, die IconEdi
beim Laden und Speichern verwenden soll.

\subsubsection{Diverses}
Standard-Tastaturk"urzel: {\alternate}D \\
 \index{Diverse Einstellungen}
\ifx\bilder\undefined
 Ein Bild der nach Funktionsaufruf erscheinenden Dialogbox
 finden Sie als 'DIVERSES.IMG' im Ordner 
 'BILDER' im Anleitungs-Ordner des IconEdi-Pakets.
\else
 \begin{draw}{210.00}{190.00}{Die Diverses-Dialogbox}
  \put(0.00,0.00){\special{CS!g 0.6 bilder/diverses.img}}
 \end{draw}
\fi

Hier wird eingestellt, ob beim Programmende nachgefragt werden
soll (in der Voreinstellung wird nachgefragt), in welchem
Format Bilder gespeichert werden sollen (Voreinstellung: IMG-Format)
und hier kann --- wenn n"otig --- das Klemmbrett initialisiert 
werden (dazu m"ussen Sie erst den Pfad f"ur dieses festlegen).



%\input chapter3.tex
\chapter{Beispielanwendungen} \index{Beispielanwendungen}
Sie k"onnen mit IconEdi wundersch"one Icons (und Images,
F"ullmuster....) erstellen. Beispiele daf"ur, was Sie dann 
mit diesen anfangen k"onnen, finden Sie in diesem Kapitel.

\section{Icons in RSC-Dateien "ubernehmen}
\index{Icons in RSC-Dateien "ubernehmen}
Wenn Sie unter 'Ausgabe' 'RSC-Datei' eingestellt haben, erzeugt 
IconEdi beim Speichern zus"atzlich eine RSC-Datei, die das Icon oder
Image enth"alt. Diese Datei kann in jeden mir bekannten Resource-
Editor "ubernommen und das Icon daraus in eine andere Resourcedatei
verschoben werden. So k"onnen mit IconEdi erstellte Icons auch in 
die Icon-Resourcedatei von Gemini oder des TOS ab Version 2.0 
gepackt werden.

Sch"oner w"are es, wenn die Resourceeditoren das Klemmbrett benutzen
w"urden oder direkt das IconEdi-Format lesen k"onnten. Nun ja,
mal schauen, was die Zukunft bringt...


\subsection{DR-RCS} \index{DR-RCS}
Die Resource-Contruction-Sets von Digital Research sind sehr 
verbreitet, da die Version 1.x als $\to$Maxon-Sonderdisk
erh"altlich ist und die Version 2.x vielen 
Software-Entwicklungspaketen beiliegt.

Diese RCS erlauben das Hinzuladen von einer oder mehreren 
Resourcen "uber den Men"upunkt 'Merge'. Haben Sie also eine
Resourcedatei geladen, k"onnen Sie die von IconEdi erstellte 
Resourcedatei hinzuladen. Die RCS melden, da"s keine 
Definitionsdatei gefunden wurde ('Can't find definition file...').
"Ubergehen Sie dies, indem Sie auf OK klicken. Sie sehen nun
im Arbeitsfenster des RCS ein Fragezeichen und darunter 'TREEx'
(x ist eine Zahl). Doppelklicken Sie auf dieses. Eine Dialogbox
erscheint, um den Typ dieses Baums (tree, engl.: Baum) zu "andern.
"Andern Sie ihn in 'Dialog' und klicken auf 'OK'. Doppelklicken
Sie nun auf das Dialogicon des Baums, und Sie sehen eine Dialogbox
mit dem erstellten Icon darin. Klicken Sie das Icon an (es wird
invertiert), und w"ahlen Sie den Men"upunkt 'Copy'. Das Icon wird dann
auf das Klemmbrett des RCS kopiert. Schlie"sen und l"oschen Sie
den Dialogbaum. Das Icon k"onnen Sie nun vom Klemmbrett in eine 
eigene Dialogbox ziehen.

\subsection{Kuma-Resourceeditor} \index{Kuma-Resourceeditor}
\index{NRSC.PRG}
Dieser Resourceeditor erlaubt das Bearbeiten mehrerer RSC-Dateien
gleichzeitig. Zur Bedienung schauen Sie bitte in dessen Handbuch.
Von IconEdi erstellte RSC-Dateien k"onnen hier problemlos geladen
werden; die Meldung 'Kann die RSD-datei nicht "offnen' "ubergehen
Sie einfach, der Editor nimmt dann an, da"s es sich um einen 
Dialogbaum handelt. In diesem finden Sie das Icon.

\subsection{WERCS} \index{WERCS}
Dieser Resourceeditor von HiSoft erlaubt nicht die Bearbeitung 
mehrerer RSC-Dateien. "Uber 'Import Image' ist es aber m"oglich, eine
von IconEdi erstellte RSC-Datei (bzw. das Icon darin) hinzuzuladen.
Dieses k"onnen Sie dann mit 'Paste' (Einf"ugen) in Ihre
RSC-Datei einf"ugen.

\subsection{Interface} \index{Interface}
Interface kann ebenfalls mehrere RSC-Dateien bearbeiten. Die
Vorgehensweise ist "ahnlich der beim Kuma-Resourceeditor.

\section{Icons in Programmen laden}
\index{Icons in Programmen laden}
Die erzeugten Icons (Images, F"ullmuster...) k"onnen auch 
direkt aus Programmen heraus geladen werden. Die Formate,
die IconEdi erzeugt, finden Sie in Anhang P beschrieben.

\section{Icons in Programmquelltext einf"ugen}
\index{Icons in Programmquelltext einf"ugen}
Die erzeugten Icons (Images, F"ullmuster...) k"onnen auch 
direkt in den Programmtext eingef"ugt werden. Dies ist in 
GFA-Basic dank des Inline-Befehls sehr leicht m"oglich.
Andere Programmiersprachen bieten eine solche Option nicht;
hier mu"s das Icon als Quelltext vorliegen.

Die Erstellung solcher Quelltexte bietet IconEdi nicht an.
Es ist m"oglich, da"s ich einmal eine derartige Option einbaue;
ich betrachte dies aber nicht als so wichtig, da sich jeder
Programmierer leicht ein Programm schreiben kann, das aus dem
IconEdi-Format einen Quelltext macht --- und zwar so, wie
derjenige es haben will. Zudem graust mir vor dem Gedanken,
die verschiedenen Geschm"acker ber"ucksichtigen zu m"ussen.
Sicher ist aber nur eins: eine Quelltextunterst"utzung 
wird es nicht f"ur Omikron-Basic geben. 

Die Formate, die IconEdi erzeugt, finden Sie in Anhang P 
beschrieben.

\section{Icons in Gemini einf"ugen} \index{Icons in Gemini einf"ugen}
Zu Gemini geh"ort eine Datei GEMINIIC.RSC. In dieser sind die
Icons, die Gemini f"ur Laufwerke, Dateien, Papierkorb usw. benutzt,
gespeichert. Die Datei enth"alt drei Objektb"aume: in dem ersten
sind die Laufwerksicons, die f"ur Papierkorb und Shredder und 
f"ur das Klemmbrett gespeichert, im zweiten die gro"sen Dateiicons,
im dritten die kleinen Dateiicons.
 
Bei den Icons im ersten Baum (Laufwerke etc.)
sind mehrere Zeilen mit jeweils 6 Icons enthalten; das erste
Icon ist das links oben, das sechste das rechts oben, das links
in der 2. Zeile ist das siebte und so weiter (die 
Sortierreihenfolge entspricht also der Leserichtung im Deutschen:
von links oben zeilenweise nach rechts unten; diese Reihenfolge
mu"s bei allen drei B"aumen der Datei eingehalten werden). Das erste
Icon ist das des 'Schredders', dann folgen M"ulleimer leer, M"ulleimer
voll, Klemmbrett leer, Klemmbrett voll und dann die Laufwerksicons
(in 'beliebiger' Anzahl). In den beiden anderen B"aumen sind die
Icons zu sieben je Zeile angeordnet. Die Anzahl pro Zeile m"u"ste 
ge"andert werden k"onnen (unter Beachtung der Reihenfolge); ich habe
das aber nicht ausprobiert, lassen Sie es besser.

In der Anleitung zu Gemini (VENUS.DOC) steht, da"s die Icons
erst in Y-, dann in X-Richtung sortiert werden m"ussen. Das 
behauptet der Kuma-Resourceeditor auch, und daher stammt 
wahrscheinlich diese Angabe. Es ist aber falsch. Nicht nur,
da"s es unlogisch ist, die anderen Editoren behaupten das 
Gegenteil. Und ich auch. Richtig ist: beim Kuma-Resourceeditor mu"s
erst in Y-, dann in X-Richtung sortiert werden, weil der
Kuma-Editor 'falschrum' sortiert, bei allen anderen erst in
X-, dann in Y-Richtung. Genauer:

Das DR-RCS bietet 4 Sortieroptionen an: \\
\begin{center}
\newdimen\mu \mu=0.0007mm
\newdimen\xu \xu=372\mu
\newdimen\yu \yu=372\mu
\setlength{\unitlength}{372\mu}
\begin{picture}(264,83)
\put(6,82){\rule{254\xu}{1\yu}}
\put(14,76){\rule{58\xu}{2\yu}}
\put(86,76){\rule{26\xu}{2\yu}}
\put(126,76){\rule{50\xu}{2\yu}}
\put(190,76){\rule{50\xu}{2\yu}}
\put(225,74){\rule{4\xu}{1\yu}}
\put(98,73){\rule{2\xu}{2\yu}}
\put(138,73){\rule{2\xu}{2\yu}}
\put(163,73){\rule{2\xu}{2\yu}}
\put(202,73){\rule{2\xu}{2\yu}}
\put(224,73){\rule{6\xu}{1\yu}}
\put(97,71){\rule{3\xu}{2\yu}}
\put(137,71){\rule{3\xu}{2\yu}}
\put(162,71){\rule{3\xu}{2\yu}}
\put(201,71){\rule{3\xu}{2\yu}}
\put(228,71){\rule{2\xu}{2\yu}}
\put(224,71){\rule{2\xu}{2\yu}}
\put(161,69){\rule{4\xu}{2\yu}}
\put(227,69){\rule{2\xu}{2\yu}}
\put(160,67){\rule{2\xu}{2\yu}}
\put(226,67){\rule{2\xu}{2\yu}}
\put(163,67){\rule{2\xu}{2\yu}}
\put(98,65){\rule{2\xu}{6\yu}}
\put(138,65){\rule{2\xu}{6\yu}}
\put(160,65){\rule{6\xu}{2\yu}}
\put(202,65){\rule{2\xu}{6\yu}}
\put(225,65){\rule{2\xu}{2\yu}}
\put(96,63){\rule{6\xu}{2\yu}}
\put(136,63){\rule{6\xu}{2\yu}}
\put(163,63){\rule{2\xu}{2\yu}}
\put(200,63){\rule{6\xu}{2\yu}}
\put(224,63){\rule{6\xu}{2\yu}}
\put(41,58){\rule{4\xu}{1\yu}}
\put(97,58){\rule{4\xu}{1\yu}}
\put(137,58){\rule{4\xu}{1\yu}}
\put(26,57){\rule{2\xu}{2\yu}}
\put(40,57){\rule{6\xu}{1\yu}}
\put(56,57){\rule{6\xu}{2\yu}}
\put(96,57){\rule{6\xu}{1\yu}}
\put(136,57){\rule{6\xu}{1\yu}}
\put(160,57){\rule{6\xu}{2\yu}}
\put(200,57){\rule{6\xu}{2\yu}}
\put(227,57){\rule{2\xu}{2\yu}}
\put(25,55){\rule{3\xu}{2\yu}}
\put(44,55){\rule{2\xu}{2\yu}}
\put(59,55){\rule{2\xu}{2\yu}}
\put(100,55){\rule{2\xu}{2\yu}}
\put(140,55){\rule{2\xu}{2\yu}}
\put(160,55){\rule{2\xu}{2\yu}}
\put(203,55){\rule{2\xu}{2\yu}}
\put(226,55){\rule{3\xu}{2\yu}}
\put(40,55){\rule{2\xu}{2\yu}}
\put(96,55){\rule{2\xu}{2\yu}}
\put(136,55){\rule{2\xu}{2\yu}}
\put(160,54){\rule{5\xu}{1\yu}}
\put(43,53){\rule{2\xu}{2\yu}}
\put(58,53){\rule{2\xu}{2\yu}}
\put(99,53){\rule{2\xu}{2\yu}}
\put(139,53){\rule{2\xu}{2\yu}}
\put(160,53){\rule{6\xu}{1\yu}}
\put(202,53){\rule{2\xu}{2\yu}}
\put(225,53){\rule{4\xu}{2\yu}}
\put(42,51){\rule{2\xu}{2\yu}}
\put(59,51){\rule{2\xu}{2\yu}}
\put(98,51){\rule{2\xu}{2\yu}}
\put(138,51){\rule{2\xu}{2\yu}}
\put(203,51){\rule{2\xu}{2\yu}}
\put(224,51){\rule{2\xu}{2\yu}}
\put(227,51){\rule{2\xu}{2\yu}}
\put(26,49){\rule{2\xu}{6\yu}}
\put(41,49){\rule{2\xu}{2\yu}}
\put(56,49){\rule{2\xu}{2\yu}}
\put(97,49){\rule{2\xu}{2\yu}}
\put(137,49){\rule{2\xu}{2\yu}}
\put(160,49){\rule{2\xu}{1\yu}}
\put(200,49){\rule{2\xu}{2\yu}}
\put(224,49){\rule{6\xu}{2\yu}}
\put(60,49){\rule{2\xu}{2\yu}}
\put(164,49){\rule{2\xu}{4\yu}}
\put(204,49){\rule{2\xu}{2\yu}}
\put(56,48){\rule{6\xu}{1\yu}}
\put(160,48){\rule{6\xu}{1\yu}}
\put(200,48){\rule{6\xu}{1\yu}}
\put(24,47){\rule{6\xu}{2\yu}}
\put(40,47){\rule{6\xu}{2\yu}}
\put(57,47){\rule{4\xu}{1\yu}}
\put(96,47){\rule{6\xu}{2\yu}}
\put(136,47){\rule{6\xu}{2\yu}}
\put(161,47){\rule{4\xu}{1\yu}}
\put(201,47){\rule{4\xu}{1\yu}}
\put(227,47){\rule{2\xu}{2\yu}}
\put(162,42){\rule{3\xu}{1\yu}}
\put(226,42){\rule{3\xu}{1\yu}}
\put(96,41){\rule{6\xu}{2\yu}}
\put(136,41){\rule{6\xu}{2\yu}}
\put(161,41){\rule{4\xu}{1\yu}}
\put(200,41){\rule{6\xu}{2\yu}}
\put(225,41){\rule{4\xu}{1\yu}}
\put(160,40){\rule{3\xu}{1\yu}}
\put(224,40){\rule{3\xu}{1\yu}}
\put(99,39){\rule{2\xu}{2\yu}}
\put(139,39){\rule{2\xu}{2\yu}}
\put(200,39){\rule{2\xu}{2\yu}}
\put(160,38){\rule{2\xu}{2\yu}}
\put(200,38){\rule{5\xu}{1\yu}}
\put(224,38){\rule{2\xu}{2\yu}}
\put(98,37){\rule{2\xu}{2\yu}}
\put(138,37){\rule{2\xu}{2\yu}}
\put(160,37){\rule{5\xu}{1\yu}}
\put(200,37){\rule{6\xu}{1\yu}}
\put(224,37){\rule{5\xu}{1\yu}}
\put(160,36){\rule{6\xu}{1\yu}}
\put(224,36){\rule{6\xu}{1\yu}}
\put(99,35){\rule{2\xu}{2\yu}}
\put(139,35){\rule{2\xu}{2\yu}}
\put(96,33){\rule{2\xu}{2\yu}}
\put(136,33){\rule{2\xu}{2\yu}}
\put(160,33){\rule{2\xu}{3\yu}}
\put(200,33){\rule{2\xu}{1\yu}}
\put(224,33){\rule{2\xu}{3\yu}}
\put(100,33){\rule{2\xu}{2\yu}}
\put(140,33){\rule{2\xu}{2\yu}}
\put(164,33){\rule{2\xu}{3\yu}}
\put(204,33){\rule{2\xu}{4\yu}}
\put(228,33){\rule{2\xu}{3\yu}}
\put(96,32){\rule{6\xu}{1\yu}}
\put(136,32){\rule{6\xu}{1\yu}}
\put(160,32){\rule{6\xu}{1\yu}}
\put(200,32){\rule{6\xu}{1\yu}}
\put(224,32){\rule{6\xu}{1\yu}}
\put(97,31){\rule{4\xu}{1\yu}}
\put(137,31){\rule{4\xu}{1\yu}}
\put(161,31){\rule{4\xu}{1\yu}}
\put(201,31){\rule{4\xu}{1\yu}}
\put(225,31){\rule{4\xu}{1\yu}}
\put(14,30){\rule{2\xu}{46\yu}}
\put(86,30){\rule{2\xu}{46\yu}}
\put(126,30){\rule{2\xu}{46\yu}}
\put(190,30){\rule{2\xu}{46\yu}}
\put(70,30){\rule{2\xu}{46\yu}}
\put(110,30){\rule{2\xu}{46\yu}}
\put(174,30){\rule{2\xu}{46\yu}}
\put(238,30){\rule{2\xu}{46\yu}}
\put(14,28){\rule{58\xu}{2\yu}}
\put(86,28){\rule{26\xu}{2\yu}}
\put(126,28){\rule{50\xu}{2\yu}}
\put(190,28){\rule{50\xu}{2\yu}}
\put(97,20){\rule{4\xu}{1\yu}}
\put(213,20){\rule{2\xu}{2\yu}}
\put(42,19){\rule{2\xu}{2\yu}}
\put(96,19){\rule{6\xu}{1\yu}}
\put(148,19){\rule{6\xu}{2\yu}}
\put(212,18){\rule{3\xu}{2\yu}}
\put(41,17){\rule{3\xu}{2\yu}}
\put(100,17){\rule{2\xu}{2\yu}}
\put(151,17){\rule{2\xu}{2\yu}}
\put(96,17){\rule{2\xu}{2\yu}}
\put(211,16){\rule{4\xu}{2\yu}}
\put(99,15){\rule{2\xu}{2\yu}}
\put(150,15){\rule{2\xu}{2\yu}}
\put(210,14){\rule{2\xu}{2\yu}}
\put(213,14){\rule{2\xu}{2\yu}}
\put(98,13){\rule{2\xu}{2\yu}}
\put(151,13){\rule{2\xu}{2\yu}}
\put(210,12){\rule{6\xu}{2\yu}}
\put(42,11){\rule{2\xu}{6\yu}}
\put(97,11){\rule{2\xu}{2\yu}}
\put(148,11){\rule{2\xu}{2\yu}}
\put(152,11){\rule{2\xu}{2\yu}}
\put(148,10){\rule{6\xu}{1\yu}}
\put(213,10){\rule{2\xu}{2\yu}}
\put(220,10){\rule{2\xu}{4\yu}}
\put(40,9){\rule{6\xu}{2\yu}}
\put(50,9){\rule{2\xu}{4\yu}}
\put(96,9){\rule{6\xu}{2\yu}}
\put(106,9){\rule{2\xu}{4\yu}}
\put(149,9){\rule{4\xu}{1\yu}}
\put(158,9){\rule{2\xu}{4\yu}}
\put(6,1){\rule{1\xu}{81\yu}}
\put(6,0){\rule{254\xu}{1\yu}}
\put(259,1){\rule{1\xu}{81\yu}}
\end{picture}
{\\ Die Sortieroptionen beim DR-RCS}
\end{center}

Sie k"onnen hierbei erst die 1., dann die 2. Option w"ahlen, oder Sie
w"ahlen nur die 4. Option. Interface bietet nur die 3. und 4. Option;
Sie w"ahlen die letztere. Beim WERCS mu"s erst 'Left to right' 
(unter 'First' einstellen), dann 'Top to bottom' (unter 'Second' 
einstellen) sortiert werden. 

Nach allen Ver"anderungen, die Sie in einem Baum vorgenommen haben,
m"ussen Sie vor dem Speichern den Baum neu sortieren. Vergessen Sie
das bitte nicht, sonst kommen die angemeldeten Regeln 
durcheinander~!! Achten Sie auch darauf, da"s die Anzahl der gro"sen
und der kleinen Dateiicons gleich ist (Gemini l"adt nur die
kleinere Anzahl: wenn Sie in einem Baum 10 Icons haben, im anderen
17, werden nur 10 geladen).

Im folgenden wird am Beispiel DR-RCS 2.0 beschrieben, wie man ein 
gro"ses Dateiicon in die Datei GEMINIIC.RSC hinzuf"ugt:

\begin{enumerate}
 \item Legen Sie sich unbedingt vor der "Anderung eine 
       Sicherheitskopie der GEMINIIC.RSC-Datei an; f"ur den Fall, 
       da"s etwas schief ging, k"onnen Sie diese benutzen, falls 
       Gemini mit der ver"anderten Datei nicht l"auft.
 \item Speichern Sie die Icons, die Sie einf"ugen wollen, mit
       IconEdi als RSC-Dateien ab (unter 'Ausgabe' 'RSC-Datei'
       einstellen).
 \item Starten Sie das DR-RCS.
 \item Laden Sie die Datei GEMINIIC.RSC (mit 'Open').
       Eventuelle Fehlermeldungen ignorieren Sie (auf 'OK' klicken).
 \item Erscheinen drei Fragezeichen-Icons (mit 'TREE1' bis
       'TREE3' darunter; diese stellen die Objektb"aume dar; das 
       Fragezeichen bedeutet, da"s das DR-RCS nicht wei"s, um was f"ur 
       einen Typ von Objektbaum es sich handelt), doppelklicken Sie 
       eins nach dem anderen an und "andern den Baumtyp in 'Dialog'.
 \item Laden Sie (mit 'Merge') die von IconEdi erstellte RSC-Datei
       hinzu (Fehlermeldungen mit 'OK' "ubergehen) 
       und "andern Sie auch deren Typ in Dialog.
 \item "Offnen Sie den Dialogbaum mit dem Icon (den vierten) durch 
       Doppelklick.
 \item Klicken Sie das Icon darin an und klicken Sie auf das 
       Klemmbrett des RCS. Das Icon verschwindet aus dem Dialog
       und befindet sich auf dem Klemmbrett.
 \item Schlie"sen Sie den Dialogbaum (mit 'Close') und l"oschen Sie
       ihn (auf den M"ulleimer schieben).
 \item "Offnen Sie den zweiten Objektbaum durch Doppelklick.
 \item Ziehen Sie das Icon vom Klemmbrett und plazieren es in der
       letzten Zeile der Icons an die letzte Position. Enth"alt die 
       letzte Zeile schon sieben Icons, plazieren Sie es an die
       erste Stelle einer neuen Zeile (Hinweis: Sie k"onnen in der 
       Dialogbox hoch- und runterbl"attern "uber den rechten 
       Rollbalken des Fensters. Wollen Sie die Dialogbox vergr"o"sern,
       klicken Sie ihre rechte untere Ecke an. Ein Gummiband 
       erscheint; ziehen Sie dies nach unten und lassen los, wird
       die Dialogbox vergr"o"sert. Eventuell m"ussen Sie jetzt noch mit
       dem Rollbalken herunterbl"attern, um in die (neue) letzte 
       Zeile zu gelangen).
 \item Klicken Sie au"serhalb aller Icons die Dialogbox an (=das 
       Arbeitsfenster). Ein schwarzes Rechteck erscheint unten 
       rechts.
 \item W"ahlen Sie im Men"u 'Optionen' 'Sort children' (oder so) an.
       W"ahlen Sie die 4. Option (siehe oben) und klicken 'OK'.
 \item Schlie"sen Sie den Dialogbaum (durch 'Close').
 \item "Offnen Sie den 3. Dialogbaum (den mit den kleinen Icons).
 \item Klicken Sie ein Icon an und kopieren es aufs Klemmbrett
       ('Copy' im Men"u).
 \item Ziehen Sie das Icon vom Klemmbrett und plazieren es an die
       letzte Position (wie oben beschrieben).
 \item Sortieren und schlie"sen Sie den Baum.
 \item Speichern Sie die Datei.
\end{enumerate}

Wenn man mehrere Icons einf"ugen will, wiederholt man die Prozedur. 

Sie m"u"sten mit Hilfe
der Anleitungen anderer Resourceeditoren in der Lage sein, 
die Beschreibung auf andere Editoren zu "ubertragen. Mit dem
DR-RCS 1.x ist die Bearbeitung der GEMINIIC.RSC nicht m"oglich,
da dieses RCS nur Dateien einer L"ange von maximal 32000 Bytes 
bearbeiten kann.

Das Einf"ugen kleiner Icons erfolgt entsprechend. Wenn Sie Icons 
austauschen wollen, so l"oschen Sie einfach eines der 
vorhandenen und f"ugen an seine Stelle ein neues ein. 
Sortieren nicht vergessen !

Icons l"oschen ist nicht so einfach, weil dann die angemeldeten
Regeln durcheinanderkommen (Gemini st"urzt dann eventuell sogar beim 
Laden ab). Dies ist auch verst"andlich: angenommen, Sie
haben eine Regel f"ur das 27. Icon angemeldet, l"oschen das 
12. und sortieren. Das 27. ist nun das 26. und das 28. das 27.
Existierte kein 28., st"urzt Gemini beim Laden ab. Nicht sortieren
bringt auch nix, weil die Resourceeditoren dann beim Speichern
die L"ucke f"ullen, indem sie die folgenden Icons vorziehen. 
Vor dem L"oschen m"ussen daher die
Regeln abgemeldet werden; wenn Sie Laufwerksicons abmelden wollen,
m"ussen auch deren Regeln abgemeldet werden. Da dies viel Arbeit ist,
werden Sie es nicht allzu h"aufig machen wollen... daher: l"oschen
Sie einmal alle Regeln, l"oschen Sie alle "uberfl"ussigen Icons
(Schredder, Klemmbrett voll/leer, Papierkorb voll/leer d"urfen Sie
nicht l"oschen; Sie k"onnen sie aber ersetzen; beachten Sie, da"s
die Icons f"ur Klemmbrett voll/leer bzw. Papierkorb voll/leer die
gleiche Gr"o"se haben sollten; zumindest ein Laufwerk sollten Sie 
besser auch lassen), setzen Sie die Regeln wieder und f"ugen in 
Zukunft nur noch hinzu.

Noch etwas zu Gemini-Icons: bei Gemini wird der Wirkbereich des
Icons "uber die Maske festgelegt~! Der Wirkbereich eines Icons
ist normalerweise ein Rechteck; klickt man in dieses, f"uhlt sich
das Icon angesprochen, es wird selektiert. Bei Gemini wird dieser 
Bereich wie gesagt "uber die Maske festgelegt; das hat den netten
Effekt, da"s man z.B. ein kreisf"ormiges Icon erstellen kann, und
dieses wird nur selektiert, wenn man auch wirklich in den Kreis
klickt~! Dies geht beim GEM nicht, da dort der Wirkbereich ein 
Rechteck ist; dort kann man bei einem solchen Icon auch au"serhalb
des Kreises klicken und das Icon selektieren.

Die Gemini-Methode ist an sich sch"oner. Nur hat sie auch einen 
Nachteil: man mu"s darauf achten, da"s die Maske keine L"ocher hat.
Sonst reagiert das Icon nicht, wenn man auf eines der L"ocher klickt.
Auch schr"ankt es die M"oglichkeiten f"ur Icons auf Objekte mit
gr"o"seren gef"ullten Fl"achen ein. Das ist alles nicht schlimm, viel
schlimmer ist, da"s es in der Gemini-Dokumentation bisher nicht 
erw"ahnt wird (ich brauchte etliche Stunden, bis ich es raus hatte... 
auf die Idee mu"s man auch erst mal kommen). In den meisten F"allen
wird die Option 'Gemini-Maske' von IconEdi Ihnen eine (nicht nur
f"ur Gemini) optimale Maske erstellen. Trotzdem: achten Sie bei
Gemini-Icons auf L"ocher in der Maske~!

\section{Icons f"ur den neuen Atari-Desktop} 
\index{Icons f"ur den neuen Atari-Desktop}
Von IconEdi erstellte Icons k"onnen selbstverst"andlich auch
in die Datei DESKICON.RSC des neuen Atari-Desktop eingebaut werden.
Die Vorgehensweise "ahnelt der oben f"ur Gemini beschriebenen.
Es ist aber nur ein Baum vorhanden und der Wirkbereich wird
in der "ublichen Weise festgelegt. Die Sortierung erfolgt
in der gleichen Weise wie bei Gemini (und die Folgen bei falscher
Sortierung d"urften auch die gleichen sein, daher: vorher eine
Sicherheitskopie der Datei anlegen~!).

\section{Icons ohne Text und Buchstaben} 
\index{Icons ohne Text und Buchstaben}
Icons haben normalerweise einen Text und optional einen Buchstaben.
Mit einem Trick kann man daf"ur sorgen, da"s diese nicht
sichtbar sind: man verschiebt Text und Buchstabe in IconEdi 
unter 'Icontext' so,
da"s sie (bzw. ihr Anfang) {\sl im} Icon liegen, und setzt 
sie auf ein Leerzeichen~! So kann man die 
das (zumindest mein) Sch"onheitsgef"uhl st"orende Textfahne sp"ater
(z.B. unter Gemini) unsichtbar machen, indem man den Text auf
ein Leerzeichen setzt.

Diesen Trick verdanke ich einem netten Menschen im Fido-Netz, dessen
Namen ich vergessen habe (Schande "uber mein Haupt). Er m"oge mir mal 
schreiben...

Der Trick funktioniert nicht f"ur Icons des neuen Atari-Desktop,
da der zust"andige Programmierer es nicht f"ur n"otig hielt, die
entsprechenden Angaben aus dem Iconblock zu benutzen. Der Text
wird einfach unter das Icon gesetzt. Kein Kommentar dazu...


%\input chapter4.tex
\chapter{Anpassung an pers"onliche Bed"urfnisse}
\index{Anpassung an pers"onliche Bed"urfnisse}
IconEdi kann in vielerlei Hinsicht an pers"onliche Bed"urfnisse
angepa"st werden. Dies kann Ihrem Geschmack dienen oder schlicht 
n"otig sein. Lesen Sie sich vor jeglichen "Anderungen bitte dieses
Kapitel aufmerksam und komplett durch, da sich verschiedene 
Anpassungen gegenseitig beeinflussen k"onnen und bewahren Sie immer 
eine nichtge"anderte Sicherheitskopie von IconEdi f"ur den Notfall 
auf~! Jegliche "Anderungen an IconEdi, die "uber die in diesem 
Kapitel beschriebenen hinausgehen, sind verboten.

\section{Anpassung an verschiedene TOS-Versionen}
\index{Anpassung an verschiedene TOS-Versionen}
TOS-Versionen verschiedener L"ander liefern teilweise
verschiedene Tastencodes f"ur den gleichen Buchstaben.
IconEdi enth"alt eine Tabelle der (deutschen) Codes.
Da zum einen anderssprachige Versionen von IconEdi geplant
sind und zum anderen es sehr gut m"oglich ist, da"s jemand
z.B. in Deutschland oder der Schweiz (hallo Patrick) mit einem 
anderssprachigen TOS arbeitet, mu"s eventuell eine Anpassung 
an dieses TOS erfolgen.

Diese Anpassung nimmt das Programm MAKETAB.PRG vor (im Ordner
ZUBEHOER zu finden). Das Programm ist ein 'quick and dirty'-Hack:
es ist spartanisch in der Bedienung, l"auft mit hoher 
Wahrscheinlichkeit nicht auf allen Systemkonfigurationen und ist
langsam. Aber: es erf"ullt seinen Zweck, eine einmalige Anpassung
vorzunehmen. Das Programm wird in Zukunft so weit wie n"otig
erweitert werden.

Die Anpassung erfolgt so:
\begin{enumerate}
 \item Kopieren Sie ICONEDI.APP, ICONEDI.RSC und MAKETAB.PRG in 
       ein Verzeichnis. ICONEDI.APP darf nicht gepackt sein 
       (siehe unten).
 \item Starten Sie MAKETAB.PRG und folgen Sie den Anweisungen.
 \item Starten Sie ICONEDI.APP und probieren Sie die Tasten durch.
       Stimmt etwas nicht, starten Sie wieder bei 2.
\end{enumerate}

\section{"Anderung der Dateinamen und Pfade} 
\index{"Anderung der Dateinamenund Pfade}
Der Name der Programmdatei ICONEDI.APP kann ver"andert werden. F"ur 
die Anpassung an verschiedene TOS-Versionen mu"s der Name ICONEDI.APP
lauten.

Der Name der Resourcedatei ICONEDI.RSC darf nicht ver"andert werden.

Die Programmdatei k"onnen Sie hinkopieren, wo Sie wollen. Die
Resourcedatei k"onnen Sie ebenfalls hinkopieren, wo Sie wollen.
Damit IconEdi die Resourcedatei aber dann finden kann, wenn Sie
sie in einen anderen Ordner oder auf ein anderes Laufwerk kopieren,
mu"s in dem Pfad, wo die Programmdatei steht, eine Datei IEDIPATH.INF
(IconEdi-Pfad-Info (Path, engl.: Pfad)) angelegt werden. In dieser
Datei mu"s der Pfad, wo die Resourcedatei liegt, eingetragen werden.
Benutzen Sie zur Erstellung der Datei einen Texteditor, z.B. Tempus.
Die Datei mu"s genau eine Zeile enthalten, und der Pfad mu"s 
komplett sein, also inklusive Laufwerksbuchstaben und abschlie"sendem
$\backslash$. Beispiel: 
'E:$\backslash$GFABASIC$\backslash$RCS$\backslash$' bedeutet,
% E:\GFABASIC\RCS\
da"s die Resourcedatei im Ordner 'RCS', der im Ordner 'GFABASIC'
auf Laufwerk 'E:' liegt, zu finden ist. Am Ende der Zeile m"ussen Sie
im Texteditor die Eingabetaste dr"ucken.

Es gibt andere, elegantere Methoden, die Resourcedatei zu finden, 
aber keine dieser ist unfehlbar, und die meisten sind erheblich 
komplizierter (sowohl f"ur den Anwender wie f"ur mich). 

Ein Vorteil des Verlegens der Resourcedatei sei noch erw"ahnt: Sie
k"onnen die Programmdatei auf ein gegen Schreiben gesichertes 
Laufwerk kopieren und die Resourcedatei auf ein nicht gesichertes,
so jederzeit "Anderungen an der Resourcedatei vornehmen und die
Programmdatei ist trotzdem gegen Programmviren gesichert. Zu den
"Anderungen an der Resourcedatei, die Sie vornehmen d"urfen, komme ich
gleich.

\section{Einstellungen in IconEdi}
\index{Einstellungen in IconEdi}
Eine ganze Reihe von Einstellungen k"onnen in IconEdi vorgenommen 
und mit 'Arbeit sichern' gespeichert werden. Dabei wird eine Datei
ICONEDI.INF erstellt. Diese wird im gleichen Verzeichnis, aus dem
die Resourcedatei geladen wurde, gespeichert und von dort wieder 
geladen. 

\section{"Anderungen an der Resourcedatei}
\index{"Anderungen an der Resourcedatei} 
\index{Anpassung der Tastaturbelegung}
Durch "Anderungen an der Resourcedatei zu IconEdi (ICONED.RSC)
k"onnen Sie bestimmen, welche Tasten IconEdi f"ur die Funktionen
in den Men"us und in den Dialogen benutzt. Warum ich dies
eingebaut habe, hat zwei Gr"unde:
\begin{enumerate}
 \item die Belegungen sind f"ur mich so einfacher zu "andern
 \item die Belegungen k"onnen von Ihnen ge"andert werden
\end{enumerate}

Der 2. Grund spielte f"ur mich dabei ehrlich gesagt keine so gro"se
Rolle. Aber es ist nun m"oglich, also kann ich Ihnen auch sagen,
wie. Bevor Sie sich aber auf die Resourcedatei st"urzen, bedenken
Sie folgendes:
\begin{enumerate}
 \item Es ist leicht m"oglich, durch "Anderungen an der Resourcedatei 
       IconEdi zum Absturz zu bringen. Das mu"s nicht sofort beim
       Programmstart geschehen, es kann passieren, nachdem Sie
       gerade 2 Stunden an einem Icon gearbeitet haben (und 
       nat"urlich nicht abgespeichert).
 \item Wenn Sie den gleichen Tastaturk"urzel mehrfach vergeben, kann
       es passieren, da"s bei der Bet"atigung der Taste(n) mehrere
       Funktionen in direkter Folge ausgef"uhrt werden --- und 
       dadurch die erste Funktion zunichte gemacht oder Daten 
       gel"oscht werden.
 \item S"amtliche Hilfe meinerseits zu Problemen bei der "Anderung
       der Tastaturbelegung geschieht nach Gutd"unken. Ich 
       garantiere in dieser Hinsicht f"ur nichts.
 \item Jede neue Version hat wieder die alte Belegung, die alte
       Resourcedatei ist mit ziemlicher Sicherheit nicht weiter
       verwendbar. Sie m"ussen also die "Anderungen erneut vornehmen.
 \item IconEdi macht keinerlei Plausibilit"atspr"ufungen bez"uglich
       der Belegung. S"amtliche denkbaren unsinnigen "Anderungen
       sind m"oglich, mit einer ganzen Reihe denkbarer fataler 
       Folgen.
\end{enumerate}

Daher: "Anderungen sollten, wenn "uberhaupt, nur nach reiflicher
"Uberlegung und in Ma"sen vorgenommen werden. Wenn Sie keinerlei
Erfahrung im Umgang mit Resourcedateien haben und die folgenden
Erkl"arungen nicht verstehen: lassen Sie es. 

Die Resourcedatei von IconEdi beinhaltet einige Objektb"aume. Der 
erste ist der Men"ubaum, die anderen sind alle Dialogb"aume.

Die "Anderungen k"onnen mit jedem Resourceeditor vorgenommen werden.
Beachten Sie dabei:
\begin{itemize}
 \item Machen Sie sich vorher eine Sicherheitskopie von ICONEDI.RSC~!
 \item "Andern Sie nie irgendwelche Reihenfolgen der Objekte~!
 \item Sortieren Sie niemals~!
 \item Probieren Sie die ge"anderte Resourcedatei ausf"uhrlich aus,
       bevor Sie sich darauf verlassen, da"s IconEdi damit korrekt 
       l"auft~!
\end{itemize}

\subsection{"Anderung der Tastaturbelegung f"ur die Men"ufunktionen}
\index{"Anderung der Tastaturbelegung f"ur die Men"ufunktionen}
In den Men"us steht hinter dem Funktionsnamen, durch mindestens
ein Leerzeichen getrennt, das Tastaturk"urzel der Funktion. Dieses
kann ge"andert werden. M"oglich sind:
\begin{itemize}
 \item Buchstaben von A-Z, Zahlen von 0-9
 \item Kombinationen der Control-Taste und eines Buchstaben. Das
       Zeichen {\control} symbolisiert dabei die Control-Taste.
       Beispiel: {\control}L
 \item Kombinationen der Alternate-Taste und eines Buchstaben. Das
       Zeichen {\alternate} symbolisiert dabei die Alternate-Taste.
       Beispiel: {\alternate}M
 \item die Tasten 'runde Klammer auf' [(], 'runde Klammer zu' [)],
       'Schr"agstrich nach rechts (Division)' [/], 'Mal-Zeichen' [*],
       'Minus-Zeichen' [-] und 'Plus-Zeichen' [+] auf dem 
       Ziffernblock
 \item folgende Sondertasten: Insert, Delete, Tab, Home, Clr, Help,
       Undo, Leertaste. Dazu mu"s im Klartext eingetragen werden:
       Insert, Delete, Tab, Home, Clr, Help, Undo, Space.
 \item die Funktionstasten F1-F10. Dazu mu"s im Klartext eingetragen
       werden: F1, F2, F3, F4, F5, F6, F7, F8, F9, F10
\end{itemize}

Beispiele f"ur fast alle diese K"urzel finden Sie voreingestellt in 
den Men"us von IconEdi. Die Funktionstasten werden von IconEdi in der
Voreinstellung nicht benutzt, da ich mir wesentlich leichter 
'{\control}L bedeutet Laden' als 'F7 bedeutet Laden' merken kann.

\subsection{"Anderung der Tastaturbelegung in den Dialogen}
\index{"Anderung der Tastaturbelegung in den Dialogen}
In den Dialogen sind die Funktionen "uber die Tastatur mittels
der {\alternate}-Taste zusammen mit einem der Buchstaben A...Z
aufrufbar. Der Buchstabe wird einfach durch eine vorangestellte
eckige Klammer auf ([) gekennzeichnet. Zum Beispiel wird ein Knopf
mit dem Text 'Set[zen' mit {\alternate}Z bedient. Die eckige
Klammer wird bei der Dialogdarstellung nicht ausgegeben, daf"ur
wird der dahinter stehende Buchstabe unterstrichen.

Beachten Sie bitte, da"s der gew"ahlte Buchstabe nur A..Z sein darf,
kein Umlaut oder ein anderes Sonderzeichen und da"s pro Funktion nur
eine eckige Klammer gesetzt werden darf.

\begin{appendix}

%\input append_b.tex
\setcounter{chapter}{1}
\chapter{Begriffserkl"arungen}  \index{Begriffserkl"arungen}
\index{Glossar} \index{Anhang B} \index{B-Anhang}
% B

{\bf Alternative Benutzeroberfl"achen} \\
\index{Alternative Benutzeroberfl"achen}
\index{Benutzeroberfl"achen}
Alternative Benutzeroberfl"achen wie Gemini, Neodesk usw. kommen 
auf Atari-ST/TT-Rechnern immer mehr in Mode. Zu recht, denn
Sie ersetzen den Standard-Desktop durch einen eigenen mit 
teilweise erheblich erweiterten M"oglichkeiten und h"oherem Komfort, 
wie z.B. der M"oglichkeit, Icons als Symbol von Programmen auf den 
Desktop zu legen, so da"s beim Anklicken des Icons das zugeh"orige 
Programm gestartet wird. 

{\bf Desktop} \\ \index{Desktop}
Der Hintergrund des Arbeitsbildschirms; auf dem 
Standard-Atari-Desktop finden Sie Laufwerks-Icons und den 
M"ulleimer. 'Desk' ist englisch und bedeutet 'Schreibtisch'.
'Desktop' wird demnach oft als 'Schreibtischoberfl"ache' 
"ubersetzt. Dem liegt die Vorstellung des Bildschirms als 
Bild eines (virtuellen) Schreibtischs zugrunde.

{\bf Extension} \\
\index{Extension} \index{Dateierweiterung}
Dateierweiterung. Beispiele: PRG, TXT, APP, TOS, IMG, ICO usw.
Dient der Unterscheidung von Dateitypen, z.B. in
ausf"uhrbare Programme (PRG, APP, TOS usw.) und Daten 
(TXT, IMG, ICO).

{\bf Gemini} \\ \index{Gemini}
Eine alternative Benutzeroberfl"ache, die aus eigentlich zwei 
unabh"angigen Teilen besteht: einer grafischen Oberfl"ache und
einem sich an UNIX-Befehlen orientierenden 
Kommandozeileninterpreter. So werden die Vorteile beider 
Bedienweisen vereint. Es ist schon viel "uber Gemini geschrieben
worden, Gemini ist in diversen ST-Fachzeitschriften getestet 
worden, daher nur soviel: Sie sollten sich Gemini 
auf jeden Gemini einmal ansehen~!!!!! Gemini ist ein
Sharewareprogramm, kostet 50 DM (und ist jeden Pfennig wert) und 
kann zur Ansicht gegen Einsendung einer formatierten, doppelseitigen
3.5-Zoll-Diskette und R"uckumschlag bezogen werden bei den beiden 
Autoren: Gereon Steffens, Elsterweg 8, D-5000 K"oln 90 und
Stefan Eissing, Dorfbauerschaft 7, D-4419 Laer-Holthausen.

{\bf Images} \\ \index{Images}
Images sind Icons ohne Maske, ohne Text und Buchstaben: Bilder,
die daf"ur gedacht sind, z.B. in Dialogboxen auf einem festgelegten
Hintergrund zu erscheinen, als Illustration und nicht als
anw"ahlbares, verschiebbares Bedienelement wie Icons.

{\bf Maxon-Sonderdisketten} \\ \index{Maxon-Sonderdisketten}
Die Firma Maxon bietet mit ihren Sonderdisketten Programme zu einem
sehr niedrigen Preis an. Eine Auswahl dieser Programme wird in 
jeder Ausgabe der Zeitschrift ST-Computer vorgestellt. Ein 
Programm aus dieser Reihe, das ich uneingeschr"ankt empfehlen kann,
ist Special Paint, das f"ur 20 DM mehr bietet als manches wesentlich
teurere Zeichenprogramm. Professionelle Anspr"uche kann man aber
an die Programme nicht immer stellen; das w"are bei den Preisen aber
auch etwas viel verlangt.

{\bf Modus} \\ \index{Modus}
Sie kennen Dialogboxen ? Dialogboxen sind in GEM modal: wenn man 
eine Funktion ausf"uhrt, die eine Dialogbox auf den Bildschirm 
bringt, befindet man sich in einem Modus. Die Men"uleiste ist 
nicht mehr anw"ahlbar, Tasten haben andere Bedeutungen, die
Zahl der ausf"uhrbaren Funktionen ist beschr"ankt, um wieder
an die Men"uleiste zu kommen, mu"s man erst 'OK' oder 'Abbruch' 
anklicken. Ein Modus ist ein Abschnitt in einem Programm, den man
wieder ausdr"ucklich verlassen mu"s. 

Modi sollten, soweit m"oglich, in einem Programm vermieden werden,
weil sie den Benutzer verwirren k"onnen und von ihm verlangen,
sich umzustellen, umzudenken. Das 'soweit m"oglich' deutet es schon
an: es ist oft gar nicht m"oglich, auf Modi zu verzichten. Sei es,
da"s die Anzahl der Funktionen in einem Teil sonst zu gro"s wird 
(was den Benutzer auch verwirren kann), sei es, da"s Funktionen
sich gegenseitig beeinflussen k"onnen und daher sauber getrennt 
werden m"ussen. 

{\bf MultiGEM} \\ \index{MultiGEM}
MultiGEM ist ein Programm, da"s auf Atari-ST/TT-Rechnern eine Art
Multitasking erm"oglicht: mehrere Programme befinden sich 
gleichzeitig im Rechner und zwischen ihnen kann man einfach hin-
und herschalten, indem man ihre Fenster anklickt. Es ist sogar
m"oglich, da"s ein Programm eine Grafik malt, w"ahrend man in einem
anderen Programm einen Text bearbeitet. Leider sind aber sehr viele
Programme so unsauber programmiert, da"s sie sich gegenseitig 
behindern, den Bildschirm vollschmieren usw. Auch ist es nicht 
m"oglich, z.B. im Hintergrund zu formatieren. Von Atari selber wird
ein Multitasking-TOS erwartet, das hoffentlich diese 
Einschr"ankungen nicht mehr aufweist. Programmierer sollten sich
aber schon heute darauf einstellen: z.B. sind eigene Desktops --- 
prinzipbedingt --- in einer Multitasking-Umgebung l"astig und
unn"otig. Wer heute ein Programm entwirft, sollte z.B. daran denken.

{\bf ST-Computer} \\ \index{ST-Computer}
Eine deutsche Fachzeitschrift f"ur Atari-ST/TT-Rechner aus dem
Heimverlag.

{\bf \TeX} \\ \index{\TeX}
{\TeX} ist ein Computer-Textsatzsystem von Donald E. Knuth.

{\TeX} erlaubt
kein WYSIWYG (what you see is what you get), stellt nicht 
unerhebliche Hardwareanforderungen und st"o"st z.B. bei der
Bildeinbindung an Grenzen. Mit {\TeX} erstellte Texte sind
voller Formatierbefehle und daher ohne {\TeX} schlecht zu lesen.

Warum ich trotzdem {\TeX} f"ur dieses Handbuch benutzt habe,
hat folgende Gr"unde:
\begin{itemize}
 \item {\TeX} erlaubt mir, dem Schriftsatzlaien, Druck und Satz in
  professioneller Qualit"at
 \item {\TeX} ist Freeware, viele {\TeX}-Implementationen sind umsonst
  oder sehr preiswert erh"altlich: daher ist {\TeX} sehr verbreitet
  --- ganz im Gegensatz zu Programmen, die vergleichbare Ergebnisse
  liefern k"onnten, deren Besitz ich aber erst recht nicht bei jedem 
  vorraussetzen kann
\end{itemize}


%\input append_f.tex
\setcounter{chapter}{5}
\chapter{Fehlermeldungen und Fragen zu IconEdi}
\index{Anhang F} \index{F-Anhang} 
% F

\section{Fragen zu IconEdi}
\index{Fragen zu IconEdi}

Vergewissern Sie sich bitte vor Nachfragen,
ob Ihre Frage nicht durch dieses Handbuch beantwortet 
wird.

Nichtk"aufern werden nur schriftliche Fragen beantwortet.

Alle schriftlichen Fragen werden nur beantwortet, wenn 
ein ausreichend frankierter Briefumschlag f"ur das 
Antwortschreiben beiliegt.

Meine Adresse ist: Stefan M"unch, Borbergstr. 38, D-47 Hamm 1.

K"aufer k"onnen zus"atzlich telefonische Anfragen stellen,
aber nur Samstags von 15-17 Uhr unter der Rufnummer
049-02381-23324.

Au"serdem besteht die M"oglichkeit, mich "uber die Mailbox-Netze
Fido und Maus zu erreichen, und zwar im Fidonetz unter den Adressen
2:245/54.22 und 2:245/53.2 und im Mausnetz in der Maus Unna (UN).

\section{Fehlermeldungen}
\index{Fehlermeldungen} 

Sollten Sie Fehler in IconEdi feststellen, die im n"achsten
Abschnitt nicht erw"ahnt sind, k"onnen Sie mich "uber diese 
informieren. Schicken Sie mir dazu eine Fehlerbeschreibung an 
obige Adresse. Wenn Sie eine Antwort m"ochten, legen Sie auf jeden
Fall einen ausreichend frankierten R"uckumschlag bei.

Fehlerbeschreibungen sollten so genau und konkret wie m"oglich sein.
Mit einer Meldung 'IconEdi st"urzt bei mir manchmal ab...' kann ich 
gar nichts anfangen. 

Geben Sie bei Fehlermeldungen Ihre genaue
Systemkonfiguration an: welcher Rechner ? Wieviel Speicher ? 
Eventuelle Erweiterungen ? Geladene Programme, als der Fehler 
auftrat (Autoordner-Programme und Accessories) ? Haben Sie
"Anderungen an IconEdi vorgenommen ? Welche ?

Versuchen Sie, den Fehler einzukreisen. Stellen Sie alle Autoordner-
Programme und Accessories ab. Tritt der Fehler noch auf ? Stellen
Sie eins der Programme nach dem anderen an und probieren es erneut.
So k"onnen Sie feststellen, mit welchem Programm es Probleme gibt.

Ist der Fehler reproduzierbar ? Welche Arbeitsschritte m"ussen getan
werden, damit er auftritt ? 

Je genauer eine Fehlerbeschreibung ist, um so schneller kann ich 
den Fehler beseitigen.

\section{Bekannte Fehler} \index{Bekannte Fehler}
Folgende Fehler in IconEdi sind bekannt und werden sobald wie
m"oglich beseitigt:
\begin{itemize}
 \item IconEdi l"auft noch nicht fehlerfrei in allen Aufl"osungen.
       Er wird in absehbarer Zeit in allen Aufl"osungen mit 
       mindestens 640 Punkten in der Breite und 200 Punkten in der
       H"ohe laufen. Kleinere Aufl"osungen werden nicht unterst"utzt 
       werden. 
 \item IconEdi l"auft noch nicht im FastRAM des TT. Lassen Sie die
       Finger von den Flags, IconEdi wird so ausgeliefert, wie
       optimal m"oglich.
 \item Farbe wird noch nicht unterst"utzt.
\end{itemize}

Ferner verwendet IconEdi 'negative' Offsets f"ur die Funktion
Menu-Event, um die rechte Maustaste f"ur Klicks in das Fenster
benutzen zu k"onnen. Diese Offsets funktionieren in allen bekannten
TOS-Versionen und bei Atari wird dar"uber nachgedacht, sie als 
offiziell zu dokumentieren. Sollte dieses Feature von Atari 
abgelehnt werden oder in k"unftigen TOS-Versionen nicht 
funktionieren, wird IconEdi dies ber"ucksichtigen und auf andere
M"oglichkeiten (die jedoch mit einem Verlust an Bedienkomfort
verbunden w"aren) umgestellt werden.


%\input append_l.tex
\setcounter{chapter}{11}
\chapter{Lieferumfang} \index{Lieferumfang}
\index{Anhang L} \index{L-Anhang}
% L

Folgende Dateien geh"oren zu IconEdi (und IconEdi
darf nur komplett mit diesen weitergegeben werden):
\index{Weitergabe von IconEdi}
\begin{itemize}
 \item Ordner ICONEDI mit folgenden Dateien:
  \begin{itemize} 
   \item Ordner ANLEITUN.G mit folgenden Dateien:
    \begin{itemize}
     \item ICONEDI.TEX (die aktuelle Anleitung)
     \item eventuell Dateien NEW1.TXT, NEW2.TXT usw., die 
           "Anderungen an der Anleitung beschreiben
    \end{itemize}
   \item Ordner ICONS mit einigen Beispiel-Icons
   \item Ordner ZUBEHOER mit folgenden Dateien:
    \begin{itemize}
     \item Ordner DEMO mit folgenden Dateien:
      \begin{itemize}
       \item DEMO.LST (Demoprogramm als Listing)
       \item DEMOICON.ICO  (ein Demoicon)
       \item DEMOIMAG.ICO  (ein Demoimage)
      \end{itemize}
     \item MAKETAB.PRG (siehe Kapitel 4)
    \end{itemize}
   \item ICONEDI.APP (das Programm)
   \item ICONEDI.RSC (das Resourcefile dazu)
   \item LIESMICH.TXT (wie kommt man an die Anleitung ?)
  \end{itemize}
\end{itemize}

\index{Fehlende Dateien}
\index{Bezugsquellen}
Sollten Ihnen Dateien fehlen, so k"onnen Sie das komplette
Paket von mir gegen Einsendung einer doppelseitigen
3.5-Zoll-Diskette (formatieren Sie diese bitte vor dem 
Verschicken mit der Formatieroption des Desktops, um 
sicherzustellen, da"s die Diskette nicht defekt ist) und 
ausreichend frankiertem R"uckumschlag erhalten.

Mein Adresse ist: \\
Stefan M"unch, Borbergstr. 38, D-4700 Hamm 1

\index{Mailboxen}
Die komplette Version finden Sie als selbstentpackendes Archiv 
unter dem Namen ICONEDI.TOS auch in folgenden Mailboxen:
\begin{itemize}
 \item Galactica Hamm, Fido-Net 2:245/54, Tel. 049-02385-5811
 \item KAB Kamen, Fido-Net 2:245/52, Tel. 049-02307-40486
 \item Maus Unna, Tel. 049-02303-63102
\end{itemize}

Selbstverst"andlich erhalten Sie von mir und in diesen Mailboxen
jeweils die aktuellste Version.

Es ist ausdr"ucklich erlaubt, IconEdi in Form dieser gepackten
Datei in andere Mailboxen zu "uberspielen. Verboten ist jedoch,
das Archiv zu "andern, egal, wie. 

Au"serdem wird IconEdi auch in Zukunft auf einer Public-Domain-
Diskette der Zeitschrift ST-Computer erh"altlich sein.


%\input append_p.tex
\setcounter{chapter}{15}
\chapter{F"ur Programmierer} \index{Programmierer}
\index{Anhang P} \index{P-Anhang}
% P

\section{Verwendete Formate} \index{Verwendete Formate}

\subsection{Das IconEdi-Format} \index{IconEdi-Format}
IconEdi speichert Icons und Images in einem Format ab,
das auf den "ublichen GEM-Strukturen beruht, trotzdem aber
sehr leicht zu handhaben und zu implementieren ist.

Dateien dieses Formats haben einen Kopf, der mit einer
Kennung beginnt (einer 'magischen Zahl'), "uber Version
und Kopfl"ange Auskunft gibt und auf einen Objektblock 
zeigt. Ab dort ist dann weiterverzeigert. Die 
Verzeigerung bezieht sich immer auf den Dateianfang, und
die Adressen liegen im Motorola-Format vor. Als Dateiextension
sollte 'ICO' gew"ahlt werden.

Der Kopf sieht so aus (die Notation ist an die im Profibuch 
verwendete angelehnt):

\begin{tabular}{lll}
{\bf ICONHEADER} &                & \\
        &                         & \\
Offset  & Struktur                & \\ \hline
        &                         & \\
        & typedef struct          & \\
        & \{                      & \\
{\$}00  & char icon\_magic[4];    & /* enth"alt 'ICON'; \$49434F4E */ \\
{\$}04  & int icon\_version;      & /* Versionsnummer, wie beim TOS; Start \\
        &                         & ~~bei \$0000 (=so wie hier vorgestellt) */ \\   
{\$}06  & int icon\_headerlength; & /* L"ange des Headers in Words, Standard : 5 */ \\
{\$}08  & int *objectblock;       & /* Zeiger auf den Objektblock */ \\
        & \} ICONHEADER;          & \\

\end{tabular}

Die Entscheidung, was f"ur ein Objekt die Datei enth"alt, wird anhand
ob\_type getroffen und ob\_spec zeigt auf eine entsprechende 
Struktur. IconEdi unterst"utzt nur Icons und Images; prinzipiell
kann in diesem Format aber jedes Objekt gespeichert werden.

Informationen zu allen wichtigen GEM-Strukturen finden Sie im
'Atari ST Profibuch' aus dem Sybex-Verlag in Anhang C und in 
Kapitel 3.

Ein Beispielprogramm zum Lesen und Schreiben von Icons und Images
im IconEdi-Format finden Sie im Ordner DEMO im Ordner ZUBEHOER
des IconEdi-Pakets unter dem Namen DEMO.LST als 
GFA-Basic-3.x-Quelltext. Das Beispiel ist simpel gehalten und
ausf"uhrlich kommentiert, so da"s es leicht verstanden und in
andere Sprachen umgesetzt werden k"onnen sollte. 

\subsection{Mauszeiger-, Sprite- und F"ullmuster-Format}
\index{Mauszeigerformat} 
IconEdi speichert als Mauszeiger einen 
Mouse-Form-Definition-Block MFORM gefolgt von Maske
und Daten (abwechselnd je ein Word) ab. 

\index{Spriteformat} 
IconEdi speichert als Sprites einen 
Sprite-Definition-Block SDB gefolgt von Maske
und Daten ab. 

N"ahere Informationen entnehmen Sie bitte dem Profibuch.

\index{F"ullmusterformat}
Als F"ullmuster werden einfach 16 Words abgespeichert.

\end{appendix}



\ifx\bilder\undefined
\begin{theindex}

  \item 16 * 16, 23
  \item 32 * 32, 23
  \item 48 * 48, 23
  \item 64 * 64, 23

  \indexspace

  \item "Andern der Extensions, 24
  \item "Anderung der Dateinamenund Pfade, 31
  \item "Anderung der Tastaturbelegung f"ur die Men"ufunktionen, 33
  \item "Anderung der Tastaturbelegung in den Dialogen, 33
  \item "Anderungen am Handbuch, 1
  \item "Anderungen an der Resourcedatei, 32

  \indexspace

  \item "Uberweisung, 3

  \indexspace

  \item Accessories, 10
  \item Aktionspunkt, 20
  \item Alternative Benutzeroberfl"achen, 35
  \item Andere Gr"o"se, 22
  \item Andere Informationen im Arbeitsfenster, 8
  \item Anhang B, 35
  \item Anhang F, 37
  \item Anhang L, 39
  \item Anhang P, 41
  \item Anpassung an pers"onliche Bed"urfnisse, 31
  \item Anpassung an verschiedene TOS-Versionen, 31
  \item Anpassung der Tastaturbelegung, 32
  \item Arbeit sichern, 12
  \item Arbeitsbildschirm von IconEdi, 5
  \item Arbeitsraster, 5
  \item Aus Bildern importieren, 18
  \item Aus RSC-Dateien 'klauen', 20
  \item Ausgabe, 23
  \item Automatische Maskenerstellung, 18

  \indexspace

  \item B-Anhang, 35
  \item Bedienung, 4
  \item Begriffserkl"arungen, 35
  \item Beide l"oschen, 19
  \item Beispielanwendungen, 26
  \item Bekannte Fehler, 38
  \item Benutzeroberfl"achen, 35
  \item Bezugsquellen, 39

  \indexspace

  \item Clipboard, 8
  \item Copy, 16

  \indexspace

  \item Danksagungen, 3
  \item Dateierweiterung, 35
  \item Dateierweiterungen "andern, 24
  \item Daten, 2
  \item Daten l"oschen, 19
  \item Desktop, 35
  \item Disk-Men"u, 10
  \item Diverse Einstellungen, 24
  \item DR-RCS, 26
  \item Drehen um 180 Grad, 17
  \item Drehen um 270 Grad, 17
  \item Drehen um 90 Grad, 17

  \indexspace

  \item Einf"uhrung, 1
  \item Eingabe, 23
  \item Einstellungen in IconEdi, 32
  \item Einstellungen sichern, 12
  \item Ellipsen, 14
  \item Ellipsenb"ogen, 14
  \item Ende, 12
  \item Erg"anzungen zum Handbuch, 1
  \item Ergebnis-K"asten, 7
  \item Export, 19
  \item Extension, 35
  \item Extensions "andern, 24

  \indexspace

  \item F"ullmodus, 7
  \item F"ullmuster, 22
  \item F"ullmusterformat, 42
  \item F"ullmustertest, 20
  \item F-Anhang, 37
  \item Farbwechsel, 13
  \item Fehlende Dateien, 39
  \item Fehlermeldungen, 37
  \item Fenster von IconEdi, 5
  \item Fill-Maske, 18
  \item Fill\&Out-Maske, 18
  \item Fl"achen, 14
  \item Fragen zu IconEdi, 37
  \item Freihandzeichnen, 6
  \item Funktionen von IconEdi, 10

  \indexspace

  \item Garantie, 0
  \item Gef"ullte Ellipsen, 15
  \item Gef"ullte-Ellipsen-Segmente, 15
  \item GEM-Fenster, 5
  \item Gemini, 35
  \item Gemini-Maske, 18
  \item Glossar, 35
  \item Gr"o"se, 22
  \item Gr"o"se des Rasters, 24

  \indexspace

  \item Haben wollen, 3
  \item Haftungsausschlu"s, 0
  \item Hintergrund, 22
  \item Hotline, 3

  \indexspace

  \item Icon/Image, 16
  \item Icondaten, 2
  \item IconEdi, 3
  \item IconEdi-Format, 41
  \item IconEdi-Men"u, 10
  \item Iconmaske, 2
  \item Icons, 2
  \item Icons f"ur den neuen Atari-Desktop, 30
  \item Icons in Gemini einf"ugen, 27
  \item Icons in Programmen laden, 27
  \item Icons in Programmquelltext einf"ugen, 27
  \item Icons in RSC-Dateien "ubernehmen, 26
  \item Icons ohne Text und Buchstaben, 30
  \item Icontext, 22
  \item Image/Icon, 16
  \item Images, 35
  \item Import, 18
  \item In Bilder exportieren, 19
  \item Informationen im Arbeitsfenster, 8
  \item Informationsleiste, 5
  \item Installation, 4
  \item Interface, 27
  \item Invertieren, 16

  \indexspace

  \item Klemmbrett, 8
  \item Kopieren, 16
  \item Kopieren "uber die X-Achse, 17
  \item Kopieren "uber die Y-Achse, 17
  \item Kreisb"ogen, 14
  \item Kreise, 14
  \item Kuma-Resourceeditor, 27

  \indexspace

  \item L"oschen der Daten, 19
  \item L"oschen der Maske, 19
  \item L"oschen im Arbeitsraster, 13
  \item L"oschen von Daten und Maske, 19
  \item L-Anhang, 39
  \item Laden, 11
  \item Laden anderer Formate, 11
  \item Lieferumfang, 39
  \item Linien, 13

  \indexspace

  \item Mailboxen, 39
  \item Malkasten-Men"u, 13
  \item Maske, 2
  \item Maske l"oschen, 19
  \item Mausaktionspunkt, 20
  \item Mausfarbe, 21
  \item Maustest, 20
  \item Mauszeigerformat, 42
  \item Maximale Gr"o"se, 23
  \item Maxon-Sonderdisketten, 35
  \item Men"u Disk, 10
  \item Men"u IconEdi, 10
  \item Men"u Malkasten, 13
  \item Men"u Optionen, 22
  \item Men"u Spezial, 18
  \item Men"u Standard, 16
  \item Modus, 36
  \item MultiGEM, 36

  \indexspace

  \item neue Versionen, 3
  \item Neuladen, 10
  \item NRSC.PRG, 27

  \indexspace

  \item Optionen-Men"u, 22
  \item Outline-Maske, 18

  \indexspace

  \item P-Anhang, 41
  \item PEllipsen, 15
  \item PEllipsensegmente, 15
  \item Programmautor, 3
  \item Programmende, 12
  \item Programmierer, 41

  \indexspace

  \item Quit, 12

  \indexspace

  \item Radiergummi, 13
  \item Raster an/ausschalten, 24
  \item Rastergr"o"se, 24
  \item Rechtecke, 13
  \item RSC${\Rightarrow}$Icon, 20

  \indexspace

  \item S/M-Farbe, 21
  \item Scheiben, 14
  \item Scheibensegmente, 14
  \item Schluss, 12
  \item Schwarz, 13
  \item Scrapdirectory, 8
  \item Segmente gef"ullter Ellipsen, 15
  \item Shareware, 2
  \item Sharewaregeb"uhr, 3
  \item Sichern, 10
  \item Sichern anderer Formate, 11
  \item Sichern unter, 11
  \item Sonstige Informationen im Arbeitsfenster, 8
  \item Spezial-Men"u, 18
  \item Spezialelemente, 6
  \item Spiegeln an der X-Achse, 17
  \item Spiegeln an der Y-Achse, 17
  \item Sprite/Maustest, 20
  \item Spriteaktionspunkt, 20
  \item Spritefarbe, 21
  \item Spriteformat, 42
  \item Spritetest, 20
  \item ST-Computer, 36
  \item Standard-Men"u, 16

  \indexspace

  \item Tastaturfunktionen, 4
  \item Tauschen, 16
  \item Telefonischer Support, 3
  \item \TeX, 36
  \item Titelleiste, 5

  \indexspace

  \item Um 180 Grad drehen, 17
  \item Um 270 Grad drehen, 17
  \item Um 90 Grad drehen, 17
  \item Umlaute in Dateinamen, 11
  \item Undo, 16
  \item Updates, 3

  \indexspace

  \item Verlassen des Programms, 12
  \item Verschiebeleiste, 7
  \item Version, 10
  \item Verwendete Formate, 41
  \item Vom Klemmbrett, 12

  \indexspace

  \item Was kostet IconEdi ?, 3
  \item Was sind Icons ?, 2
  \item Wechsel, 16
  \item Wei"s, 13
  \item Weitere Informationen im Arbeitsfenster, 8
  \item Weitergabe von IconEdi, 39
  \item WERCS, 27

  \indexspace

  \item XCopy, 17
  \item XSpiegel, 17

  \indexspace

  \item YCopy, 17
  \item YSpiegel, 17

  \indexspace

  \item Zum Klemmbrett, 12
  \item Zur"uck, 16
  \item Zur Benutzung dieses Handbuchs, 1
  \item Zus"atzliche Informationen im Arbeitsfenster, 8
  \item Zwischenablage, 8

\end{theindex}
\else
\begin{theindex}

  \item 16 * 16, 27
  \item 32 * 32, 27
  \item 48 * 48, 27
  \item 64 * 64, 28

  \indexspace

  \item "Andern der Extensions, 29
  \item "Anderung der Dateinamenund Pfade, 36
  \item "Anderung der Tastaturbelegung f"ur die Men"ufunktionen, 38
  \item "Anderung der Tastaturbelegung in den Dialogen, 38
  \item "Anderungen am Handbuch, 1
  \item "Anderungen an der Resourcedatei, 37

  \indexspace

  \item "Uberweisung, 3

  \indexspace

  \item Accessories, 10
  \item Aktionspunkt, 24
  \item Alternative Benutzeroberfl"achen, 40
  \item Andere Gr"o"se, 27
  \item Andere Informationen im Arbeitsfenster, 9
  \item Anhang B, 40
  \item Anhang F, 42
  \item Anhang L, 44
  \item Anhang P, 46
  \item Anpassung an pers"onliche Bed"urfnisse, 36
  \item Anpassung an verschiedene TOS-Versionen, 36
  \item Anpassung der Tastaturbelegung, 37
  \item Arbeit sichern, 12
  \item Arbeitsbildschirm von IconEdi, 5
  \item Arbeitsraster, 6
  \item Aus Bildern importieren, 21
  \item Aus RSC-Dateien 'klauen', 22
  \item Ausgabe, 28
  \item Automatische Maskenerstellung, 20

  \indexspace

  \item B-Anhang, 40
  \item Bedienung, 4
  \item Begriffserkl"arungen, 40
  \item Beide l"oschen, 22
  \item Beispielanwendungen, 31
  \item Bekannte Fehler, 43
  \item Benutzeroberfl"achen, 40
  \item Bezugsquellen, 44

  \indexspace

  \item Clipboard, 8
  \item Copy, 18

  \indexspace

  \item Danksagungen, 3
  \item Dateierweiterung, 40
  \item Dateierweiterungen "andern, 29
  \item Daten, 2
  \item Daten l"oschen, 22
  \item Desktop, 40
  \item Disk-Men"u, 10
  \item Diverse Einstellungen, 30
  \item DR-RCS, 31
  \item Drehen um 180 Grad, 18
  \item Drehen um 270 Grad, 19
  \item Drehen um 90 Grad, 18

  \indexspace

  \item Einf"uhrung, 1
  \item Eingabe, 28
  \item Einstellungen in IconEdi, 37
  \item Einstellungen sichern, 12
  \item Ellipsen, 16
  \item Ellipsenb"ogen, 16
  \item Ende, 13
  \item Erg"anzungen zum Handbuch, 1
  \item Ergebnis-K"asten, 7
  \item Export, 22
  \item Extension, 40
  \item Extensions "andern, 29

  \indexspace

  \item F"ullmodus, 7
  \item F"ullmuster, 26
  \item F"ullmusterformat, 47
  \item F"ullmustertest, 23
  \item F-Anhang, 42
  \item Farbwechsel, 14
  \item Fehlende Dateien, 44
  \item Fehlermeldungen, 42
  \item Fenster von IconEdi, 5
  \item Fill-Maske, 20
  \item Fill\&Out-Maske, 20
  \item Fl"achen, 15
  \item Fragen zu IconEdi, 42
  \item Freihandzeichnen, 6
  \item Funktionen von IconEdi, 10

  \indexspace

  \item Garantie, 0
  \item Gef"ullte Ellipsen, 16
  \item Gef"ullte-Ellipsen-Segmente, 16
  \item GEM-Fenster, 5
  \item Gemini, 40
  \item Gemini-Maske, 20
  \item Glossar, 40
  \item Gr"o"se, 27
  \item Gr"o"se des Rasters, 29

  \indexspace

  \item Haben wollen, 3
  \item Haftungsausschlu"s, 0
  \item Hintergrund, 26
  \item Hotline, 3

  \indexspace

  \item Icon/Image, 17
  \item Icondaten, 2
  \item IconEdi, 3
  \item IconEdi-Format, 46
  \item IconEdi-Men"u, 10
  \item Iconmaske, 2
  \item Icons, 2
  \item Icons f"ur den neuen Atari-Desktop, 35
  \item Icons in Gemini einf"ugen, 32
  \item Icons in Programmen laden, 32
  \item Icons in Programmquelltext einf"ugen, 32
  \item Icons in RSC-Dateien "ubernehmen, 31
  \item Icons ohne Text und Buchstaben, 35
  \item Icontext, 25
  \item Image/Icon, 17
  \item Images, 40
  \item Import, 21
  \item In Bilder exportieren, 22
  \item Informationen im Arbeitsfenster, 9
  \item Informationsleiste, 6
  \item Installation, 4
  \item Interface, 32
  \item Invertieren, 18

  \indexspace

  \item Klemmbrett, 8
  \item Kopieren, 18
  \item Kopieren "uber die X-Achse, 18
  \item Kopieren "uber die Y-Achse, 18
  \item Kreisb"ogen, 15
  \item Kreise, 15
  \item Kuma-Resourceeditor, 32

  \indexspace

  \item L"oschen der Daten, 22
  \item L"oschen der Maske, 22
  \item L"oschen im Arbeitsraster, 15
  \item L"oschen von Daten und Maske, 22
  \item L-Anhang, 44
  \item Laden, 11
  \item Laden anderer Formate, 11
  \item Lieferumfang, 44
  \item Linien, 15

  \indexspace

  \item Mailboxen, 44
  \item Malkasten-Men"u, 14
  \item Maske, 2
  \item Maske l"oschen, 22
  \item Mausaktionspunkt, 24
  \item Mausfarbe, 24
  \item Maustest, 23
  \item Mauszeigerformat, 47
  \item Maximale Gr"o"se, 28
  \item Maxon-Sonderdisketten, 40
  \item Men"u Disk, 10
  \item Men"u IconEdi, 10
  \item Men"u Malkasten, 14
  \item Men"u Optionen, 25
  \item Men"u Spezial, 20
  \item Men"u Standard, 17
  \item Modus, 41
  \item MultiGEM, 41

  \indexspace

  \item neue Versionen, 3
  \item Neuladen, 11
  \item NRSC.PRG, 32

  \indexspace

  \item Optionen-Men"u, 25
  \item Outline-Maske, 20

  \indexspace

  \item P-Anhang, 46
  \item PEllipsen, 16
  \item PEllipsensegmente, 16
  \item Programmautor, 3
  \item Programmende, 13
  \item Programmierer, 46

  \indexspace

  \item Quit, 13

  \indexspace

  \item Radiergummi, 15
  \item Raster an/ausschalten, 29
  \item Rastergr"o"se, 29
  \item Rechtecke, 15
  \item RSC${\Rightarrow}$Icon, 22

  \indexspace

  \item S/M-Farbe, 24
  \item Scheiben, 15
  \item Scheibensegmente, 15
  \item Schluss, 13
  \item Schwarz, 14
  \item Scrapdirectory, 8
  \item Segmente gef"ullter Ellipsen, 16
  \item Shareware, 2
  \item Sharewaregeb"uhr, 3
  \item Sichern, 11
  \item Sichern anderer Formate, 12
  \item Sichern unter, 12
  \item Sonstige Informationen im Arbeitsfenster, 9
  \item Spezial-Men"u, 20
  \item Spezialelemente, 6
  \item Spiegeln an der X-Achse, 18
  \item Spiegeln an der Y-Achse, 18
  \item Sprite/Maustest, 23
  \item Spriteaktionspunkt, 24
  \item Spritefarbe, 24
  \item Spriteformat, 47
  \item Spritetest, 23
  \item ST-Computer, 41
  \item Standard-Men"u, 17

  \indexspace

  \item Tastaturfunktionen, 4
  \item Tauschen, 17
  \item Telefonischer Support, 3
  \item \TeX, 41
  \item Titelleiste, 5

  \indexspace

  \item Um 180 Grad drehen, 18
  \item Um 270 Grad drehen, 19
  \item Um 90 Grad drehen, 18
  \item Umlaute in Dateinamen, 12
  \item Undo, 17
  \item Updates, 3

  \indexspace

  \item Verlassen des Programms, 13
  \item Verschiebeleiste, 8
  \item Version, 10
  \item Verwendete Formate, 46
  \item Vom Klemmbrett, 12

  \indexspace

  \item Was kostet IconEdi ?, 3
  \item Was sind Icons ?, 2
  \item Wechsel, 17
  \item Wei"s, 14
  \item Weitere Informationen im Arbeitsfenster, 9
  \item Weitergabe von IconEdi, 44
  \item WERCS, 32

  \indexspace

  \item XCopy, 18
  \item XSpiegel, 18

  \indexspace

  \item YCopy, 18
  \item YSpiegel, 18

  \indexspace

  \item Zum Klemmbrett, 12
  \item Zur"uck, 17
  \item Zur Benutzung dieses Handbuchs, 1
  \item Zus"atzliche Informationen im Arbeitsfenster, 9
  \item Zwischenablage, 8

\end{theindex}

\fi

\end{document}


